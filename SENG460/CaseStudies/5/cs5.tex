%%%%%%%%%%%%%%%%%%%%%%%%%%%%%%%%%%%%%%%%%
% Short Sectioned Assignment
% LaTeX Template
% Version 1.0 (5/5/12)
%
% This template has been downloaded from:
% http://www.LaTeXTemplates.com
%
% Original author:
% Frits Wenneker (http://www.howtotex.com)
%
% License:
% CC BY-NC-SA 3.0 (http://creativecommons.org/licenses/by-nc-sa/3.0/)
%
%%%%%%%%%%%%%%%%%%%%%%%%%%%%%%%%%%%%%%%%%

%----------------------------------------------------------------------------------------
%	PACKAGES AND OTHER DOCUMENT CONFIGURATIONS
%----------------------------------------------------------------------------------------

\documentclass[paper=a4, fontsize=11pt]{scrartcl} % A4 paper and 11pt font size

\usepackage[T1]{fontenc} % Use 8-bit encoding that has 256 glyphs line to return to the LaTeX default
\usepackage[english]{babel} % English language/hyphenation
\usepackage{amsmath,amsfonts,amsthm} % Math packages

\usepackage{lipsum} % Used for inserting dummy 'Lorem ipsum' text into the template

\usepackage{sectsty} % Allows customizing section commands
\allsectionsfont{\centering \normalfont\scshape} % Make all sections centered, the default font and small caps

\usepackage{fancyhdr} % Custom headers and footers
\pagestyle{fancyplain} % Makes all pages in the document conform to the custom headers and footers
\fancyhead{} % No page header - if you want one, create it in the same way as the footers below
\fancyfoot[L]{} % Empty left footer
\fancyfoot[C]{} % Empty center footer
\fancyfoot[R]{\thepage} % Page numbering for right footer
\renewcommand{\headrulewidth}{0pt} % Remove header underlines
\renewcommand{\footrulewidth}{0pt} % Remove footer underlines
\setlength{\headheight}{13.6pt} % Customize the height of the header

\numberwithin{equation}{section} % Number equations within sections (i.e. 1.1, 1.2, 2.1, 2.2 instead of 1, 2, 3, 4)
\numberwithin{figure}{section} % Number figures within sections (i.e. 1.1, 1.2, 2.1, 2.2 instead of 1, 2, 3, 4)
\numberwithin{table}{section} % Number tables within sections (i.e. 1.1, 1.2, 2.1, 2.2 instead of 1, 2, 3, 4)

\setlength\parindent{0pt} % Removes all indentation from paragraphs - comment this line for an assignment with lots of text

%----------------------------------------------------------------------------------------
%	TITLE SECTION
%----------------------------------------------------------------------------------------

\newcommand{\horrule}[1]{\rule{\linewidth}{#1}} % Create horizontal rule command with 1 argument of height

\title{	
\normalfont \normalsize 
\textsc{university, school or department name} \\ [25pt] % Your university, school and/or department name(s)
\horrule{0.5pt} \\[0.4cm] % Thin top horizontal rule
\huge Investigation Case Study \\ % The assignment title
\horrule{2pt} \\[0.5cm] % Thick bottom horizontal rule
}

\author{Jordan Ell \\ V00660306} % Your name

\date{\normalsize\today} % Today's date or a custom date

\begin{document}

\maketitle % Print the title

%----------------------------------------------------------------------------------------
%	PROBLEM 1
%----------------------------------------------------------------------------------------

\section{The compromised user account}
Working as part of a presumed IT department at a company, this case study focuses on the 
exploits of employee Ulric Issac Darwin. Ulric has reported that his user id and subsequent
email address at the company he works at has been hacked and broken into. This case study
will focus on the ensuing example investigation that will take place follow the complaint
of Ulric. This report is broken up into four subsections which outline the process involved
in the investigation. First, initiation will outline the details of the investigation and
set up any background knowledge necessary to begin the investigation. Second, coordination
(done out of order for coherence of this report) will convey all coordination steps that should
be taken in order to conduct the investigation. Next, documentation will outline what I did,
what I found and my concluding recommendations. The documentation subsection will contain
all the evidence found as part of the investigation.

%------------------------------------------------

\subsection{Initiation}
The issue at hand, is that Ulric Issac Darwin believes this user ID and subsequent email address
has been compromised within the company he works for. This could have many different consequences
based on the types of information he originally had access to with his account or what he
was capable of. The obvious actor within this issue is Ulric himself. Some of the lesser known
actors in this issue may involve Ulric's boss who will be primarily concerned with what is found
out about this account. The IT department is also an actor as they are responsible for the
security of information within the company and could be having to deal with potential security
leaks because of the account being compromised. Stakeholders of the company such as employees
are also actors in this situation as their personal information or business information may
now be compromised because of the account. Finally, every employee could be considered an actor
as the compromise of the account could be an inside job within the company. Key assets of the
investigation as previously stated are all the employee, company, or customer information is 
now potentially in jeopardy as the account may have access to this information. Also, the IT
infrastructure in both technical and policy is an asset as these could either help or hinder
the investigation and may have to be changed based on what is found.\\

For this investigation, two forms of resources will be used (although one of them is not
actually preformed in this case study). For one, an interview will be conducted with Ulric in
order to determine his story for how the account might have been compromised. Here, an
attempt to establish a chronological nature of events will be made. Once this is done,
Ulric's story will be broken down to check for inconsistencies or lies. This interview
may lead to further interviews with other employees or stakeholders depending on Ulric's
answers and demeanor. This interview is not actually conducted for this case study as
Ulric is made up. The second resource used for this investigation will be technical logs
stored and monitored by the company. These logs include Active Directory authentication, Outlook
web access, and PIXIE logs. 

%------------------------------------------------

\subsection{Coordination}
For the investigation team, I would use a team of IT team members from inside the company as
well as an HR representative. The IT team members are there to check technical log and check
for any technical irregularities that may arise during the investigation. These team members are
key as they will have the expertise to find quantitative evidence to support any claims that
the investigation team may find. The HR representative is there to be apart of the interview of
Ulric and any interviews that may emanate from that. The HR employee has made a career of dealing
with people and should become a great assistance in any interrogations that may occur.\\

In terms of notification and stakeholder information, the most obvious person who comes to mind
is Ulric himself. He should be kept notified of any findings in the investigation as they may
cause positive or negative repercussions for his future career. Ulric's boss as well as any 
superior employees whose reputation, job, or own employees should also be notified to the
investigations findings. Ulric's boss may have to take immediate action against individuals
from the findings. Messages may need to be relayed to the rest of the employees at the company
to avoid future complications that may be found.

%------------------------------------------------

\subsection{Documentation}
I would outline this section in the following ways: what I did, what I found, and future
recommendations for Ulric and the company. For the purposes of this case study, I did not
interview Ulric, however, given a real investigation, this would have occurred as well as any
follow up interviews of company employees that might have been necessary. What I did do however,
is parse through all the given email and Outlook logs that were given for this case study. These
logs were able to provide all the email send and received got eh account in question as well as
remote access to the account from outside the company network.\\ 

What I found was that Ulric was using his email account for personal use to email and receive
emails from outside the company addresses including personal contacts as well as personal
information such as email notifications from restaurants and other services. I have also found
that Ulric's email account was accessed from outside the company at times where Ulric is known
to be at work. These accesses were all from the same machine and what is believed to be
his wife's laptop. Therefore, I believe one of two things has occurred. Either one, Ulric's
wife has unauthorized access to his email account, or two, the email account is shared between
the two. If she has unauthorized access, it seems as though she has found him in a personal
relationship outside of his marriage with the owner of the email address ladybird1@yahoo.net.
If the email address is shared it seems as though Ulric is lieing to us to avoid being caught
in his extra marital affair.\\

Where either of these stories are true, I have a few recommendations to make for the company.
Number one is that a work email address should never be used for personal or non work related
items such as chain mail, emailing non employees or non customers, or personal financing etc.
Two, email accounts should never be shared with anyone outside or even inside the company. The
risk of outside party members is obvious with non authorized access to company data, while not
sharing with internal company members is to retain accountability between members of the company
and who is taking which actions when investigations are being done. Finally, the company should
have a better authentication process in place for remote access of these accounts. This may
be done by only authorizing certain machines into the network or notifying personnel when their
account is being used for the first time by an unauthorized machine.

%------------------------------------------------

\subsubsection{Evidence}
The evidence for the claims listed above can all be found inside the logs provided to the
investigation team. From the Outlook Web Access logs, we know that Ulric's account is being
accessed by a third party from outside the network at ip address 76.193.130.252. These accesses
are also at times when Ulric is known to be at work, thus it must be a third party. We can 
also see from email header provided that this IP address matches the user using the account
unknsub@shaw.com from who Ulric has frequent emails with. Next, we can see that Ulric has
emailed ladybird1@yahoo.net with the subject line Happy Tuesday Baby, which we can assume
is not pleasent news to his wife who is actually the owner of unknsub@shaw.com. From these
pieces of evidence we were able to draw up the two scenarios presented above. Either Ulric's
wife has unauthorized access to his email and has caught him in a relationship outside his
marriage (in which case his email has technically been hacked) or two, Ulric has shared
his email with his wife, which is far more likely given her frequent activities on his account,
and is now lieing to try and presumably protect himself from his wife (in which case he has
not been hacked). 


\end{document}