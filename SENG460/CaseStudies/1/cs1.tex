%%%%%%%%%%%%%%%%%%%%%%%%%%%%%%%%%%%%%%%%%
% Short Sectioned Assignment
% LaTeX Template
% Version 1.0 (5/5/12)
%
% This template has been downloaded from:
% http://www.LaTeXTemplates.com
%
% Original author:
% Frits Wenneker (http://www.howtotex.com)
%
% License:
% CC BY-NC-SA 3.0 (http://creativecommons.org/licenses/by-nc-sa/3.0/)
%
%%%%%%%%%%%%%%%%%%%%%%%%%%%%%%%%%%%%%%%%%

%----------------------------------------------------------------------------------------
%	PACKAGES AND OTHER DOCUMENT CONFIGURATIONS
%----------------------------------------------------------------------------------------

\documentclass[paper=a4, fontsize=11pt]{scrartcl} % A4 paper and 11pt font size

\usepackage[T1]{fontenc} % Use 8-bit encoding that has 256 glyphs line to return to the LaTeX default
\usepackage[english]{babel} % English language/hyphenation
\usepackage{amsmath,amsfonts,amsthm} % Math packages

\usepackage{lipsum} % Used for inserting dummy 'Lorem ipsum' text into the template

\usepackage{sectsty} % Allows customizing section commands
\allsectionsfont{\centering \normalfont\scshape} % Make all sections centered, the default font and small caps

\usepackage{fancyhdr} % Custom headers and footers
\pagestyle{fancyplain} % Makes all pages in the document conform to the custom headers and footers
\fancyhead{} % No page header - if you want one, create it in the same way as the footers below
\fancyfoot[L]{} % Empty left footer
\fancyfoot[C]{} % Empty center footer
\fancyfoot[R]{\thepage} % Page numbering for right footer
\renewcommand{\headrulewidth}{0pt} % Remove header underlines
\renewcommand{\footrulewidth}{0pt} % Remove footer underlines
\setlength{\headheight}{13.6pt} % Customize the height of the header

\numberwithin{equation}{section} % Number equations within sections (i.e. 1.1, 1.2, 2.1, 2.2 instead of 1, 2, 3, 4)
\numberwithin{figure}{section} % Number figures within sections (i.e. 1.1, 1.2, 2.1, 2.2 instead of 1, 2, 3, 4)
\numberwithin{table}{section} % Number tables within sections (i.e. 1.1, 1.2, 2.1, 2.2 instead of 1, 2, 3, 4)

\setlength\parindent{0pt} % Removes all indentation from paragraphs - comment this line for an assignment with lots of text

%----------------------------------------------------------------------------------------
%	TITLE SECTION
%----------------------------------------------------------------------------------------

\newcommand{\horrule}[1]{\rule{\linewidth}{#1}} % Create horizontal rule command with 1 argument of height

\title{	
\normalfont \normalsize 
\textsc{University of Victoria, Software Engineering} \\ [25pt] % Your university, school and/or department name(s)
\horrule{0.5pt} \\[0.4cm] % Thin top horizontal rule
\huge Central Credit Union Case Study \\ % The assignment title
\horrule{2pt} \\[0.5cm] % Thick bottom horizontal rule
}

\author{Jordan Ell\\
V00660306} % Your name

\date{\normalsize\today} % Today's date or a custom date

\begin{document}

\maketitle % Print the title

%----------------------------------------------------------------------------------------
%	PROBLEM 1
%----------------------------------------------------------------------------------------

\section{Case Study Problem}

Plan the implementation of Enterprise Security Architecture (ESA) at Central 
Credit Union by answering the following questions: What organizational 
structure should be used to support the development of the ESA (1.1), what will
be the key components to Central Credit Unions ESA (1.2), who in the organization
should be ultimately responsible for information security and why (1.3), what
will be some CFS's is the ESA's development (1.4), and where should Central 
Credit Union starts and what are the priorities (1.5).

%------------------------------------------------

\subsection{ESA Organizational Structure}
For the organizational structure of the ESA development, I would create a
security committee as well as a security team with a liaison between the 
two groups. The committee
will consist of all employees at the "C-level", being the corporate security
officer, chief information officer, chief operating officer, and chief
financial officer. The team will consist of each manager listed in the given
organizational figure. The liaison between the two security groups would be 
the corporate security officer Linda McDougall. Linda is in the best position
to be the liaison for these groups as she already is involved in company security
and has the ability to create effective change at both the managerial level
as well as the "C-level". If a budget permitted, it may be favourable to have
a new hire become the liaison and whose sole purpose is the organization of the
ESA as well as communicating change between the higher and lower powers. This
may have the consequence of the ESA becoming stronger as a whole job is dedicated
to the ESA as opposed to having it tacked onto pre-existing work.

%------------------------------------------------

\subsubsection{Key Components}
The key components can be broken down into two sets: technical and plans. The 
technical components which are key will be internet security. As the Central
Credit Union offers extremely high risk services and highly targeted services
(online banking), their online security will be the highest of their concerns. 
Other technical concerns which are lower risk but still technical concerns non
the less would be internal technical structures such as data confidentiality
and accessibility to employees.\\

As for key components of plans, seeing as the Central Credit Union offers 
services to end users, training of new security components to those end users
will be critical to ensuring the regular business flow continues. Regular
interval testing will also be key to ensure the ESA is moving forward in its
development and to ensure that standards are being met as well as intermediate
and end goals. Finally, interval training will be key as employees inside of
the Central Credit Union must become aware of new security standards and 
understand their impact on personal job performance as well as business goals.


%------------------------------------------------

\subsubsection{Responsibility}
Ultimately, I believe that the person responsible for the outcome of the ESA
should be Linda McDougall. Linda as being designated the liaison between the
security committee and security team is in the best position to be fully aware
of the project's developments and status. She also is in the best position as
being at the "C-level" to effectively change the processes and thoughts of the
company when it comes to the ESA. Another potential candidate would be the CEO
Bob Smith. With the CEO becoming the responsible party for the ESA, the business
end of the ESA is brought more to the forefront and may be taken more seriously
by internal structures inside the company.\\

A side note on this matter is of course that everyone in the company is
ultimately responsible for the success and outcome of the ESA. Any employee could
be responsible for a security breach and a failure of the ESA and thus assumes
some level of responsibility.


%------------------------------------------------

\subsubsection{Critical Success Factors}
Some critical success factors include the following. Linda's ability to
communicate and cause actual change among the "C-level" employees who are
responsible for the business goals of the rest of the company. Lower employees
ability to adapt and effectively follow new ESA rules and procedures which
will be soon put in place. Interval testing of the ESA to ensure that the
development of the ESA is moving forward toward end goals laid out by the
security committee and team. In with interval testing is being able to
properly evaluate the company's base line of their current security measures
in order to properly identify weaknesses both technical and procedural in order
to set meaningful goals.\\

I believe a success factor of any organization which is commonly overlooked
is the ability not only to document proper procedures, but also make it easily
available to all employees where needed and easy to understand. Too many times
are procedures either not documented or are hidden or difficult to find 
certain procedures for a given employee or task.


%------------------------------------------------

\subsubsection{Where To Start and Priorities}
The Central Credit Union should start by forming the appropriate committee and
security team as well as finding an appropriate sponsor without whom the project
would be taken lightly. Once the organizational aspects have been assigned, the
ESA should work to establishing a base line of where security currently stands
as well as determining goals for the near and distance future. Base lines can
be determined with current penetration testing and internal surveys to the
employees. Goals can be used to determine how interval testing should be 
accomplished in order to test and meet end goals.

\end{document}