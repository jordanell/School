%%%%%%%%%%%%%%%%%%%%%%%%%%%%%%%%%%%%%%%%%
% Short Sectioned Assignment
% LaTeX Template
% Version 1.0 (5/5/12)
%
% This template has been downloaded from:
% http://www.LaTeXTemplates.com
%
% Original author:
% Frits Wenneker (http://www.howtotex.com)
%
% License:
% CC BY-NC-SA 3.0 (http://creativecommons.org/licenses/by-nc-sa/3.0/)
%
%%%%%%%%%%%%%%%%%%%%%%%%%%%%%%%%%%%%%%%%%

%----------------------------------------------------------------------------------------
%	PACKAGES AND OTHER DOCUMENT CONFIGURATIONS
%----------------------------------------------------------------------------------------

\documentclass[paper=a4, fontsize=11pt]{scrartcl} % A4 paper and 11pt font size

\usepackage[T1]{fontenc} % Use 8-bit encoding that has 256 glyphs line to return to the LaTeX default
\usepackage[english]{babel} % English language/hyphenation
\usepackage{amsmath,amsfonts,amsthm} % Math packages

\usepackage{lipsum} % Used for inserting dummy 'Lorem ipsum' text into the template

\usepackage{sectsty} % Allows customizing section commands
\allsectionsfont{\centering \normalfont\scshape} % Make all sections centered, the default font and small caps

\usepackage{fancyhdr} % Custom headers and footers
\pagestyle{fancyplain} % Makes all pages in the document conform to the custom headers and footers
\fancyhead{} % No page header - if you want one, create it in the same way as the footers below
\fancyfoot[L]{} % Empty left footer
\fancyfoot[C]{} % Empty center footer
\fancyfoot[R]{\thepage} % Page numbering for right footer
\renewcommand{\headrulewidth}{0pt} % Remove header underlines
\renewcommand{\footrulewidth}{0pt} % Remove footer underlines
\setlength{\headheight}{13.6pt} % Customize the height of the header

\numberwithin{equation}{section} % Number equations within sections (i.e. 1.1, 1.2, 2.1, 2.2 instead of 1, 2, 3, 4)
\numberwithin{figure}{section} % Number figures within sections (i.e. 1.1, 1.2, 2.1, 2.2 instead of 1, 2, 3, 4)
\numberwithin{table}{section} % Number tables within sections (i.e. 1.1, 1.2, 2.1, 2.2 instead of 1, 2, 3, 4)

\setlength\parindent{0pt} % Removes all indentation from paragraphs - comment this line for an assignment with lots of text

%----------------------------------------------------------------------------------------
%	TITLE SECTION
%----------------------------------------------------------------------------------------

\newcommand{\horrule}[1]{\rule{\linewidth}{#1}} % Create horizontal rule command with 1 argument of height

\title{	
\normalfont \normalsize 
\textsc{University of Victoria, Software Engineering} \\ [25pt] % Your university, school and/or department name(s)
\horrule{0.5pt} \\[0.4cm] % Thin top horizontal rule
\huge ABC Security Policy Case Study \\ % The assignment title
\horrule{2pt} \\[0.5cm] % Thick bottom horizontal rule
}

\author{Jordan Ell\\
V00660306} % Your name

\date{\normalsize\today} % Today's date or a custom date

\begin{document}

\maketitle % Print the title

%----------------------------------------------------------------------------------------
%	PROBLEM 1
%----------------------------------------------------------------------------------------

\section{Information Security Policy Research}

This document will outline background research and analysis performed in order
to develop a policy to protect information on laptops against cold boot attacks
with stolen laptops. This document will: analyze the issue (i), define the 
purpose (ii), identify and analyze risks (iii), search for policy and technical
controls (iv), analyze impacts of controls (v), identify stakeholders (vi), 
and define compliance metrics (vii). The final policy developed from these
findings can be found attached to this initial document.

%------------------------------------------------

\subsection{Analyze The Issue}

The prominent issue at hand is two fold. First off, the physical issue of
laptops being stolen is a concern. While hard drive (HD) encryption prevents
some forms of data loss, it in no way addresses the initial issue of laptops
being stolen or misplaced by employees. The second issue is the HD encryption
does not protect against cold boot attacks which are becoming a serious concern
as these attacks are relatively cheap and quick to perform. Also, depending on 
how the company deals with the encryption keys, a single key being discovered may 
lead to further data loss on other systems.With these two
issues, ABC faces great financial loss potential in both virtual data and
physical property, both of which are damaging to the company's reputation and
potential customers.

%------------------------------------------------

\subsection{Define The Purpose}

The purpose of this policy development task is to mitigate the loses outlined
in the previous section. Trying to prevent laptop theft as well as HD encryption
key discovery which may be a result of theft or carelessness. Both of these 
preventions are used to stop physical cost as well as potential data loss leading
to large scale damage for financially and reputability of ABC.

%------------------------------------------------

\subsection{Identify and Analyze Risk, Threats, and Vulnerabilities}

The chief vulnerability of the cold boot attack is the access to a powered
or recently powered laptop's DRAM. If the DRAM is accessible and is powered
or recently powered, the HD encryption key and information inside of it can
be stolen. There are multiple threats to this vulnerability both from stolen
or unattended laptops. An external device may be plugged into the laptop such
as an external HD or USB thumb drive which can be used for gaining access to
the DRAM information. These type of threats can be done relatively quickly, so
even an unattended laptop is a threat to this vulnerability. Laptop theft is
another threat as with a stolen machine, the DRAM can physical be removed and 
cooled in order to allow the hacker more time and freedom when it comes to
obtaining information from the device. The risk being run with these attacks
are both physical cost to ABC with laptop replacements as well as potential
data loss which could lead to enormous damage costs. These risks also scale 
with potential higher ranked employees as they may have access to higher levels
of confidential information about the company or customers. 

%------------------------------------------------

\subsection{Identify and Analyze Risk, Threats, and Vulnerabilities}

The chief vulnerability of the cold boot attack is the access to a powered
or recently powered laptop's DRAM. If the DRAM is accessible and is powered
or recently powered, the HD encryption key and information inside of it can
be stolen. There are multiple threats to this vulnerability both from stolen
or unattended laptops. An external device may be plugged into the laptop such
as an external HD or USB thumb drive which can be used for gaining access to
the DRAM information. These type of threats can be done relatively quickly, so
even an unattended laptop is a threat to this vulnerability. Laptop theft is
another threat as with a stolen machine, the DRAM can physical be removed and 
cooled in order to allow the hacker more time and freedom when it comes to
obtaining information from the device. The risk being run with these attacks
are both physical cost to ABC with laptop replacements as well as potential
data loss which could lead to enormous damage costs. These risks also scale 
with potential higher ranked employees as they may have access to higher levels
of confidential information about the company or customers. 

%------------------------------------------------

\subsection{Policy and Technical Controls}

\begin{enumerate}
\item BIOS Password - Even though the system administration department does not 
agree, a BIOS password will ensure that the boot order of a laptop cannot be
changed which would prevent the laptop from being powered on to an external HD.
\item High refresh DRAM - Ensuring that only high refresh DRAM is purchased and
installed in laptops helps defeat the reboot style cold boot attack as the RAM
will be cleared quicker upon power removal.
\item BitLocker DRAM Flush - If possible, ensure that BitLocker flushes the DRAM
on laptop shutdown, this ensures the encryption key is removed from the DRAM and
cannot be stolen on a quick reboot.
\item Laptop lock - Laptops must be tethered to an immovable object when being 
left unattended. This prevents laptop theft which can lead to data loss.
\item Laptop storage - Laptops must be in a locked cabinet or other closed storage
device while being left unattended for short to medium periods of time. This 
prevents theft as well as external devices being attached for quick data
theft of the DRAM.
\item Sensitive Data - Keep sensitive data off of personal laptops and only
stored on authorized network storage. This keeps stolen encryption keys from
being able to access local sensitive data.
\item Mandatory shutdown - If a laptop is not to be used for an extended period
of time (end of the work day), it must be powered off. Being powered off prevents
encryption keys from being in DRAM for very long.
\item Block ports - Physically block USB and other ports which would otherwise
enable harmful devices to be attached to the laptop. This would stop attackers
from running malicious hardware / software in an external fashion.
\end{enumerate}

%------------------------------------------------

\subsection{Analyze Impact of Controls}

Out of the 4 technical controls listed above (1,2,3,6), BIOS password has the
most inconvenience for the end user. As explained in the interview it would
interrupt the current path management software as an employee must enter a
password to allow patching as opposed to automation. High refresh DRAM and
BitLocker flush would be hidden from the user but may cost slightly more money.
Network storage for sensitive data may cause laptops users slight inconvenience
as additional steps may have to be taken to access company data, as well as
may lead to additional costs of server data farms and their security measures.\\

Out of the 4 policy controls listed above (4,5,7,8), mandatory shutdowns are
potentially the most harmful. Depending on current boot times of the devices,
having to power on a laptop more than once a day can be a serious issue which
may lead to less productivity of employees. Blocking physical ports can also
be damaging depending on how currently move data from one device to the other.
(The possibility of losing USB storage devices.) Laptop locks and storage should
not damage the user's experience at all and should more or less just be 
considered responsible use of the laptops.

%------------------------------------------------

\subsection{Identify Stakeholders}

A stakeholder of this policy is anyone who can be affected positively or 
negatively by the new policy. Because of this, every employee at ABC who uses
a laptop is labelled as a stakeholder of this new information security policy.
Aside from laptop users, the IT department is also a large stakeholder as they
will likely be in charge of ensuring technical controls are up to date and 
being administered properly. HR will likely play a large role in training and
possibly enforcing policy controls company wide. The high level employees of
ABC will be seen a large stakeholders as security is a business plan and must
have "buy-in" from top level employees in order to be successful.

%------------------------------------------------

\subsection{Define Compliance Metrics}

In order to enforce the new information security policy, both the technical
and policy controls must be in compliance with potential random spot checks
as well as initial training and setup. The IT department can be tasked with
ensuring both new and old laptops are installed or updated with new technical
controls. These technical controls can also be enforced by conducting possible
random checks of department laptops across ABC. A set of standard checks could
be devised by which current laptops are tested. For policy controls, individual
departments can enforced regulation by simply spotting non compliant procedures
that happen from time to time by its workers. (Laptops being left unattended 
or not being properly secured with locks.) This ensures that departments and
employees are self regulated which may cause greater awareness of security 
policy. The IT or HR departments may also wish to have meetings with other
departments to go over new security policy or have refresher sessions.

\end{document}