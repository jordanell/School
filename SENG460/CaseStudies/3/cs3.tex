%%%%%%%%%%%%%%%%%%%%%%%%%%%%%%%%%%%%%%%%%
% Short Sectioned Assignment
% LaTeX Template
% Version 1.0 (5/5/12)
%
% This template has been downloaded from:
% http://www.LaTeXTemplates.com
%
% Original author:
% Frits Wenneker (http://www.howtotex.com)
%
% License:
% CC BY-NC-SA 3.0 (http://creativecommons.org/licenses/by-nc-sa/3.0/)
%
%%%%%%%%%%%%%%%%%%%%%%%%%%%%%%%%%%%%%%%%%

%----------------------------------------------------------------------------------------
%	PACKAGES AND OTHER DOCUMENT CONFIGURATIONS
%----------------------------------------------------------------------------------------

\documentclass[paper=a4, fontsize=11pt]{scrartcl} % A4 paper and 11pt font size

\usepackage[T1]{fontenc} % Use 8-bit encoding that has 256 glyphs line to return to the LaTeX default
\usepackage[english]{babel} % English language/hyphenation
\usepackage{amsmath,amsfonts,amsthm} % Math packages

\usepackage{graphicx}
\usepackage{array}

\usepackage{lipsum} % Used for inserting dummy 'Lorem ipsum' text into the template

\usepackage{sectsty} % Allows customizing section commands
\allsectionsfont{\centering \normalfont\scshape} % Make all sections centered, the default font and small caps

\usepackage{fancyhdr} % Custom headers and footers
\pagestyle{fancyplain} % Makes all pages in the document conform to the custom headers and footers
\fancyhead{} % No page header - if you want one, create it in the same way as the footers below
\fancyfoot[L]{} % Empty left footer
\fancyfoot[C]{} % Empty center footer
\fancyfoot[R]{\thepage} % Page numbering for right footer
\renewcommand{\headrulewidth}{0pt} % Remove header underlines
\renewcommand{\footrulewidth}{0pt} % Remove footer underlines
\setlength{\headheight}{13.6pt} % Customize the height of the header

\numberwithin{equation}{section} % Number equations within sections (i.e. 1.1, 1.2, 2.1, 2.2 instead of 1, 2, 3, 4)
\numberwithin{figure}{section} % Number figures within sections (i.e. 1.1, 1.2, 2.1, 2.2 instead of 1, 2, 3, 4)
\numberwithin{table}{section} % Number tables within sections (i.e. 1.1, 1.2, 2.1, 2.2 instead of 1, 2, 3, 4)

\setlength\parindent{0pt} % Removes all indentation from paragraphs - comment this line for an assignment with lots of text

%----------------------------------------------------------------------------------------
%	TITLE SECTION
%----------------------------------------------------------------------------------------

\newcommand{\horrule}[1]{\rule{\linewidth}{#1}} % Create horizontal rule command with 1 argument of height

\title{	
\normalfont \normalsize 
\textsc{University of Victoria, Software Engineering} \\ [25pt] % Your university, school and/or department name(s)
\horrule{0.5pt} \\[0.4cm] % Thin top horizontal rule
\huge Security Threat and Risk Assessment Case Study \\ % The assignment title
\horrule{2pt} \\[0.5cm] % Thick bottom horizontal rule
}

\author{Jordan Ell \\ V00660306} % Your name

\date{\normalsize\today} % Today's date or a custom date

\begin{document}

\maketitle % Print the title

%----------------------------------------------------------------------------------------
%	PROBLEM 1
%----------------------------------------------------------------------------------------

\section{Security Threat and Risk Assessment}
This case study focuses on the scenario of the University of Victoria's IT department conducting a security threat and risk assessment
review. The University acknowledges that their IT infrastructure has potential security related issues but does not know how to
properly handle the situation. This case study will walk through the first three phases of a security threat and risk 
assessment of the University of Victoria's IT infrastructure. These three steps are: preparation, asset identification, and
threat assessment. The end result of this study is not to provide recommendations towards the University, but to
lay the frame work for which these recommendations can be based upon.

%------------------------------------------------

\subsection{Preparation Phase}
The preparation phase involves the preparation for assessment of risk inside the University's IT infrastructure. I recommend
the following steps be taken in preparation for the assessment to follow.

\begin{itemize}
  \item Identify stakeholders - Stakeholders may include but are not limited to: students, faculty, professors, IT employees,
  corporate sponsors, board of directors, student employees, university's reputation, etc \ldots
  \item Reviewing current system attacks - Software and hardware attacks are happening constantly around the world. In order
  to properly combat the attacks being made, you should be up to date on how attackers are successful against other systems
  and what steps could have been take to avoid the attack or prevent it.
  \item Reviewing current STRA procedures - Knowing how security threat and risk assessments are being done with
  other companies or by other contractors allows us to keep the same levels of security standards as the rest of the world.
  \item Understand the System Being Evaluated - Discover exactly what the University is using for their IT infrastructure. This
  type of preparation allows the STRA evaluation to know ahead of time where potential leaks in security are more likely to
  occur as well as avoid unwanted research time for problems that are not present in the tools  being used.
\end{itemize}

With these steps, a STRA evaluator will be well prepared to deal with the problems that the University if facing in
regards to their IT infrastructure. Preparation is an important step in the STRA process as it is more than likely
to save time and money down the road where larger problems are likely to occur. After these preparations have
taken place, time frames and target goals should be set for the later stages of the STRA examination. These steps
should define a scope of the project as well as help produce a work plan for future steps.


%------------------------------------------------

\subsubsection{Asset Identification Phase}
The following table is used by the STRA team to identify assets within the Univerisity's IT infrastructure. This table will
later be used in the threat assessment phase of the STRA process as outlined in Section~\ref{subsec:tap}.

\begin{center}
\resizebox{\textwidth}{!}{%
    \begin{tabular}{| l | l | l | l | p{2cm} | p{2cm} | p{2cm} | p{2cm} |}
    \hline
    Class & Category & Group & Univ IT Dept & Confid. & Avail Int & Avail op & Integrity \\
    People & Employees & Univ IT Dept & Univ IT Dept & & High & High & \\
    People & Employees & Univ Staff & Univ IT Dept & & Medium & Medium & \\
    People & University & Students & Univ IT Dept & & Medium & Medium & \\
    People & University & Professors & Univ IT Dept & Medium & Medium & Medium & \\
    Tangible & Information & Univ IT Dept & Univ IT Dept & High & High & High & High \\
    Tangible & Hardware & Univ IT Dept & Univ IT Dept & & Medium & Medium & High \\
    Tangible & Software & Univ IT Dept & Univ IT Dept & & Medium & Medium &  \\
    Tangible & Firmware & Univ IT Dept & Univ IT Dept & & Medium & Medium & Medium \\
    Tangible & Facilities & Computer Store & Univ IT Dept & & Low & Low &  \\
    Tangible & Facilities & Campus Computers & Univ IT Dept & & Medium & Medium &  \\
    Tangible & Facilities & Library & Univ IT Dept & & Medium & Medium &  \\
    Intangible & University & Reputation & Univ IT Dept & High & High & High & High \\
    \hline
    \end{tabular}}
\end{center}

This previous table represents my asset identification phase of the University of Victoria's IT infrastructure. Here
I have identified many tangible and people class assets to the university. These assets for the purpose of this
case study are very broad and each asset could be further broken down into sub categories for a real
STRA evaluation of the IT infrastructure.

%------------------------------------------------

\subsubsection{Threat Assessment Phase}
\label{subsec:tap}
Finally, the threat assessment of the University's IT infrastructure can be completed here in this section.
In the following table, the assets being assessed have been provided for the purposes of this threat 
assessment phase. The values in the likelihood, gravity, and one of: confid., avail., or integrity have been filled
in for the purposes of this case study.

\begin{center}
\resizebox{\textwidth}{!}{%
    \begin{tabular}{| l | l | l | p{2cm} | p{2cm} | p{2cm} | p{2cm} | p{2cm} | p{2cm} |}
    \hline
    ID No. & Class & Agent & Event & Likelihood & Gravity & Confid. & Avail. & Integrity \\
    31 & Deliberate & Individuals & Network Exploitation & Medium & High &  & High &  \\ 
    32 & Deliberate & Individuals & Social Engineering & Medium & Medium & Medium &  &  \\
    40 & Deliberate & Groups/Individuals & Delete/Destroy Records & High & Medium &  &  & High \\
    41 & Deliberate & Groups/Individuals & Corrupt Data & Medium & Medium &  &  & Medium \\
    42 & Deliberate & Groups/Individuals & Encrypt Files & Medium & Medium &  & Medium &  \\
    43 & Deliberate & Groups/Individuals & Misconfigure Software & High & Medium &  & High & High \\
    44 & Deliberate & Groups/Individuals & Misconfigure Hardware & Medium & High &  & High &  \\
    46 & Deliberate & Wannabees & DOS Attack & Medium & Medium &  & Medium &  \\
    47 & Deliberate & Wannabees & Malicious Code & Low & Medium & Low & Low & Low \\
    48 & Deliberate & Wannabees & File Corruption & Medium & Medium &  &  & Medium \\
    60 & Deliberate & Script Kiddies & Web Defacement & Low & Low &  &  & Very Low \\
    94 & Deliberate & Hackers & Identity Theft & Medium & High & High &  &  \\
    103 & Deliberate & Companies & Patent Infringement & Low & Medium & Low &  &  \\
    106 & Deliberate & Individuals & Spam & High & Low & Medium &  &  \\
    108 & Deliberate & Individuals & Unauthorized Use & Medium & Medium & Medium & Medium & Medium \\
    118 & Accidents & Individuals & Inaccurate Data Input & High & Low &  &  & Medium \\
    121 & Accidents & Office Staff & Delete Files & High & Low &  &  & Medium \\
    122 & Accidents & Office Staff & Spill Liquids & Low & Low &  & Very Low & Very Low \\
    126 & Accidents & Cleaning Staff & Unplug Equipment & Medium & Low &  & Low &  \\
    127 & Accidents & Individuals & Lose Laptop & High & High & Very High &  &  \\
    129 & Accidents & Data Entry Clerks & Data Entry Errors & High & Low &  &  & Medium \\
    130 & Accidents & Database Admin & Operating Errors & High & Low &  & Medium & Medium \\
    131 & Accidents & Companies & Software Bugs & High & Medium & High &  & High \\
    132 & Accidents & Organizations & Software Integration Errors & High & Medium &  & High &  \\
    133 & Accidents & Individuals & Coding Errors & High & Low &  &  & Medium \\
    134 & Accidents & Individuals & Software Configuration Errors & High & Low &  & Medium &  \\
    135 & Accidents & Companies & Design Flaws & High & Medium & High &  & High \\
    136 & Accidents & Companies & Equipment Malfunction & Medium & Medium &  & Medium &  \\
    137 & Accidents & Organizations & Installation Errors & Medium & Low &  & Low &  \\
    138 & Accidents & Individuals & Hardware Configuration Errors & Medium & Low &  & Low &  \\
    139 & Accidents & Individuals & Operator Errors & High & Low &  & Medium & Medium \\
    147 & Accidents & Individuals & Inadvertent Misuse & High & Low &  & Medium & Medium \\
    156 & Accidents & Equipment Operators & Disrupt Production & High & Medium &  & High &  \\
    208 & Natural Hazards & Dust & Media Contamination & Low & Low &  & Very Low & Very Low \\
    \hline
    \end{tabular}}
\end{center}

The table created above was created by first entering the likelihood and gravity values based on the provided
tables and ruberic in the case study. Next the value of risk was assessed using the likelihood and gravity conversion
table provided for the case study.At this point in the STRA process, the table above would now be
used to make recommendations to the university's 
IT Dept staff. However, for the purposes of this case study, we stop here and go no further in the STRA process.


\end{document}