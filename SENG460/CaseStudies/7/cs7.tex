%%%%%%%%%%%%%%%%%%%%%%%%%%%%%%%%%%%%%%%%%
% Short Sectioned Assignment
% LaTeX Template
% Version 1.0 (5/5/12)
%
% This template has been downloaded from:
% http://www.LaTeXTemplates.com
%
% Original author:
% Frits Wenneker (http://www.howtotex.com)
%
% License:
% CC BY-NC-SA 3.0 (http://creativecommons.org/licenses/by-nc-sa/3.0/)
%
%%%%%%%%%%%%%%%%%%%%%%%%%%%%%%%%%%%%%%%%%

%----------------------------------------------------------------------------------------
%	PACKAGES AND OTHER DOCUMENT CONFIGURATIONS
%----------------------------------------------------------------------------------------

\documentclass[paper=a4, fontsize=11pt]{scrartcl} % A4 paper and 11pt font size

\usepackage[T1]{fontenc} % Use 8-bit encoding that has 256 glyphs line to return to the LaTeX default
\usepackage[english]{babel} % English language/hyphenation
\usepackage{amsmath,amsfonts,amsthm} % Math packages

\usepackage{lipsum} % Used for inserting dummy 'Lorem ipsum' text into the template

\usepackage{sectsty} % Allows customizing section commands
\allsectionsfont{\centering \normalfont\scshape} % Make all sections centered, the default font and small caps

\usepackage{fancyhdr} % Custom headers and footers
\pagestyle{fancyplain} % Makes all pages in the document conform to the custom headers and footers
\fancyhead{} % No page header - if you want one, create it in the same way as the footers below
\fancyfoot[L]{} % Empty left footer
\fancyfoot[C]{} % Empty center footer
\fancyfoot[R]{\thepage} % Page numbering for right footer
\renewcommand{\headrulewidth}{0pt} % Remove header underlines
\renewcommand{\footrulewidth}{0pt} % Remove footer underlines
\setlength{\headheight}{13.6pt} % Customize the height of the header

\numberwithin{equation}{section} % Number equations within sections (i.e. 1.1, 1.2, 2.1, 2.2 instead of 1, 2, 3, 4)
\numberwithin{figure}{section} % Number figures within sections (i.e. 1.1, 1.2, 2.1, 2.2 instead of 1, 2, 3, 4)
\numberwithin{table}{section} % Number tables within sections (i.e. 1.1, 1.2, 2.1, 2.2 instead of 1, 2, 3, 4)

\setlength\parindent{0pt} % Removes all indentation from paragraphs - comment this line for an assignment with lots of text

%----------------------------------------------------------------------------------------
%	TITLE SECTION
%----------------------------------------------------------------------------------------

\newcommand{\horrule}[1]{\rule{\linewidth}{#1}} % Create horizontal rule command with 1 argument of height

\title{	
\normalfont \normalsize 
\textsc{University of Victoria, Software Engineering} \\ [25pt] % Your university, school and/or department name(s)
\horrule{0.5pt} \\[0.4cm] % Thin top horizontal rule
\huge Case Study 7: Ethical Hacking \\ % The assignment title
\horrule{2pt} \\[0.5cm] % Thick bottom horizontal rule
}

\author{Jordan Ell \\ V00660306} % Your name

\date{\normalsize\today} % Today's date or a custom date

\begin{document}

\maketitle % Print the title

%----------------------------------------------------------------------------------------
%	PROBLEM 1
%----------------------------------------------------------------------------------------
This report will outline five weaknesses or vulnerabilities seen or discussed
in the etchical hacking guest lecture given n SENG460 on Friday March 1st 
2013. Each of the five sections below will outline a weakness or vulnerability,
how the weakness or vulnerability could be exploited, and a countermeasure
to ensure protection agains the weakness.

\section{Social Engineering}
It is often said that social engineering working not because the attackers are
smart, but because the targets are so dumb. Social engineering is the act of
using people, not nesisarily through technology, to manipulate standard
procedures or thinking in order to gain access to protected or secure
information or other material. The weakness of people is often exploited by
engineering situations where people behaive more trusting than they normally 
would. The weakness can be exploited in many ways such as finding personelle
in online directories and impersonating an authratative role to them over
the phone in order to manipulate them. There was also the case of simply
taking advantage of people's laziness by sneaking into a factory during a shift
change when a high volume of employees pass through security unchecked. 
Technology can also be used by send around fake emails asking for people to
use authentication credentials on fake websites (this takes advantage of 
people's trusting demenor).\\

The major countermeasure to social engineering is simply awareness to social
engineering attacks and proper training on technical procedures. For example,
a company should know that IT support will never ask for their password
directly or any form of extra authentication to websites. Employees can be
trained to be less trusting so that if they get a phone call from an 
impersonator, proper steps can be taken to insure that not volatile information
if given out.

\section{Phising}
Phising is a type of exploit once again that preys on people's tendency to 
be trusting in nature. Phising is a way of using fake or impostered website
and emails in an attempt to fool a user into giving up authentication 
credentials. An example of this may be to create the website favebook.com,
a mispelling of facebook, then create a website that looks exactly like the
login page to facebook in an attempt to get users to sign in and thus give
up their credentials to this fake website. A real world example of how phising
is used to exploit people is the UVic banking information issues. About a year
ago UVic had a security breach in which bannking information was given out,
phising attackers took this opportunity to send out emails from Canadian banks
asking users to sign in on a phising website in our to insure their information
was secure. Once the user signed into this fake site, their accounts would
be compromised. \\

Again, the real countermeasure to this attack is end user knowledge and
training. A user should know right off the bat that a Bank or other
instituion which requires authentication will never ask you to sign in to
secure your credentials. Emails and other messaging services provide links
all the time, however, links can be manipulated in HTML in order to look
like a secure path. This being said, URLs should also be inspected when fraud
is a possibility. The user should have a sense for what a phising attemp 
will look like compared to the legitimate website.

\section{Baiting}
Baiting is yet another weakness in people that is exploited through the use
of technology. Baiting, in a sense, is to offer a user something curious or
extravagent in order to have them be manipulated and preform some service
for the attacker. There have been a couple real world scenarios where baiting
has been used. A bank employee was trying to get his or her boss to understand
the dangers of USB keys used by other employees. To help show the danger, the
employee simply created multiple USB sticks with auto running malware. Once
the USB stick was inserted into a machine, the malware would go to work. Next,
the employee simply left a bunch of the USB sticks out and scattered around
the bank. The next day several employees, out of curiosity, plugges the sticks
into their machines. This is a perfect example of baiting by taking advantage
of curiosity. The next example is that of a conference with a QR code. A,
in this case pretend, attacker simply put up a QR code around a conference
and waited for people to scan it. Once scanned the user would be taken to
a website which attempted to execute malicious software. The QR code worked
and baited in several conference goers. This is another example of real world
baiting techniques that rely on people's curiosity. \\

A countermeasure towards baiting would involve corporate policiy for an 
orginization or just personal control over curiosity. A corporation should
have policy in place to prevent unknown devices from being plugged into their
systems or networks as well as random suggested websites from being used (or
QR codes from our example) with company equipment. As for personal use, 
end users simply need to understand the dangers in what seems like abandonded
technology. Once a device is inserted into a machine, there is no telling what
could be run from a malicious coder. 


\section{Web Application Vulernabilities}
With the internet becoming more of an accessibility tool for more and more
applications, web base applications can often be target for their security 
holes and potential to be linked to database or other system components, since
by their nature these applications tend to be the link between front and
back end systems. A recent study by SANS indicates that more than 60\% of
all attacks on the internet target web applications. Tools such as websploit
can be used to scan and find these vulernabilities in order to generate
custom attacking code for the targetted website. Examples of these types of
weaknesses and vulnerabilities being exploited include the following.
SQL injections, by manipulating URLS or input boxes on a website, an attacker
can send malicious SQL queries to a database which enables access to otherwise
secure information being stored. Cross-site scripting allows for the running
of scripts in the user's browser which can lead to the stealing of cookies
or redirects to other websites which are used to take over a user's session. \\

While each web application vulnerability may require unique processes or
countermeasure in order to stop malaicious attackers from gaining access
to unwanted information, there are a few general purpose steps that can
be taken. Always insure input from and user given command is sanatized. Many
attacks begin with passing malicious code inside of user given commands or
required inputs. Stay up to date on new vulnerabilities such as Java (as many
web apps use Java as a backend) in order to avoid new attacks through
patching of your middleware and backend facilities.

\section{Sniffing / MITM / ARP Spoofing}
Sniffing, man in the middle attacks, and ARP spoofing all rely on an attacker
getting access to your network and packet flow. Sniffing involves the actual
detection and reading of packets as they are sent over a network. Man in 
the middle attacks involve an attacked coming between a user and their 
destination while exploiting weaknesses such as ARP spoofing and
MAC flooding. An example of potential sniffing exploits would be to sniff out
VOIP packets on a network in order to record user's internet base phone calls.
The attacker could listen to what was said and find damning information. An
example man in the middle attack could involve ARP spoofing in order to
redirect a user to a malicious website which appears to be their end
destination, such as creating a banking phising website and using redirects
to make it appear like the legitimate website orgiinally requested by the end
user. Sniffing and other man in the middle attacks are extremely dangers
as the tools to use them are often easily availible and user friendly enough
for a basic understanding to go a long way in attacking someone. (Wireshark,
ettercap, etc..) \\

There are a couple of ways to deter man in the middle based attacks or
packet sniffing. For one, access to the network is required for these types
of attacks and thus, access should only be given to trusted employees or
users of the network. This will stop third party users from coming onto the 
network in the first place in attempting to sniff packets without
permission. The second way to stop these types of attacks is with secure
network connections with encryption. Packet sniffers often cannot unencrypt
network packets which are encrypted from source to destination. This ensures
that even if an attacker got access to the network, the packets being sniffed
would be unreadable for their content.


%----------------------------------------------------------------------------------------

\end{document}