%%%%%%%%%%%%%%%%%%%%%%%%%%%%%%%%%%%%%%%%%
% Short Sectioned Assignment
% LaTeX Template
% Version 1.0 (5/5/12)
%
% This template has been downloaded from:
% http://www.LaTeXTemplates.com
%
% Original author:
% Frits Wenneker (http://www.howtotex.com)
%
% License:
% CC BY-NC-SA 3.0 (http://creativecommons.org/licenses/by-nc-sa/3.0/)
%
%%%%%%%%%%%%%%%%%%%%%%%%%%%%%%%%%%%%%%%%%

%----------------------------------------------------------------------------------------
%	PACKAGES AND OTHER DOCUMENT CONFIGURATIONS
%----------------------------------------------------------------------------------------

\documentclass[paper=a4, fontsize=11pt]{scrartcl} % A4 paper and 11pt font size

\usepackage[T1]{fontenc} % Use 8-bit encoding that has 256 glyphs line to return to the LaTeX default
\usepackage[english]{babel} % English language/hyphenation
\usepackage{amsmath,amsfonts,amsthm} % Math packages

\usepackage{lipsum} % Used for inserting dummy 'Lorem ipsum' text into the template

\usepackage{sectsty} % Allows customizing section commands
\allsectionsfont{\centering \normalfont\scshape} % Make all sections centered, the default font and small caps

\usepackage{fancyhdr} % Custom headers and footers
\pagestyle{fancyplain} % Makes all pages in the document conform to the custom headers and footers
\fancyhead{} % No page header - if you want one, create it in the same way as the footers below
\fancyfoot[L]{} % Empty left footer
\fancyfoot[C]{} % Empty center footer
\fancyfoot[R]{\thepage} % Page numbering for right footer
\renewcommand{\headrulewidth}{0pt} % Remove header underlines
\renewcommand{\footrulewidth}{0pt} % Remove footer underlines
\setlength{\headheight}{13.6pt} % Customize the height of the header

\numberwithin{equation}{section} % Number equations within sections (i.e. 1.1, 1.2, 2.1, 2.2 instead of 1, 2, 3, 4)
\numberwithin{figure}{section} % Number figures within sections (i.e. 1.1, 1.2, 2.1, 2.2 instead of 1, 2, 3, 4)
\numberwithin{table}{section} % Number tables within sections (i.e. 1.1, 1.2, 2.1, 2.2 instead of 1, 2, 3, 4)

\setlength\parindent{0pt} % Removes all indentation from paragraphs - comment this line for an assignment with lots of text

%----------------------------------------------------------------------------------------
%	TITLE SECTION
%----------------------------------------------------------------------------------------

\newcommand{\horrule}[1]{\rule{\linewidth}{#1}} % Create horizontal rule command with 1 argument of height

\title{	
\normalfont \normalsize 
\textsc{University of Victoria, Software Engineering} \\ [25pt] % Your university, school and/or department name(s)
\horrule{0.5pt} \\[0.4cm] % Thin top horizontal rule
\huge Pony Entertainment and Conficker Case Study \\ % The assignment title
\horrule{2pt} \\[0.5cm] % Thick bottom horizontal rule
}

\author{Jordan Ell \\ V00660306} % Your name

\date{\normalsize\today} % Today's date or a custom date

\begin{document}

\maketitle % Print the title

%----------------------------------------------------------------------------------------
%	PROBLEM 1
%----------------------------------------------------------------------------------------

\section{Pony Entertainment}
This case study will focus on the scenario presented in class of Pony Entertainment. Pony
has identified that security compromises have been attacked and identified by anonymous
sources online. The intrusion has targeted areas such as DMZ, database servers, and
sensitive user data. I have been tasked with improving security through a monitoring and
logging system. The following subsections describe the preparation I will take, the
logging systems which will be used, and the monitoring processes which will be put into
place.

%------------------------------------------------

\subsection{Preparation}
The first step of preparation is to analyze the weaknesses of the current infrastructure
as well as where previous attacks or exploits have been found to occur. This will create
a type of baseline for what needs to be solved. This analysis also includes ensuring
that the internal infrastructure of networks and company policy are secure. While
protecting against external threats, internal threats remain if this is not looked at
or taken care of.

The second step of the preparation phase involves people inside the company. Here we
are looking for 3 targets. One, ensure that we have employees who can handle any of 
the new procedures we throw in. This includes both technical people (perhaps a dedicated
security IT team), as well as HR people for policy training if the technical items
we wish to put in involve all employees at a procedural level. (This could be logging
of individual machines or self reporting of suspicious activities.) Two, we need to 
ensure the company has sufficient funding as this procedure may involve large scale time
effort of employees (which will effect salaries). Without funding this project is dead
in the water. Finally, we need management buy in from the managerial and corporate
levels. Security is ultimately a business strategy and without the consent of the
business operators, upgrading the monitoring process will ultimately fail.


%------------------------------------------------

\subsubsection{Logging}
For logging I have identified several new procedures which should help Pony Entertainment
notice intrusions before being notified, notice these activities in real time, and be
able to assist law enforcement in legal prosecuting actions. My first step is to log
network activity. This can be accomplished by installing either Netflow or Wireshark
at all external gateways that control traffic from external to internal facilities
that Pony may have. Netflow will allow suspicious activity over the network to be
logged in a concise manor of data download and upload to external computers. This allows
monitors to check suspicious ports being accessed or if certain IPs are overloading the
systems, etc.

The next step in logging is to log internal servers to Pony and their activities. This
can be accomplished through the use of Syslog (or Windows equivalent given Windows 
servers). Syslog logs activity on a per server basis, tracking items such as log in 
attempts, failed commands etc. This will allow Pony to check for suspicious activity
for persons who have direct access to the internal servers. This will also improve
accountability for actions taken in case of emergency or in case of errors on said
servers. 

Pony also identified databases as a critical point of attack for intruders. I suggest
a combination of the previously mention Syslog as well as a separate database log. A 
database log keeps track of connections to the database as well as specific actions
taken while inside of the database. This can allow the tracking of suspicious activity
inside any given database such as editing user accounts, changing passwords, or tampering
with monetary values. Aside from the database logs, I also suggest some form of safe data 
storage both on and offsite. Onsite may be as simple as installing CCTVs inside of server
rooms while offsite may be using companies such as Iron Mountain and performing audits
on a regular basis.

Finally, if employees of Pony entertainment as given personal laptops which are used
outside the office (chance of being lost or stolen), I suggest local event logging
software on each laptop. The local event logs should be sent to a server inside the
company. These logs will allow monitoring of stolen or misplaced laptops to ensure
that they are not being used in a malicious way. 


%------------------------------------------------

\subsubsection{Monitoring}
Through the tools mention in the section above, we have created an environment which
will be tracked in almost ever important action. We now need policies and tools for
allowing Pony employees to monitor the information that is being collected in an
efficient manor. For monitoring of logs being generated, I suggest two items. The first 
is email monitoring. In most Unix environments, logs can be emailed to technical
staff (IT or security team) for further analysis. This can be easily set up through
the use of the cron tab. The cron tab can be configured to email these employees on a
regular (I suggest daily) basis for analysis. However, these logs can be enormous
and time consuming to go through. This is why I suggest my second item with the use
of log parsing tools such as awk or grep or other commercial tools. These tools will
allow the emailing of only certain parts of a log file. This may be only security
notifications of warning or higher. These tools allow for employee analysis of only
critical notifications that are coming from the log.

The second monitoring process involves the network logs from Netflow. Again the procedure
above of emailing and parsing should be used, however, and additional step should
be taken by Pony employees. The team in charge of monitoring should keep an updated
black list of dangerous IPs which should not be allowed to access the network. This will
allow blocking some malicious attackers from repeated attempts at breaking into Pony 
systems.

Finally, some steps should be taken to ensure these new logging and monitoring systems
are preforming as expected. One, an initial baseline should be taken (as previously 
mentioned) as well as regular audits to create a sense of then and now. See where
our security measures have come from as well as where they are heading to ensure that
we are meeting goals as well as obtaining results. Lastly is to ensure accurate
time is being kept between all logs and servers. The time of servers is critical to
establishing a chain of events and may be of critical importance to any legal
prosecution which may be caused by attackers.


%----------------------------------------------------------------------------------------
%	PROBLEM 2
%----------------------------------------------------------------------------------------

\section{Conficker}
This section of the case study will focus on the worm known as Conficker. Here, I
will outline the background of the worm (what is it, how does it work, etc.), describe
3 major worms prior to Conficker, and any new worms.


%------------------------------------------------

\subsection{Background}
Conficker is a computer worm which targets the Windows operating system. Conficker uses
flaws in the Windows security architecture to gain access to administrator passwords
which are then used to add the infected machine to a botnet. Conficker was detected 
in November of 2008 and infected millions of computers, some of which were government,
business, and home machine in over 200 countries.

Conficker used malware techniques in order to become such a successful worm. While most
malware techniques and procedures are known to anti-virus researchers, Conficker's 
advantage was that is used so many malware techniques which made is hard to stop as
all would need to be dealt with. Conficker's developers are also thought to be monitoring
anti-virus patches and efforts in order to consistently update the worm in order to 
propagate it further. Conficker propagates itself by taking advantage of a network
service (MS08-067) in the Windows platform. Once this weakness is exploited, the infected
machine becomes apart of the Conficker botnet and can be used and controller by the
creators of Conficker.

Once s security breach has been accomplished, Conficker (one of its variants A,B,C,D,E)
uses tools such as bugger overflows to execute shellcode, HTTP servers, attaching DLLs
to running processes, sharing itself by infecting removable media or over network
connections. Through these procedures, the computer's virus can be updated as well as
controller by the creators. Conficker also takes measure to protect itself such as 
encrypting its network payloads, re-configuring system restore points, and disabling
services such as automatic updates and security centers.

Direct consequences of the virus include: account lockout policies being reset, user
accounts being locked out, anti virus websites being locked out, domain controllers,
slow response, slow local network, Microsoft services not working. Some of the known
damage caused by Conficker include: causing the French Navy to not be able
to download flight plans, The United Kingdom Ministry of Defense being infected,
the armed forces of Germany having infected machines, and the Greater Manchester
Police being disconnected from their internal police network.


%------------------------------------------------

\subsection{Prior Worms}
Here, I will describe three computer worms that were prior to Conficker: Badtrans, 
Code Red, and the Kak worm. I will describe how these worms spread as well as
what damage each worm was known to inflict.

Badtrans was a computer worm which attacked Microsoft Windows machines. BadTrans was
primarily distributed by email because of a known issues with older email programs.
The problem was that these email clients would oftentimes run attachments as soon
as the email message was viewed as opposed to waiting for the user to execute 
an attachment. Once Batrans has been run, it propagated itself by emailing itself out
to all email addresses it could fine. Badtrans main attack was the installation
of key logging software. This software logs the keystrokes of individuals and then
emails this information out to the creators.

Code Red was a computer worm which attacked machines running Microsoft's IIS web
server. Code Red exploited a security vulnerability known as a buffer overflow.
The worm essentially just used a very long string in order to allow the worm to
execute any code it wished on the machine. The damage caused by Code Red was a defacement
of the server's web sites. The website would display the text "HELLO! Welcome to 
http://www.worm.com! Hacked By Chinese!". Infected machines were also know to launch
DOS attacks.

The Kak worm was a computer worm which attacked machines using JavaScript and propagates
itself by using Outlook Express. The worm was most known for causing an automatic
shutdown of the infected computer on the first day of every month at 5:00 pm. The machine
also often displayed fake alert messages to the user. 


%------------------------------------------------

\subsection{Post Conficker}
Computer worms continue to be an issue post Conficker. More and more sophisticated
malware techniques are being introduced and taken advantage of on a daily basis. 
Arguably a bigger worm than Conficker, Stuxnet, has been even more damaging by taking
control of PLC units in industrial settings such as power plants. Computer viruses
will always continue to be a problem as it is near, if not completely impossible
to write always secure code.


%----------------------------------------------------------------------------------------

\end{document}