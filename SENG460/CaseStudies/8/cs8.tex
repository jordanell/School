%%%%%%%%%%%%%%%%%%%%%%%%%%%%%%%%%%%%%%%%%
% Short Sectioned Assignment
% LaTeX Template
% Version 1.0 (5/5/12)
%
% This template has been downloaded from:
% http://www.LaTeXTemplates.com
%
% Original author:
% Frits Wenneker (http://www.howtotex.com)
%
% License:
% CC BY-NC-SA 3.0 (http://creativecommons.org/licenses/by-nc-sa/3.0/)
%
%%%%%%%%%%%%%%%%%%%%%%%%%%%%%%%%%%%%%%%%%

%----------------------------------------------------------------------------------------
%	PACKAGES AND OTHER DOCUMENT CONFIGURATIONS
%----------------------------------------------------------------------------------------

\documentclass[paper=a4, fontsize=11pt]{scrartcl} % A4 paper and 11pt font size

\usepackage[T1]{fontenc} % Use 8-bit encoding that has 256 glyphs line to return to the LaTeX default
\usepackage[english]{babel} % English language/hyphenation
\usepackage{amsmath,amsfonts,amsthm} % Math packages

\usepackage{lipsum} % Used for inserting dummy 'Lorem ipsum' text into the template

\usepackage{sectsty} % Allows customizing section commands
\allsectionsfont{\centering \normalfont\scshape} % Make all sections centered, the default font and small caps

\usepackage{fancyhdr} % Custom headers and footers
\pagestyle{fancyplain} % Makes all pages in the document conform to the custom headers and footers
\fancyhead{} % No page header - if you want one, create it in the same way as the footers below
\fancyfoot[L]{} % Empty left footer
\fancyfoot[C]{} % Empty center footer
\fancyfoot[R]{\thepage} % Page numbering for right footer
\renewcommand{\headrulewidth}{0pt} % Remove header underlines
\renewcommand{\footrulewidth}{0pt} % Remove footer underlines
\setlength{\headheight}{13.6pt} % Customize the height of the header

\numberwithin{equation}{section} % Number equations within sections (i.e. 1.1, 1.2, 2.1, 2.2 instead of 1, 2, 3, 4)
\numberwithin{figure}{section} % Number figures within sections (i.e. 1.1, 1.2, 2.1, 2.2 instead of 1, 2, 3, 4)
\numberwithin{table}{section} % Number tables within sections (i.e. 1.1, 1.2, 2.1, 2.2 instead of 1, 2, 3, 4)

\setlength\parindent{0pt} % Removes all indentation from paragraphs - comment this line for an assignment with lots of text

%----------------------------------------------------------------------------------------
%	TITLE SECTION
%----------------------------------------------------------------------------------------

\newcommand{\horrule}[1]{\rule{\linewidth}{#1}} % Create horizontal rule command with 1 argument of height

\title{	
\normalfont \normalsize 
\textsc{University of Victoria, Software Engineering} \\ [25pt] % Your university, school and/or department name(s)
\horrule{0.5pt} \\[0.4cm] % Thin top horizontal rule
\huge Case Study 8: Application Security \\ % The assignment title
\horrule{2pt} \\[0.5cm] % Thick bottom horizontal rule
}

\author{Jordan Ell \\ V00660306} % Your name

\date{\normalsize\today} % Today's date or a custom date

\begin{document}

\maketitle % Print the title

%----------------------------------------------------------------------------------------
%	PROBLEM 1
%----------------------------------------------------------------------------------------

\section{Application Security of Biergarten}

This case study focuses on application security for the town of Biergarten. This town is currently involved in overseeing
a large change to their technical infrastructure. This new infrastructure is replacing an existing legacy system and
can be broken down into two large main components with some over lap between the two. The first component of the new
system is designed to improve how the rest of the world views Biergarten. This involved tourism, and the large beer industry
that resides in the town and promoting these services to the world. The second component is a web portal for residents 
of the town. This portal will allow municipal services to be transacted such as payment of property taxes, civil
worker information, and more.\\

The rest of this case study will be laid out as follows. First, an attempt to classify the components common to both
systems as well as mutually exclusive components will be made using the S.T.R.I.D.E methodology. Justification will be
given for each classification as well. Next, the effects of cross site scripting will be investigated with the vulnerabilities
detected in the previous section. This investigation will use the D.R.E.A.D framework as its basis for concern.

%----------------------------------------------------------------------------------------

\section{S.T.R.I.D.E}
In this section, I will outline the potential security risks using the S.T.R.I.D.E methodology in regards to the common
components of the two web portals as well as the components specific to each web portal. I will give justification
for each classification made.

\subsection{Common Components}
Here a list is presented of all S.T.R.I.D.E issues in the common components to each system.

\begin{itemize}
\item Site-specific templates will be user for overall look and feel. Using templates site wide opens the website to potential
spoofing identity attacks such as phising. Once the attacker is able to duplicate the website template, emails with links to
look a like sites can be sent out in an attempt to spoof the identity of the original site and lure in victims with passwords
or other information.
\item User creation and administration will be performed centrally. This can be identified as a tampering risk. Since all users
and riskier yet, administrators, are stored in a central database, once access has been enabled on a potential attacker, all
user information is now at risk. Worse yet, the attacker can create an administrator account for themselves and hide amongst the
other user accounts. This would also be classified as an elevation of privileges.
\item Users will be assigned roles and will have access to different information across the site based on said roles. This can
be identified as spoofing and elevation of privilege. An attacker may be able to change his or her role inside of the database
and thus gain access to higher level information inside the system or may be able to change said information. The spoofing comes
from the fraudulent role and access an attacker would be able to achieve.
\item A user friendly web interface for uploading general content. This can be identified as a tampering and spoofing risk as
well as open the door to cross site scripting (XSS). Here, since the content being uploaded is general, any content including
malicious software could be uploaded to the server. This type of content can be used to gain access to private or hidden information
and can also communicate with the outside world in order to send the information to other malicious websites or servers. This 
all of course runs the risk of data tampering and spoofing of identities to gain further access to data.
\item Cookies will be stored to allow users to remain logged in for extended periods of time. This can again be identified
as tamper and spoofing. The spoofing here comes from session hijacking. If a user leaves themselves logged in on a public 
machine, as person coming along can use their account for an elevated privileges they may have. This can also be use to elevate
another user's privileges. The tampering comes again from session hijacking and being able to change the currently logged in
user's information.
\item Finally, because of the nature of all these common components being centrally located, denial of service (DoS) attacks
can be employed against the entire system. These attacks can leave the entire website down as the traffic load is not distributed
accross multiple machines.
\end{itemize}

\subsection{Tourist Information Portal}
Here a list is presented of all S.T.R.I.D.E issues in the tourist information portal.

\begin{itemize}
\item The portal is expected to be based on a content management system with a friendly web based interface. This raises the
same concerns as the general content uploading with tampering and spoofing. However, this time, the security issues with
the content managements system (CMS) are now placed on a third party, the developers of the CMS. This takes responsibility
away from Biergarten but places concern of outside security implementations. And of course, this still can be labeled as potential
tampering and spoofing with the content being uploaded. The content can once again be used for malicious purposes in both
data access, identify spoofing and elevation of privileges.
\item The sites will contain dedicated administrators but will also have the hopes of local businesses managing their
own content. This can be identified as an information disclosure issue. The problem here is that the site will contain a large
amount of administrators or users will high privileges. This being the case it may be easy for information to fall into the
wrong hands of an administrator who should not have the information. This of course is also associated with elevation of
privileges in that administrators will have access to a wide assortment of information.
\item Users will be able to access and edit their own personal information while some administrators information will be
accessible for editing by the business users. Here elevation of privilege is the main cause for concern while XSS is also
to be dealt with. The fact that users have access to so many types of information, even in some cases administrator information
for businesses is a large concern. The user may act as a user with elevated privileges without even having to make an attack.
This also concerns the tampering of data the user has access to. XSS is also an issue as there appears to be a large number
of input forms for the user and each form is potential of XSS.
\end{itemize}

\subsection{Municipal Service Site}
Here a list is presented of all S.T.R.I.D.E issues in the municipal service site.

\begin{itemize}
\item Each department will have a separate subsection while functionality will be added with the core framework. This can
be classified as a denial of service threat. The problem here is that many subsections are all hosted and operated through
a core framework. If that core framework was to be attacked using a denial of service attack, all subsections would be lost
and unreachable. The DoS will cause many systems to go down while only attacking a single server.
\item Departments will have privileged users to edit content while civil employees will have read only access. This can be
classified as elevation of privilege and tamping. The problem here again is that there are too many administrators on a single
web server. These administrators have the capability to tamper with information stored on the central database as well as
elevate the privileges of other users inside the system. A reduced number of administrators would help combat this issue.
\item Users will be able to perform civil actions such as pay property taxes using the web portal. This can be classified as
an information disclosure and denial of service threat. Here, the users are supplied very private and critical information
to the system. If any attacker got a hold of this information, the attacked user may run into legal and financial issues from
the disclosed information. And again, since the servers are centralized, the denial of service attack would bring down many
critical components to the users such as paying taxes on time which may leave the town crippled.
\item The website must be able to provide own information to users while hide it from others. This type of roles system 
can be vulnerable to elevation of privilege and data tampering attacks. Since the users are assigned roles which give them
access to certain items of information, any user obtaining elevated privileges may be able to change data at will of another
user or the system itself.
\end{itemize}

%----------------------------------------------------------------------------------------

\section{XSS and D.R.E.A.D}

\subsection{XSS}

The main concern with this website in terms of cross site scripting is the number of input forms a regular user has access
to across the website. Each of these forms has the potential to be involved in a cross site scripting attack on the
website and on the user. Input must be validated for each form to be able to stop this kind of attack. In terms of
XSS being triggered as part of emails targeted at user and administrators of the web portal, the availability of
information is a large concern. As stated in the requirements for the website, a regular user has access to all kinds of
information of other users such as email. This will allow attackers to only target users of the website site from the
get go. A second concern is the template of the website. Since the whole website shares a common template, a phising website
will be created with relative ease by potential attackers. Here an email can be sent to users asking them to authenticate
to the phising website and now the attackers have user names and passwords for the website. From here, the attacker can
take advantage of the many elevation of privilege issues raised in the previous section in order to create data tampering
or other malicious intents.

\subsection{D.R.E.A.D}

Damage Potential - The damage potential to these two web services is high (3). Elevation of privilege is such a high threat
that the potential damage that could be made is escalated solely because of this. The number of administrators and potential
administrators on the website is so high that even if an elevation of privilege attack was made it would be almost impossible
to trace it. Administrators also have too much access to information site wide. Administrators of a local business has access
to almost any information they wish.\\

Reproducibility - The attacks that this website is vulnerable to can be reproduced at any given time and so this is rated
as high (3). The privileges and denial of service attacks, the two most harmful attacks this web portal can face are not
dependent on some timing or open window, they can be repeated at whim by any given attackers.\\

Exploitability - Since the two main attacks of elevation and DoS can be made by a novice and a skilled programmer, this
will be rated as a (2.5). Here the elevation attacks could range from simply someone giving access where they shouldn't
have through social engineering to attempting to break into the user database and change roles. The DoS attacks are more
sophisticated in nature and require a somewhat skilled programmer to execute them.\\

Affected Users - This must be rated as high (3). Once an unauthorized attacker has gain elevated privileges, all user
information across the site could be compromised as administrators have access to way too much information on the whole
site. Secondly, in a denial of service attack, all users are affected. Since the web portals and subsections of all the
businesses managed on the website are run through central servers and frameworks, a single DoS attack could bring down
every user action. This would have extreme consequences in the case of paying property taxes.\\

Discoverability - This is rated as between medium and high (2.5). The phising attacks that could be used against the 
website are easily understood by attacks with all the information needed being public. These attacks can result in
cross site scripting and eventually lead to data tampering issues within the web portal. The denial of service attacks
are seldom used against such small websites and web businesses. They are mainly directed at large companies or federal
government organizations. 

\end{document}