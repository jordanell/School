%%%%%%%%%%%%%%%%%%%%%%%%%%%%%%%%%%%%%%%%%
% Short Sectioned Assignment
% LaTeX Template
% Version 1.0 (5/5/12)
%
% This template has been downloaded from:
% http://www.LaTeXTemplates.com
%
% Original author:
% Frits Wenneker (http://www.howtotex.com)
%
% License:
% CC BY-NC-SA 3.0 (http://creativecommons.org/licenses/by-nc-sa/3.0/)
%
%%%%%%%%%%%%%%%%%%%%%%%%%%%%%%%%%%%%%%%%%

%----------------------------------------------------------------------------------------
%	PACKAGES AND OTHER DOCUMENT CONFIGURATIONS
%----------------------------------------------------------------------------------------

\documentclass[paper=a4, fontsize=11pt]{scrartcl} % A4 paper and 11pt font size

\usepackage[T1]{fontenc} % Use 8-bit encoding that has 256 glyphs line to return to the LaTeX default
\usepackage[english]{babel} % English language/hyphenation
\usepackage{amsmath,amsfonts,amsthm} % Math packages

\usepackage{lipsum} % Used for inserting dummy 'Lorem ipsum' text into the template

\usepackage{sectsty} % Allows customizing section commands
\allsectionsfont{\centering \normalfont\scshape} % Make all sections centered, the default font and small caps

\usepackage{fancyhdr} % Custom headers and footers
\pagestyle{fancyplain} % Makes all pages in the document conform to the custom headers and footers
\fancyhead{} % No page header - if you want one, create it in the same way as the footers below
\fancyfoot[L]{} % Empty left footer
\fancyfoot[C]{} % Empty center footer
\fancyfoot[R]{\thepage} % Page numbering for right footer
\renewcommand{\headrulewidth}{0pt} % Remove header underlines
\renewcommand{\footrulewidth}{0pt} % Remove footer underlines
\setlength{\headheight}{13.6pt} % Customize the height of the header

\numberwithin{equation}{section} % Number equations within sections (i.e. 1.1, 1.2, 2.1, 2.2 instead of 1, 2, 3, 4)
\numberwithin{figure}{section} % Number figures within sections (i.e. 1.1, 1.2, 2.1, 2.2 instead of 1, 2, 3, 4)
\numberwithin{table}{section} % Number tables within sections (i.e. 1.1, 1.2, 2.1, 2.2 instead of 1, 2, 3, 4)

\setlength\parindent{0pt} % Removes all indentation from paragraphs - comment this line for an assignment with lots of text

%----------------------------------------------------------------------------------------
%	TITLE SECTION
%----------------------------------------------------------------------------------------

\newcommand{\horrule}[1]{\rule{\linewidth}{#1}} % Create horizontal rule command with 1 argument of height

\title{	
\normalfont \normalsize 
\textsc{University of Victoria, Software Engineering} \\ [25pt] % Your university, school and/or department name(s)
\horrule{0.5pt} \\[0.4cm] % Thin top horizontal rule
\huge Case Study 11: Physical and Environmental Security \\ % The assignment title
\horrule{2pt} \\[0.5cm] % Thick bottom horizontal rule
}

\author{Jordan Ell \\ V00660306} % Your name

\date{\normalsize\today} % Today's date or a custom date

\begin{document}

\maketitle % Print the title

%----------------------------------------------------------------------------------------
%	PROBLEM 1
%----------------------------------------------------------------------------------------

\section{Introduction}

This case study is focused on the physical and environmental security of the Upperton city
IT governance structure. The scenario states that a new Corporate Security Officer has been
added to the security team while a laptop was recently stolen from the data center in the 
previous week. The CSO has been tasked with the job of improving information security practices
for the city's data center. This includes short and long term goals as well as obtaining funding
from the city to implement these goals. These rest of this case study will be laid out as follows.
First, I will explain what should be immediately done about the stolen laptop with an investigation.
Next, I will explain both short and long terms plans to improve the physical and environmental
security. Lastly, I will explain how possible funding could be achieved to help implement the
aforementioned goals.


\section{Investigation}

Due to the laptop theft, an immediate investigation should take place involving the circumstances
and damage caused by the theft. The investigation should focus on the significance of the theft.
How easily were the thieves able to obtain the stolen laptop? Did they have to go through 
already implemented security measures, or are there back doors in place in which a thief might
circumvent the whole security infrastructure in order to steal the laptop. This emphasis on significance
should outline exactly where the weak spots are in the pre existing security environment and should
be the main focus of adaptation of a new security plan. The significance of the theft report should
also focus on what was actually stolen. Are laptops just laying around the data center unattended?
How easy is it for a thief to steal a rack server. What information is being placed on these machines
that may be physically less demanding to steal. If unencrypted data is being stored on a non secured
laptop in the data center, this is a serious cause for concern.\\

\section{Short Term Goals}

The easiest short term goal, is to ensure that everyone is following pre existing security architecture.
This will at the very least help with liability through accounting for legitimate actions in the
data center warehouse. The second quick solution to laptop thefts are to reduce the potential 
points of entry to only secure entry points. If the data center has a key card door, but at the same
time has a ground level window or a back door (even if locked) this is a large concern. The data center
should be reduced to the single secure entry point where actions are accounted for (employees entering
and leaving the data center).\\

A second (possible) short term goal would be to install CCTVs in the data center. CCTVs are relatively
inexpensive and can be installed with general ease. This is what makes them a good potential short term
security measure as they can be up and running relatively quickly. A few things to note about the CCTVs
in the data center. First, they should only be installed where they cannot be tampered with or otherwise
turned off without the camera itself, or other cameras catching the action in their field of view. This
will stop thieves from simply disabling the cameras on their own. Second, the footage captured from the
CCTVs should be stored externally from the data center they are monitoring. Thieves should not be able
to break into both the recorded data center as well as the footage collection facility. This will
stop thieves from simply stealing the recording of the break in.\\

A final short term solution of the break-in is to ensure that only those employees with a need-to-access
authority actually have access to the data center. Having too many employees with heightened privileges
is a bad idea for security as the accountability decreases as volume increases. Limiting the data
center to a need-to-access authority will impede potentially hazardous employees who normally would
have no need to enter the data center other that to steal from it from entering. For most non secure
data centers this can be achieved with simple key privileges.

\section{Long Term Goals}

The first long term goal I would suggest that the CSO implement, is key card access with proper
authentication procedures in place. Having key card access improves liability among employees
as logs of entering and exiting the data warehouse can be kept. These logs allow for future analysis
in case of any mishap in the data center. You can check to see who was in the room at the time.
The key card access can also help raise suspicious activity to the IT team. If one employee
is rapidly entering and exiting the data center, this may be cause for concern, or if an employee
enters and does not exit for a long period of time, this may also be a concern. The second side
of key cards is to have proper procedures in place to allow for need-to-access authority to be
given to the correct people. Key cards can be set up to only allow certain employees into specific
rooms of a building. Have a proper, documented, procedure in place for these elevated privileges will also
allow for higher security standards as it will be known who has access to which equipment.\\

A second long term goal will to be to analyses the types of threats the company has to deal with
on a regular or potential basis. If the office resides on the ground floor of a building, windows
may cause a very large risk to potential physical environment harm. A plan to potential bar, or
install shatter resistant windows may be a good step in the right direction to securing a large 
building in the log term.

\section {Funding}

In order to implement some of these suggested short and long term security measures, proper
funding must be in place, especially for more costly measure such as CCTVs, key cards, and 
window installation. The best way to achieve funding for security projects, as with every case
study done in this class so far, is to obtain corporate level buy-in and explain the situation
from a business goal stand point. If the executives of the company (or city of Upperton) can
view physical security as a business expense, they may be more willing to hand over more money
for potential security measures.\\

To explain physical security as a business expense, I would recommend that the CSO explains
the cost of a data breach or physical break in to the corporate level executives. This involves
explain physical costs as well as potential damages from following lawsuits caused by the
loss of private data from citizens. The physical costs would be that of the loss of equipment.
Examples of this might be the laptop theft (the cost of the laptop), damages to locks from
the break in, and damage to heavy equipment during the break in (this could be damaged rack
servers, or heavy and expensive main frames). The larger for potential loss is the damage from
stolen information. If credit card information is stolen, lawsuits could become an imminent
threat to Upperton. These monetary losses could far exceed physical damage costs and should
be mitigated through proper physical environment security measures.\\

With these explanation, any CSO should be able to find funding for the security measures
that I have outlined in the short and long term goals previously. The funding supplied
should also account for future upgrades, future damages, and employee salaries to have
a security team that can monitor and act on security infrastructure.
%----------------------------------------------------------------------------------------

\end{document}