%%%%%%%%%%%%%%%%%%%%%%%%%%%%%%%%%%%%%%%%%
% Short Sectioned Assignment
% LaTeX Template
% Version 1.0 (5/5/12)
%
% This template has been downloaded from:
% http://www.LaTeXTemplates.com
%
% Original author:
% Frits Wenneker (http://www.howtotex.com)
%
% License:
% CC BY-NC-SA 3.0 (http://creativecommons.org/licenses/by-nc-sa/3.0/)
%
%%%%%%%%%%%%%%%%%%%%%%%%%%%%%%%%%%%%%%%%%

%----------------------------------------------------------------------------------------
%	PACKAGES AND OTHER DOCUMENT CONFIGURATIONS
%----------------------------------------------------------------------------------------

\documentclass[paper=a4, fontsize=11pt]{scrartcl} % A4 paper and 11pt font size

\usepackage[T1]{fontenc} % Use 8-bit encoding that has 256 glyphs line to return to the LaTeX default
\usepackage[english]{babel} % English language/hyphenation
\usepackage{amsmath,amsfonts,amsthm} % Math packages

\usepackage{lipsum} % Used for inserting dummy 'Lorem ipsum' text into the template

\usepackage{sectsty} % Allows customizing section commands
\allsectionsfont{\centering \normalfont\scshape} % Make all sections centered, the default font and small caps

\usepackage{fancyhdr} % Custom headers and footers
\pagestyle{fancyplain} % Makes all pages in the document conform to the custom headers and footers
\fancyhead{} % No page header - if you want one, create it in the same way as the footers below
\fancyfoot[L]{} % Empty left footer
\fancyfoot[C]{} % Empty center footer
\fancyfoot[R]{\thepage} % Page numbering for right footer
\renewcommand{\headrulewidth}{0pt} % Remove header underlines
\renewcommand{\footrulewidth}{0pt} % Remove footer underlines
\setlength{\headheight}{13.6pt} % Customize the height of the header

\numberwithin{equation}{section} % Number equations within sections (i.e. 1.1, 1.2, 2.1, 2.2 instead of 1, 2, 3, 4)
\numberwithin{figure}{section} % Number figures within sections (i.e. 1.1, 1.2, 2.1, 2.2 instead of 1, 2, 3, 4)
\numberwithin{table}{section} % Number tables within sections (i.e. 1.1, 1.2, 2.1, 2.2 instead of 1, 2, 3, 4)

\setlength\parindent{0pt} % Removes all indentation from paragraphs - comment this line for an assignment with lots of text

%----------------------------------------------------------------------------------------
%	TITLE SECTION
%----------------------------------------------------------------------------------------

\newcommand{\horrule}[1]{\rule{\linewidth}{#1}} % Create horizontal rule command with 1 argument of height

\title{	
\normalfont \normalsize 
\textsc{University of Victoria, Software Engineering} \\ [25pt] % Your university, school and/or department name(s)
\horrule{0.5pt} \\[0.4cm] % Thin top horizontal rule
\huge Case Study 10: Business Continuity \\ % The assignment title
\horrule{2pt} \\[0.5cm] % Thick bottom horizontal rule
}

\author{Jordan Ell \\ V00660306} % Your name

\date{\normalsize\today} % Today's date or a custom date

\begin{document}

\maketitle % Print the title

%----------------------------------------------------------------------------------------
%	PROBLEM 1
%----------------------------------------------------------------------------------------

\section{Outline}

This case study is centered around the hypothetical Fast Grow Farms business (FGF). This
business operate a fish farm with key points of operation being distribution to Vancouver
Island and mainland BC as well as service imports from Sayward on Vancouver Island
and Fraser Valley on the interior of BC. This case study will attempt to create and design
a business continuity plan to ensure that FGF can undure even the largest of crisies it
may face during its operation.\\

This case study is laid out into the following section. Initiation will determin the scope,
objects and assumptions for the business analysis, strategy development and continuity
place. The  Functional requirements will determin what is critical for recovering from a
disaster as well as identify potential threats and could harm the farm. The functional
requirements will also preform business impact analysis and threat assessment. Finally,
the business continuity plan will be described in bried on the whole of the project.


%----------------------------------------------------------------------------------------
%	PROBLEM 1
%----------------------------------------------------------------------------------------

\section{Initiation}

The main continuity problem attempting to be addresses by this case study is that of
continued business of FGF even during the worst disaster or crisis. Here, the continued
business of FGF will be the ability to maintain live stock and the selling of these live
stock assessts. The scope of the business plan should be the center of the FGF operations,
being the actual fish farm itself and the import export routes of any type of produce.
These types of imports and exports will mainly be the trucking routes, trucks themselves
and drivers which are responsible for bringing in product and well as exporting product
to market. These types of products will be the import of the fish feed from Sayward and
the export of the actual fish once they have been grown on the farm. This continuity
plan will not cover incidents regarding suppliers base of operations such as an individual
fish feed supplier in Sayward, but should cover alternatives for seperate suppliers
given that the disaster my come from a supplier. Finally, this continuity plan should
cover reasonable responses to natural disaster among trading routed or the FGF base of
operations.\\

The assumptions being made for this case study are two fold. We are assuming that the 
Sayward fish feed importer is the sole importer to the FGF business. This helps narrow
the scope in terms of importer continuity and the levels of importance places on each
importer for the FGF busness. Secondly, we are assuming that FGF exports all of its
fish products to only two distributions centers, one on Vancouver Island and one
on the interior of BC. This also narrows the scope for and continuity involving the 
export of product. Finally, and this goes without saying, we assume that FGF own only
one fish farm in its business so that the continuity is in regards to the single farm
as opposed to being spread accross multiple locations.\\

Memebers of the Business Continuity Steering Committee will be as follows. The chief
financial officer needs to be aboard for any fiscal potential damage that may come to
the company incase of a lsoe in continuity. The head of technology being IT or otherwise
should also be there as any sales and marketing technology used will be heavily involved
in continuity. Finally, adding a third party person who speacilizes in risk management
and threat assessment will join the team as someone who can lead the discussions of
where protentical continuity problems may arise in the FGF business.\\

I would have the chief financial office make the presentation to the executive. Having
a c-level executive make a presentation really ephasises the business strategy in the
decision to have a business continuity plan. This corprate level buy-in is a critical
success factor towards the business continuity plan.


%----------------------------------------------------------------------------------------
%	PROBLEM 1
%----------------------------------------------------------------------------------------

\section{Functional Requirements}

Above all assest in the FGF business, I believe that the most critical assest to the company
are the live fish that are currently being farmed. Without the live fish, no product can be
sold and no selling results in no buying of feed or more fish eggs. This being said, feed is
second to live stock life as it maintains their life. This being said, we can move forward to
potential steps to recover from a disaster.\\

The number one step should be to assure life of the live stock is maintained. We will analysis
this step by posing two possible scenarios to form a disaster. The first scenario which is
potentially the most likely is a loss of power to the farm. There should be preemptive steps
taken to assure that a loss of power does not disrupt automatic systems such as feeders, lights
or heating that may take place. Backup generators should be installed an have sufficient power
to last the length of a typical power black out as seen on Vancouver Island. If a blackout lasts
longer than the generators can last, the following steps should be taken to ensure continuity
through the safety of live stock. Procedures should be in place to ensure feeding is done manually
when nessicary. The farm should be in such a place where any loss of heat will not become a factor
until a very large time span has occured. Finally, the business aspects of FGF should not becomes
jeopardized because of loss of power. This means that any IT infrastructure should have backup
plans. These plans can be that of hosting the server from a dedicated hosting company whish ensures
no power loss or having cell phones as backup phones for the company to ensure customers and suppliers
can still be reached to inform of any prolonged situation.\\

A second possible scenario could extreme weather forecasts which threaten the lives of the fish
as they are contained in a facility. Continuity measures such as having an indoor facility
could compensate this but seem to be extremely costly towards the company. A much more appealing
solution to this potential disaster is to have seperate farms in different locations to ensure
that if disaster strikes one farm, another may be left unharmed. This will ensure that business
continues during a disaster as some of the product will be saved and profits can still be made.\\

\subsection{Business Impact Analysis}

Three high level business functions are as follows. The importing of fish feed. This involves
both account payments and assest recievements. An owner of this function may be the accounting
team as they will be responsible for correctly managing how much fish feed is bought given the amount
of fish the company is raising and what the budget will allow. Two key attributes of this function
are are the import services such as trucks and drivers and the fish feed company itself as a
supplier to FGF. The farm's live stock is dependant on this function. Without the import of feed,
the fish have no way to eat and grow thus this function is the start of revenue. The RTO of this
function is less than a month. This is assuming that enough fish feed to brough at any given time
to last at least a month. This function is recovered either by switching fish feed suppliers or
by hiring new trucks for import delivery.\\

The exporting of fish product to distributions
centers. Again this involves account receivable and ledge reporting while it will also have potential
effects on payroll depending on the size of the company. This will likely involve the marketing and
sales teams as they will be responsible for the unloading of product in order to keep profits and
revenue streams high. Two key attributes of this function are the export services such as trucks
and drivers and the distribution export buyers themselves. The revenue streams of FGF are soley
dependant of this function and the company would cease to exist without it. The RTO of this 
function is hard to know precisely but it would be whatever the life cycle of the fish farm
would be. If a distribution center stopped buying product, a new one would have to be found
by the end of the fish life cycle or a risk of products loss and revenue loss would be likely.
This function is recovered by having new distribution centers or new truck drivers is the 
deleviry system begins to fail.\\

Thirdly, We have live stock maintenance which
involves the health and safety of the live stock being raised to be sold. The owners of this
function would be the farmers themselves. The farmers are responsible for the growth and
health of the fish being raised. The two key attributes of this function are, obviously, the
fish but also the employees in charge of the fish. The entire company is dependant on this 
function. Without the fish, the company would not exist. The RTO of this function incase 
of loss of fish would be one fish life cycle. Having a single missed revenue of fish export
may not be a large loss, but two would be devistating. The RTO of employees would be a week
to a month. Employees should be replacible in case of disaster within this time. Human 
resources may be needed to recover from this or outside consulting to help with the
breeding of fish if the process ever encures a disaster.\\

\subsection{Threat Assessment}

For this threat assessment I will analysis the function of the importing of fish feed
from the Sayward supplier. Five potential risks of this function are as follows:

\begin{enumerate}

\item The supplier goes out of business. Hopefully this risk will not occure, or if it
does then FGF will have had ample time to prepare to transfer the importing of feed to 
another company.
\item A delivery truck does not make it to the destination. This is a very real possiblity
as Vancouver Island can be trecherous for driving large trucks on.
\item The supplier has a shortage of feed and cannot deliver the promised amount. A 
very real possibility if the supplier has quick growth in customers or other problems
generating feed.
\item The delivery service goes out of business. Hauling companies are constantly
changing hands and this may need to be dealt with.
\item The fish feed delivered has been tainted. A delivery may be contaminated and not
be usable on the farm.

\end{enumerate}

The following list gives the threat rating to the matching threat listed above.

\begin{enumerate}

\item Likelihood: Low, Impact: Medium, Threat Medium (The consequences are that a new
supplier must be found in time for the next delivery of feed. Failure to do so may cause
loss of live stock and profit.)
\item Likelihood: Medium, Impact: High, Threat High (The consequences are of food 
shortages for a small amount of time. Failure of delivery may cause minor casualties
to the live stock.)
\item Likelihood: Medium, Impact: Medium, Threat Medium (The consequences are a loss
of feed for fish which may lead to death in fish depending on the shortage of the
supply.)
\item Likelihood: Low, Impact: Low, Threat Low (The consequences are having no delivery
from the supplier. This will have very large potential casualties if a new delivery 
service is not found before the next feed purchase)
\item Likelihood: Low, Impact: High, Threat High (The consequences are a loss of 
spent revenue in the feed and potential loss of fish life from those fed the contaminet
and death of those not fed all together.)

\end{enumerate}



%----------------------------------------------------------------------------------------
%	PROBLEM 1
%----------------------------------------------------------------------------------------

\section{Business Continuity}

The business functions that need to be recovered within a 72 hours time window (
from those listed) are the importing of fish feed and the growth and health of 
the fish being farmed. The threats of the delivery serivce or supplier serivce of fish
feed going out of business would cause an issues for the importing of fish feed as
well as the failed delivery of fish feed. The threat of fish feed contaminant is a
serious threat towards the health and growth of the fish live stock and would be a
major problem for this critical function.\\

I will select the highest planning priority of being the health and growth of the
fish live stock on the farm. For facility threats, as it was already mentioned,
the building of the fish farm with thought is paramount. In loss of power, generators
and manual procedures for feeding will need to be in place. These recovery strategies
also apply to infrastructure of the fish. Be sure that not all eggs are in one basket,
meaning place fish in different infrastructural locations to ensure disaster does not
strike all live stock at once. For personnel, make sure that employees can be replaces
through HR when needed and that experts can be brought in when needed to assist in 
any fish farming issues that may arise leading to harming the live stock.\\

Decisions made are as follows. Have power generators at the fish farms. Have
manual processes in place in case of automation failure (the opposite of Jurassic Park).
Have hiring process and eomplyee turnover processes in place with training. Have
potential experts in fish farming on consulting teams to help with any potential
live stock disasters.

%----------------------------------------------------------------------------------------

\end{document}