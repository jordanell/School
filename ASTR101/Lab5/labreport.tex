%%%%%%%%%%%%%%%%%%%%%%%%%%%%%%%%%%%%%%%%%
% Laboratory Report LaTeX Template
%
% This template has been downloaded from:
% http://www.latextemplates.com
%
%%%%%%%%%%%%%%%%%%%%%%%%%%%%%%%%%%%%%%%%%

%----------------------------------------------------------------------------------------
%	DOCUMENT CONFIGURATIONS
%----------------------------------------------------------------------------------------

\documentclass{article}

\usepackage{amssymb, amsmath}

\title{Search for Extraterrestrial Intelligence \\ ASTR 101} % Title

\author{Jordan \textsc{Ell} \\ jordan.ell7@gmail.com \\ V00660306} % Author name

\begin{document}

\maketitle % Insert the title, author and date

\begin{tabular}{lr}
Date Performed: 19/11/2012\\ % Date the experiment was performed
Instructor: Jillian Scudder % Instructor/supervisor
\end{tabular}

\setlength\parindent{0pt} % Removes all indentation from paragraphs

\renewcommand{\labelenumi}{\alph{enumi}.} % Make numbering in the enumerate environment by letter rather than number (e.g. section 6)

%----------------------------------------------------------------------------------------
%	SECTION 1
%----------------------------------------------------------------------------------------

\section{Objective}

This report will demonstrate how to properly estimate the number of intelligent civilizations there may be inside of the Milky Way Galaxy and
how one can calculate how far away the nearest one to Earth may be. \\
 
%----------------------------------------------------------------------------------------
%	SECTION 2
%----------------------------------------------------------------------------------------

\section{Introduction}

The search for extraterrestrial intelligence (SETI) in our universe is a number of methods and procedures carried out by professional and amateur
astronomers in the hopes of detecting other forms of life, specifically, intelligent life. Methods include electromagnetic radiation
monitoring with radio or other types of telescopes~\cite{Peter:2006} seeing as it would be the quickest way to communicate
across the largest distances. The first of these listening methods was conducted against Mars in 1924 with large radio telescopes~\cite{Dick:1999}, to which
people thought they might get a response from little green men.\\

With the recent discoveries and influx of planetary
discoveries in our universe, scientists have begun to wonder if there is actually a high percentage of possibility that life exists in complex
forms not just on our planet.  Given that the Zoo Hypothesis~\cite{Ball:1973} is not a likely solution, the Drake Equation tries to tackle this problem.\\

The equation discovered by Frank Drake in 1961 is a composite of the obstacles which are believed to be the larger circumstances which make
up the possibility for complex life to form. These obstacles include how many stars have plants, how likely it is life to form and evolve and so
on. The problem with this equation and SETI in general is that all fields are virtually unknown. Only very rough estimates can be made for
each field, although they continue to become better estimates.\\

While these estimates become better through scientific measurements, this report will focus on the Drake equation in a much more reduced
manor. This report will attempt to count the number of intelligent civilizations that are currently alive inside of our Milky Way Galaxy. This reports
will make use of much given data and very large estimates based on largely accepted numbers about human existence, life existence on Earth etc.
However, not all of the given data is strong enough to rely heavily on the outcome of this report. After the number of civilizations is derived, this
report will hold a minor discussion in Section~\ref{sec:qna} about possibilities of communication and contact with these civilizations.\\

The major assumptions this report makes are as follows: stars are spread evenly across the Milky Way; many small planets are undetectable
by current means but do exist; surface temperature of a planet is caused mostly by distance from its star; wherever life can form it will rapidly;
intelligent civilizations can craft tools; the human civilization will last for at least 500,000 years which may be a good approximation for such
a largely unknown factor~\cite{nova:2010}.

%----------------------------------------------------------------------------------------
%	SECTION 3
%----------------------------------------------------------------------------------------

\section{Equipment}

The equipment used for this report is as follows: an image of the centre of the Milky Way Galaxy, a measurement magnifying glass,
 a Sharp EL-510R calculator, a computer running Windows XP, and a modelling program for radial velocity and star luminosity 
calculations and data graphs.

%----------------------------------------------------------------------------------------
%	SECTION 4
%----------------------------------------------------------------------------------------

\section{Procedure}
\label{sec:proc}

Given an image of the centre of the Milky Way Galaxy, the first step done was to measure how many stars appear inside of a ${1mm}^{2}$
area on the picture. This was done using the provided measurement magnifying glass. The result of this measurement can be seen inside of
Table~\ref{tab:drake}. Next, Equation~\ref{eq:stars} was used to extrapolate this information into the number of total stars inside of the 
Milky Way Galaxy.\\

Next, the given computers with the modelling program installed were used. This program was used to determine whether or not current 
methods could detect planets of size Earth and Jupiter. This was done by editing the planet models radius, mass and orbital inclination inside
of the program. The results for these finding can be found in Section~\ref{sec:qna}. Then, an estimate for what fraction of stars have planets
was made based off of the given data from the Kepler space telescope which was observing around 4000 stars with approximately 2000 planet
candidates. The result of this estimation was calculated with Equation~\ref{eq:planets} and results can be seen inside of Table~\ref{tab:drake}.\\

Next, an estimation of how many habitable planets there are inside of each solar system was made. This was done by assuming that habitable means there
is liquid water on the planet. Therefore, maximum and minimum distances away from the star were calculated using Equation~\ref{eq:mm} for a given
planet to have liquid water. These distance were put onto the data given in Table~\ref{tab:planets}. Equation~\ref{eq:ne} was then used in order to give 
a rough estimate on average how many solar systems have planets. Results can be seen inside of Table~\ref{tab:drake}.\\

Next, an estimate was made of what fraction of habitable planets contain life. This was done by looking at Earth as an example and given the data
that Earth has had liquid water for 4 billion years and life for 3.9 billion years. The result of this estimation was calculated with Equation~\ref{eq:life} 
and results can be seen inside of Table~\ref{tab:drake}.\\

Next, an estimate was made of what fraction of life is intelligent. This term intelligent for this report is defined as the ability to create and use
tools inside of the civilization. This being the case, it was estimated that civilizations last for 500,000 years given that a civilization ending 
disaster will occur every 1 million years. This time is also subject to the lifetime of the star that the civilization depends on.
The result of this estimation was calculated with Equation~\ref{eq:civ} and results can be seen inside of Table~\ref{tab:drake}.\\

Finally, the Drake equation was carried out by multiplying all drake parameters together in Equation~\ref{eq:drake} which resulted in
17915.625 civilizations being inside of our Milky Way Galaxy. As an extra step, it was found that the nearest civilization to ours would
be approximately 595.89 light years away by using Equation~\ref{eq:near}.\\


%----------------------------------------------------------------------------------------
%	SECTION 5
%----------------------------------------------------------------------------------------

\section{Observations}

All observations for this report were made on the day of 2012-Nov-19. As all calculations and measurements made inside of this report are based off of
given data, weather conditions and time of day have not been reported as they hold no effect. The measurement of how many stars are inside of our
Milky Way Galaxy can be found in the Section~\ref{sec:tnm}. All other measurements are made using very rough approximations and their explanations
can be found inside of Section~\ref{sec:proc} along with how they were calculated. Rough calculations and measurements can be found at the back of
this lab report as well.\\

\vspace{30 mm}

%----------------------------------------------------------------------------------------
%	SECTION 6
%----------------------------------------------------------------------------------------

\section{Tables and Measurements}
\label{sec:tnm}

\begin{table}[h]
\begin{center}
\begin{tabular}{l c c}
\hline
Drake Parameter & Measurement & Result\\
\hline
\hline
${N}_{*}$ & 8 & 1.2 Billion\\
${f}_{p}$ & 2600 & .65\\ 
${n}_{e}$ & 5 & 1.25\\
${f}_{L}$ & - & 0.98\\
${f}_{i}$ & - & 0.3\\
${F}_{s}$ & 500,000 & 6.25x${10}^{-5}$\\
\hline
\end{tabular}
\end{center}
\caption{Measurements of parameters inside the Drake Equation.\label{tab:drake}}
\end{table}

\begin{table}[h!]
\begin{center}
\begin{tabular}{l c c}
\hline
Solar System & Planet & Distance (AU)\\
\hline
\hline
Our Solar System & Mercury & 0.387\\
Our Solar System & Venus & 0.723\\
Our Solar System & Earth & 1.00\\
Our Solar System & Mars & 1.524\\
Our Solar System & Jupiter & 5.203\\
Ups Andromedae & - & 0.06\\
Ups Andromedae & - & 0.83\\
Ups Andromedae & - & 2.51\\
55 Cancre & - &  0.04\\
55 Cancre & - &  0.11\\
55 Cancre & - &  0.24\\
55 Cancre & - &  0.78\\
55 Cancre & - &  5.77\\
HD160691 & - &  0.09\\
HD160691 & - &  0.92\\
HD160691 & - &  1.5\\
HD160691 & - &  4.17\\
\hline
\end{tabular}
\end{center}
\caption{Given data of solar system planet's distance from the star.\label{tab:planets}}
\end{table}

%----------------------------------------------------------------------------------------
%	SECTION 7
%----------------------------------------------------------------------------------------

\section{Calculations}
\label{sec:calc}

To calculate the real number of stars in our Milky Way Galaxy, the measurement of ${N}_{*}$ in Table~\ref{tab:drake} can be
multiplied by 150,000,000 to make up the ratio of the measured size of 1 square millimetre. Equation~\ref{eq:stars} below 
shows this.
\begin{equation}
\label{eq:stars}
\text{N}_{*} = S * 150,000,000
\end{equation}

To calculate the fraction of stars which have planets, Equation~\ref{eq:planets} below is used. By dividing the measurement of
${f}_{p}$ in Table~\ref{tab:drake} by the total number of stars in our sample size, we get the percentage of stars which are 
estimated to have planets around them.
\begin{equation}
\label{eq:planets}
\text{f}_{p} = \frac{{P}_{estimated}}{{S}_{total}}
\end{equation}

To calculate the distance a planet must be away from its star to account for a given surface temperature, the following
equation may be used.
\begin{equation}
\label{eq:mm}
\text{D}_{p} = \frac{82944}{{T}_{P}^{2}}
\end{equation}

To calculate the number of habitable planets in each solar system we simply take the measurement of 
${n}_{e}$ in Table~\ref{tab:drake} which is the total number of habitable planets in a sample size and divide by the number 
of total planets in that sample. This can be seen in Equation~\ref{eq:ne}.
\begin{equation}
\label{eq:ne}
\text{f}_{e} = \frac{{P}_{habitable}}{{P}_{total}}
\end{equation}

To calculate the chance that planets have life on them and get the measurement of ${f}_{L}$ in Table~\ref{tab:drake}, 
we take the number of years Earth is known to have life and divide it by the number of years Earth is known to have liquid
water on its surface. This can be seen in Equation~\ref{eq:life}.
\begin{equation}
\label{eq:life}
\text{f}_{L} = \frac{3.9 Billion}{4.0 Billion}
\end{equation}

To calculate the lifetime of a civilization, we take the measurement of ${F}_{S}$ in Table~\ref{tab:drake} which is the estimated
lifetime of a civilization and divide it by the lifetime of a normal start such as our Sun. This can be found in Equation~\ref{eq:civ}.
\begin{equation}
\label{eq:civ}
\text{f}_{L} = \frac{500,000}{8.0 Billion}
\end{equation}

To calculate the number of intelligent civilizations inside of out Milky Way Galaxy, we simply take all of the parameters to the
Drake equation and multiply them together. This can be seen in Equation~\ref{eq:drake}.
\begin{equation}
\label{eq:drake}
\text Intelligent Civilizations = {N}_{*} * {f}_{p} * {n}_{e} * {f}_{L} * {f}_{i} * {F}_{s}
\end{equation}

To calculate the nearest civilization in our galaxy to ours, we find the average area a civilization occupies in the Galaxy and then 
take the square root of it. This can be seen below in Equation~\ref{eq:near}.
\begin{equation}
\label{eq:near}
\text Nearest Civilization = \sqrt{\frac{\pi * 45,000^{2}}{Intelligent Civilizations}}
\end{equation}



%----------------------------------------------------------------------------------------
%	SECTION 7
%----------------------------------------------------------------------------------------

\section{Questions}
\label{sec:qna}

The following questions and answers are asked inside of lab 8,Search for Extraterrestrial Intelligence, inside of the lab manual
for ASTR101. The questions have been repeated for the reader.

\begin{enumerate}
\item[Q.] Would you be able to detect a Jupiter mass planet in a one year orbit?
\item[A.] You would be able to detect a Jupiter mass planet as it has enough mass to pull the star around giving a reading in 
radial velocity and it is also large enough to give off a dip in the star's luminosity reading. Being able to see these types of
planets is a direct cause of the selection effect with our current detection methods.
\item[Q.] Would it be possible to detect planets like the Earth?
\item[A.] An Earth sized planet would not be able to be detected because it is too small to show a dip in the star's luminosity
and it is not massive enough to show movement in the star causing radial velocity. These are the reasons the main methods
of searching for planets right now do not often show Earth sized planets that are around 1AU away from the star.
\item[Q.] Compare the average distance between civilizations to the lifetime of a civilization.
\item[A.] In this report's estimated case, the distance to the nearest civilization is 595.89 light years while each civilization is
estimated to survive for around 500,000 years. This means that the civilizations would be detectable sometime during
the lifetime of the civilizations. They would be able to send about 830 messages in total between civilizations in this
time span.
\item[Q.] Would our earliest radio signals have made it to the nearest civilizations yet?
\item[A.] Seeing as our earliest radio signals have only been travelling for about 50-60 years, these distant civilizations would
not have picked up on them, nor would they see them for another 520 years.
\item[Q.] Based upon your above calculated distance to the nearest alien civilizations, do you expect it to detect any civilizations?
Would conversations between civilizations be possible?
\item[A.] I would expect, that sometime during our civilization's lifetime, we would be able to detect some of the more near
civilizations in our Galaxy. Communications would be possible, however the full time for one conversation would be well over 
1000 years, thus no single human would be able to carry our such a conversation. Also, being able to send and receive signals is
one thing, but given the amount of radio noise inside of space and Earth and supposedly another advanced civilizations,
just detecting them from the noise may prove to be a large part of the problem~\cite{noise:372}.
\item[Q.] What can you decipher about the creatures that made this plaque? Explain what you base it on.
\item[A.] We can determine that the satellite left from the third planet of its origin solar system and the direction it took
to leave that solar system. We know there are two different sexes among the same origin species. We know the origin
species height in comparison to the satellite. We can see how the origin species has mapped their star given certain
distances away from pulsars and which pulsars are indicated by the distance and spin rate given in the left hand side
of the plaque. Finally, we know the hydrogen molecule for a standard measuring device.
\end{enumerate}

%----------------------------------------------------------------------------------------
%	SECTION 8
%----------------------------------------------------------------------------------------

\section{Conclusions / Discussions}

This report has shown how to calculated an estimated number for how many intelligent civilizations there
may be inside of our own Milky Way Galaxy as well as how close those civilizations may be to Earth. This report
has also answered, through these calculations, the likelihood of contact with these civilizations.

%----------------------------------------------------------------------------------------
%	SECTION 9
%----------------------------------------------------------------------------------------

\section{Evaluation}

I found this report and lab very tedious. I believe that the Drake equation is one of the most ludicrous
equations ever created inside of the sciences. Why have we continued to indulge an equation who's margin
of error approaches 100\%? This makes the equation itself meaningless. Although, the equation does
have us thinking in the correct direction and bring to the surface many of the variables that would
have to be understood. I am a user of the SETI@home project though and found this report an
interesting little side note to the data that is actually be collected and processed in the real world.


%----------------------------------------------------------------------------------------
%	BIBLIOGRAPHY
%----------------------------------------------------------------------------------------

\begin{thebibliography}{9}

\bibitem{Peter:2006}
Schenkel, Peter (May 2006). "SETI Requires a Skeptical Reappraisal". Skeptical Inquirer. Retrieved June 28, 2009.

\bibitem{Dick:1999}
 Dick, Steven (1999). The Biological Universe: The Twentieth Century Extraterrestrial Life Debate. ISBN 0-521-34326-7.

\bibitem{Ball:1973}
Ball, John A. (Jul 1973). "The Zoo Hypothesis". Icarus 19 (3): 347–349

\bibitem{nova:2010}
"PBS NOVA: Origins - The Drake Equation". Pbs.org. Retrieved 7 March 2010.

\bibitem{noise:372}
Radio noise. ITU-R Recommendation P.372, International Telecommunication Union, Geneva.

\end{thebibliography}

\end{document}