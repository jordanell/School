%%%%%%%%%%%%%%%%%%%%%%%%%%%%%%%%%%%%%%%%%
% Laboratory Report LaTeX Template
%
% This template has been downloaded from:
% http://www.latextemplates.com
%
%%%%%%%%%%%%%%%%%%%%%%%%%%%%%%%%%%%%%%%%%

%----------------------------------------------------------------------------------------
%	DOCUMENT CONFIGURATIONS
%----------------------------------------------------------------------------------------

\documentclass{article}

\usepackage{amssymb, amsmath}

\title{Search for Extraterrestrial Intelligence \\ ASTR 101} % Title

\author{Jordan \textsc{Ell} \\ jordan.ell7@gmail.com \\ V00660306} % Author name

\begin{document}

\maketitle % Insert the title, author and date

\begin{tabular}{lr}
Date Performed: 19/11/2012\\ % Date the experiment was performed
Instructor: Jillian Scudder % Instructor/supervisor
\end{tabular}

\setlength\parindent{0pt} % Removes all indentation from paragraphs

\renewcommand{\labelenumi}{\alph{enumi}.} % Make numbering in the enumerate environment by letter rather than number (e.g. section 6)

%----------------------------------------------------------------------------------------
%	SECTION 1
%----------------------------------------------------------------------------------------

\section{Objective}

This report will demonstrate how to properly estimate the number of intelligent civilizations there may be inside of the Milky Way Galaxy and
how one can calculate how far away the nearest one to Earth may live. \\
 
%----------------------------------------------------------------------------------------
%	SECTION 2
%----------------------------------------------------------------------------------------

\section{Introduction}



%----------------------------------------------------------------------------------------
%	SECTION 3
%----------------------------------------------------------------------------------------

\section{Equipment}

The equipment used for this lab is as follows: an image of the centre of the Milky Way Galaxy, a measurement magnifying glass,
 a Sharp EL-510R calculator, a computer running Windows XP, and a modelling program for radial velocity and star luminosity 
calculations and data graphs.

%----------------------------------------------------------------------------------------
%	SECTION 4
%----------------------------------------------------------------------------------------

\section{Procedure}
\label{sec:proc}




%----------------------------------------------------------------------------------------
%	SECTION 5
%----------------------------------------------------------------------------------------

\section{Observations}

All observations for this report were made on the day of 2012-Nov-19. As all calculations and measurements made inside of this report are based off of
given data, weather conditions and time of day have not been reported as they hold no effect. The measurement of how many stars are inside of our
Milky Way Galaxy can be found in the Section~\ref{sec:tnm}. All other measurements are made using very rough approximations and their explanations
can be found inside of Section~\ref{sec:proc} along with how they were calculated. Rough calculations and measurements can be found at the back of
this lab report as well.

%----------------------------------------------------------------------------------------
%	SECTION 6
%----------------------------------------------------------------------------------------

\section{Tables and Measurements}
\label{sec:tnm}

\begin{table}[h]
\begin{center}
\begin{tabular}{l c c}
\hline
Drake Parameter & Measurement & Result\\
\hline
\hline
${N}_{*}$ & 8 & 1.2 Billion\\
${f}_{p}$ & 2600 & .65\\ 
${n}_{e}$ & 5 & 1.25\\
${f}_{L}$ & - & 0.98\\
${f}_{i}$ & - & 0.3\\
${F}_{s}$ & 500,000 & 6.25x${10}^{-5}$\\
\hline
\end{tabular}
\end{center}
\caption{Measurements of parameters inside the Drake Equation.\label{tab:drake}}
\end{table}

%----------------------------------------------------------------------------------------
%	SECTION 7
%----------------------------------------------------------------------------------------

\section{Calculations}
\label{sec:calc}

To calculate the real number of stars in our Milky Way Galaxy, the measurement of ${N}_{*}$ in Table~\ref{tab:drake} can be
multiplied by 150,000,000 to make up the ratio of the measured size of 1 square millimetre. Equation~\ref{eq:stars} below 
shows this.
\begin{equation}
\label{eq:stars}
\text{N}_{*} = S * 150,000,000
\end{equation}

To calculate the fraction of stars which have planets, Equation~\ref{eq:planets} below is used. By dividing the measurement of
${f}_{p}$ in Table~\ref{tab:drake} by the total number of stars in our sample size, we get the percentage of stars which are 
estimated to have planets around them.
\begin{equation}
\label{eq:planets}
\text{f}_{p} = \frac{{P}_{estimated}}{{S}_{total}}
\end{equation}

To calculate the number of habitable planets in each solar system we simply take the measurement of 
${n}_{e}$ in Table~\ref{tab:drake} which is the total number of habitable planets in a sample size and divide by the number 
of total planets in that sample. This can be seen in Equation~\ref{eq:ne}.
\begin{equation}
\label{eq:ne}
\text{f}_{e} = \frac{{P}_{habitable}}{{P}_{total}}
\end{equation}

To calculate the chance that planets have life on them and get the measurement of ${f}_{L}$ in Table~\ref{tab:drake}, 
we take the number of years Earth is known to have like and divide it by the number of years Earth is known to have liquid
water on its surface. This can be seen in Equation~\ref{eq:life}.
\begin{equation}
\label{eq:life}
\text{f}_{L} = \frac{3.9 Billion}{4.0 Billion}
\end{equation}

To calculate the lifetime of a civilization, we take the measurement of ${F}_{S}$ in Table~\ref{tab:drake} which is the estimated
lifetime of a civilization and divide it by the lifetime of a normal start such as our Sun. This can be found in Equation~\ref{eq:civ}.
\begin{equation}
\label{eq:civ}
\text{f}_{L} = \frac{500,000}{8.0 Billion}
\end{equation}

To calculate the number of intelligent civilizations inside of out Milky Way Galaxy, we simply take all of the parameters to the
Drake equation and multiply them together. This can be seen in Equation~\ref{eq:drake}.
\begin{equation}
\label{eq:drake}
\text Intelligent Civilizations = {N}_{*} * {f}_{p} * {n}_{e} * {f}_{L} * {f}_{i} * {F}_{s}
\end{equation}

To calculate the nearest civilization in our galaxy to ours, we find the average area a civilization occupies in the Galaxy and then 
take the square root of it. This can be seen below in Equation~\ref{eq:near}.
\begin{equation}
\label{eq:near}
\text Nearest Civilization = \sqrt{\frac{\pi * 45,000^{2}}{Intelligent Civilizations}}
\end{equation}



%----------------------------------------------------------------------------------------
%	SECTION 7
%----------------------------------------------------------------------------------------

\section{Questions}
\label{sec:qna}

The following questions and answers are asked inside of lab 8,Search for Extraterrestrial Intelligence, inside of the lab manual
for ASTR101. The questions have been repeated for the reader.

\begin{enumerate}
\item[Q.] Would you be able to detect a Jupiter mass planet in a one year orbit?
\item[A.] You would be able to detect a Jupiter mass planet as it has enough mass to pull the star around giving a reading in 
radial velocity and it is also large enough to give off a dip in the star's luminosity reading. Being able to see these types of
planets is a direct cause of the selection effect with our current detection methods.
\item[Q.] Would it be possible to detect planets like the Earth?
\item[A.] An Earth sized planet would not be able to be detected because it is too small to show a dip in the star's luminosity
and it is not massive enough to show movement in the star causing radial velocity. These are the reasons the main methods
of searching for planets right now do not often show Earth sized planets that are around 1AU away from the star.
\item[Q.] Compare the average distance between civilizations to the lifetime of a civilization.
\item[A.] In this report's estimated case, the distance to the nearest civilization is 595.89 light years while each civilization is
estimated to survive for around 500,000 years. This means that the civilizations would be detectable sometime during
the lifetime of the civilizations. They would be able to send about 830 messages in total between civilizations in this
time span.
\item[Q.] Would our earliest radio signals have made it to the nearest civilizations yet?
\item[A.] Seeing as our earliest radio signals have only been travelling for about 50-60 years, these distant civilizations would
not have picked up on them, nor would they see them for another ~520 years.
\item[Q.] Based upon your above calculated distance to the nearest alien civilizations, do you expect it to detect any civilizations?
Would conversations between civilizations be possible?
\item[A.] I would expect, that sometime during our civilizations lifetime, we would be able to detect some of the more near
civilizations in our Galaxy. Communications would be possible, however the full time for one conversation would be well over 
1000 years, thus no single human would be able to carry our such a conversation.
\item[Q.] What can you decipher about the creatures that made this plaque? Explain what you base it on.
\item[A.] We can determine that the satellite left from the third planet of its origin solar system and the direction it took
to leave that solar system. We know there are two different sexes among the same origin species. We know the origin
species height in comparison to the satellite. We can see how the origin species has mapped their star given certain
distances away from pulsars and which pulsars are indicated by the distance and spin rate given in the left hand side
of the plaque. Finally, we know the hydrogen molecule for a standard measuring device.
\end{enumerate}

%----------------------------------------------------------------------------------------
%	SECTION 8
%----------------------------------------------------------------------------------------

\section{Conclusions / Discussions}

This report has shown how to calculated an estimated number for how many intelligent civilizations there
may be inside of our own Milky Way Galaxy as well as how close those civilizations may be to Earth. This report
has also answered, through these calculations, the likelihood of contact with these civilizations.

%----------------------------------------------------------------------------------------
%	SECTION 9
%----------------------------------------------------------------------------------------

\section{Evaluation}

I found this report and lab very tedious. I believe that the Drake equation is one of the most ludicrous
equations ever created inside of the sciences. Why have we continued to indulge an equation who's margin
of error approaches 100\%? This makes the equation itself meaningless. Although, the equation does
have us thinking in the correct direction and bring to the surface many of the variables that would
have to be understood. I am a user of the SETI@home project though and found this report an
interesting little side note to the data that is actually be collected and processed in the real world.


%----------------------------------------------------------------------------------------
%	BIBLIOGRAPHY
%----------------------------------------------------------------------------------------

\begin{thebibliography}{9}

\bibitem{Kleine:2005}
 Kleine, T.; Palme, H.; Mezger, K.; Halliday, A.N. (2005). "Hf–W Chronometry of Lunar Metals and the Age and Early Differentiation of the Moon". Science 310 (5754): 1671–1674.
Bates, Robin (series producer), Chesmar, Terri and Baniewicz, Rich (associate producers) (1992). The Dinosaurs! Episode 4: "Death of the Dinosaur" (TV-series). PBS Video, WHYY-TV.

\end{thebibliography}

\end{document}