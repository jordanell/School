%%%%%%%%%%%%%%%%%%%%%%%%%%%%%%%%%%%%%%%%%
% Laboratory Report LaTeX Template
%
% This template has been downloaded from:
% http://www.latextemplates.com
%
%%%%%%%%%%%%%%%%%%%%%%%%%%%%%%%%%%%%%%%%%

%----------------------------------------------------------------------------------------
%	DOCUMENT CONFIGURATIONS
%----------------------------------------------------------------------------------------

\documentclass{article}

\usepackage{amssymb, amsmath}

\title{Lunar Imaging \\ ASTR 101} % Title

\author{Jordan \textsc{Ell} \\ jordan.ell7@gmail.com \\ V00660306} % Author name

\begin{document}

\maketitle % Insert the title, author and date

\begin{tabular}{lr}
Date Performed: 29/10/2012\\ % Date the experiment was performed and partner's name
Instructor: Jillian Scudder % Instructor/supervisor
\end{tabular}

\setlength\parindent{0pt} % Removes all indentation from paragraphs

\renewcommand{\labelenumi}{\alph{enumi}.} % Make numbering in the enumerate environment by letter rather than number (e.g. section 6)

%----------------------------------------------------------------------------------------
%	SECTION 1
%----------------------------------------------------------------------------------------

\section{Objective}

This report will demonstrate how to correctly map objects of interest on the moon onto images taken of the moon near first quarter. It will
also explain how to calculate various dimensional aspects of meteors and impact craters on the surface of the moon and Earth.
 
%----------------------------------------------------------------------------------------
%	SECTION 2
%----------------------------------------------------------------------------------------

\section{Introduction}

The moon has always been one of the bigger interests in the night sky due to its luminosity in our night sky as well as its
proximity to Earth and being our one natural satellite. The moon is thought to have formed some 30-50 million years after
the birth of the Solar System, which would put its birth date about 4.5 billion years ago~\cite{Kleine:2005}. The moon is
believed to have formed during a type of giant impact involving the Earth and a Mars sized object. The impact broke
off large pieces of the Earth which eventually accreted to form the moon~\cite{Taylor:1998}. The moon appears extremely
barren and cratered due to two large time frames of bombardment.  The heavy bombardment and light bombardment stages of
the moons life combined to create the Mares (from heavy) which were at one time large fields of lava that cooled over time,
and the craters (from light) which scatter the moons surface. These craters formed inside, outside and sometime on the perimeter
of the mares which caused them to be partially flooded by lava. The moon, as of late, has even also shown signs of the possibility
that there is water on its surface or underground~\cite{Margot:1999}.\\

During its lifetime the moon has always served as a point of research for mankind. Its cycles were first studied by 5th 
century BC by Babylonian astronomers who recorded its phases and lunar eclipses~\cite{Aaboe:1999}. Aristotle marked
the moon as the boundary between the changing and unchanging~\cite{Lewis:1964}. The space race in the 1960s saw man
first land on the moon and collect lunar rocks for experimentation, as well as many other scientific endeavours. While these,
expeditions serve all of mankind's knowledge base, this report will be focusing more on a simpler task of identifying lunar
features as well as carrying out some measurements of these features.\\

For the purposes of this report and to simplify calculations, it is being assumed that the diameter of the moon is 3476km
and that all meteorites are 25 times less the size of the craters that they create. We know this last assumption not to be
true due to velocity at impact and density of the object but it will make for a more relate-able approach to the calculations 
to follow.\\



%----------------------------------------------------------------------------------------
%	SECTION 3
%----------------------------------------------------------------------------------------

\section{Equipment}

The equipment used for this lab is as follows: three partials images of the moon near first quarter printed onto opaque paper, a map of the moon
at full moon with a listing of known craters, mares and Apollo landing sites,  a pair of scissors, a ruler,
and a roll of scotch tape.

%----------------------------------------------------------------------------------------
%	SECTION 4
%----------------------------------------------------------------------------------------

\section{Procedure}

Given the three images of the moon near first quarter, each image was cut out of the paper while ensuring not to cut too close to the moon 
itself as this may distort the final image to be constructed. Once the images were cut out, they were aligned to form a single image of the moon
near first quarter. During this process some of the images overlapped one another as each image did not end where the next one began. Once
the images were aligned to form a single image, a strip of scotch tape was placed on their boundaries to keep the image from falling apart.
The final image produced by these steps can be seen in Figure 1 at the back of the report.\\

Next, given a map of the moon at full moon, labelling was done on the newly created image of the moon near first quarter. The full moon map
contained a listing of all craters and mares on the moon as well as landing sites of the Apollo space missions. The items that were labelled and
seen on Figure 1 at the back of this report are as follows: Mare Tranquilitatas, Mare Fecunditatis, Mare Crisium, Pilinius, Macrobius, Maskelyne,
 Arago, Theophilis, 
Fracastorius, Langrenus, Goelenius. Apollo 11, Apollo 16, and Apollo 17. To map these objects was a simple process of aligning the new first quarter
image with the full moon image and finding the same craters on each image as the full moon map has craters labelled already and copying these labels
onto the new map.\\

Next, some calculations were carried out in regards to the size of craters on the moon. The following calculations can be found in Section~\ref{sec:calc}.
To obtain initial measurements of the craters on the newly formed first quarter 
moon picture, a simple measurement using a ruler was taken with the results being shown in Table~\ref{tab:craters}.
Two craters were then selected, one being relatively large and the other being relatively small. To find the size of these craters, Equation~\ref{eq:csize} was used,
and the final size of the craters found can be seen in Table~\ref{tab:craters}. 
After, calculations were preformed to determine the approximate size of the
meteorite which caused these craters. These calculations can be found in Section~\ref{sec:calc}, using Equation~\ref{eq:msize}. The results of these
calculations can also be found in Table~\ref{tab:craters}.  \\

Finally, some calculations were carried out on how often meteorites impact the moon and how that can be translated into how often meteorites
would impact the Earth which are bigger than 1km across. These calculations can be found in Section~\ref{sec:calc} using Equation~\ref{eq:cavg}. 
Also, a lookup was made given
information in Table~\ref{tab:waves} of whether or not a person would be safe if a 2km wide meteorite impacted Earth, 300km from Victoria,
standing on top of My. Doug. These results can also be found in Section~\ref{sec:qna}.\\


%----------------------------------------------------------------------------------------
%	SECTION 5
%----------------------------------------------------------------------------------------

\section{Observations}

All observations for this report were made on the day of 2012-Oct-29. As all calculations and measurements made inside of this report are based off of
given data, weather conditions and time of time have not been reported as they hold no effect. For all observations and
calculations made using both the near first quarter moon image, the image and labelling
can be found as Figure 1 in the back of this lab report. These observations include labels of 3 mares, 3 Apollo landing sites, and 9 various types of craters.
Other rough calculations and simple tables can be found at the back
of this lab report as well.

%----------------------------------------------------------------------------------------
%	SECTION 6
%----------------------------------------------------------------------------------------

\section{Tables and Measurements}
\label{sec:tnm}

\begin{table}[h]
\begin{center}
\begin{tabular}{l c c c}
\hline
Crater Size & Scale Size (mm) & Real Size (km) & Meteorite Real Size (km)\\
\hline
\hline
Small & 3mm & 38.338km & 1.534km\\
Large & 7mm & 89.456km & 3.578km\\
\hline
\end{tabular}
\end{center}
\caption{Measurements of impact craters on the moon.\label{tab:craters}}
\end{table}

\begin{table}[h!]
\begin{center}
\begin{tabular}{c c c c}
\hline
\hline
Crater Diameter (km) & 150 & 50 & 10\\
\hline
Energy (ergs) & 1 X 10$^{30}$ & 6 X 10$^{28}$ & 2.6 X 10$^{26}$\\
\hline
\hline
Distance (km) & Tsunami Height (meters) &  & \\
10000 & 100m & 15m & 1m\\
3000 & 250m & 40m & 3m\\
1000 & 540m & 90m & 5m\\
300 & 1300m & 200m & 15m\\
\hline
\end{tabular}
\end{center}
\caption{Crater and Tsunami Size for Various Impact Energies.\label{tab:waves}}
\end{table}

%----------------------------------------------------------------------------------------
%	SECTION 7
%----------------------------------------------------------------------------------------

\section{Calculations}
\label{sec:calc}

To calculate the real diameter of a given crater on the newly created first quarter moon image, Equation~\ref{eq:csize}
is used below. By simply taking a ratio of measured sizes in millimetres and multiplying it by the given size
of the moon in kilometres, the real size of the craters are obtained.

\begin{equation}
\label{eq:csize}
\text{D}_{Rcrater} = \frac{D_{Mcrater}}{D_{Mmoon}} * D_{Rmoon}
\end{equation}

To calculate the size of a meteorite which created a given impact crater, Equation~\ref{eq:msize} is used below.
Given the assumption that all craters are 25 times bigger than the meteorites which created them, a simple division
of the crater's diameter by 25 yields the asteroids approximate size.

\begin{equation}
\label{eq:msize}
\text{A}_{real} = \frac{{D}_{Rcrater}}{25}
\end{equation}

To calculate how often a meteorite of diameter larger than 1km will impact the Earth, Equation~\ref{eq:cavg} is
used below. Given the information that there are 29 impact craters that formed on the moon of this size in the last
3.5 billion years , and the Earth is 100 times larger than the moon's maria, a simple ratio of craters per year is formed.

\begin{equation}
\label{eq:cavg}
\text{craters} = \frac{29 craters}{3.5 billion years} * 100
\end{equation}


%----------------------------------------------------------------------------------------
%	SECTION 7
%----------------------------------------------------------------------------------------

\section{Questions}
\label{sec:qna}

The following questions and answers are asked inside of lab 9, Lunar Imaging, inside of the lab manual
for ASTR101. The questions have been repeated for the reader.

\begin{enumerate}
\item[Q.] How big was the meteorite, which made these [the small and large craters in Table~\ref{tab:craters}]
lunar craters, assuming the craters are 25 times bigger than the meteorite?
\item[A.] This question was answered using Equation~\ref{eq:msize} given the measurements made in Table~\ref{tab:craters}
in column Real Size. The final calculations for this question can be found in the same table in column 
Meteorite Real Size.
\item[Q.] In the Yucatan peninsula a crater has now been identified which is 190km in diameter and 65 million years old.
 Is this crater about the right size [to have killed off the dinosaurs]?
\item[A.] Given the assumption that the estimated asteroid to kill the dinosaurs was 10km in diameter and that all
craters are 25 times larger than the meteorites which caused them, it was found that this new crater was formed
by a meteorite 7.2km in diameter. It can be concluded that this meteorite is similar to the one that would
have killed off the dinosaurs.  In fact, this is the crater from a meteorite which is widely beleived to have
caused the mass extinction of the dinosaurs~\cite{Bates:1992}.
\item[Q.] There are 29 craters bigger than 25km in diameter on the lunar Maria. The maria are 3.5 billion years
old, so how often do craters of this size form on the maria?
\item[A.] To answer this question, Equation~\ref{eq:cavg} was used, which gives a simple answer of 8.28 craters per
billion years when not multiplying by the constant 100. This answer assumes that there is a constant rate
of cratering which we know isn't exactly true.
\item[Q.] If the area of the maria is about 5 million sq. km and, if the are of the Earth is 500 million sq. km, how 
often would you expect a crater of this size to be formed
on the Earth and end civilization?
\item[A.]  To answer this question, Equation~\ref{eq:cavg} was used, which gives a simple answer of 828 craters per
billion years. This answer assumes that there is a constant rate of cratering which we know isn't exactly true.
\item [Q.] Given a meteorite impacting the Earth which is 2km in diameter, how big a crater would it make?
How large would the tsunami be 300km from the impact sight?
Would you be safe on Mt. Doug (altitude=210m)?
\item [A.] Given that all meteorites cause craters 25 times larger than themselves, the crater would be 50km 
in diameter. Given the information in Table~\ref{tab:waves}, if the meteorite landed in the ocean, the 
tsunami would be 200m high at a distance of 300km from the impact sight. A person would just be safe on
top of Mt. Doug which has an elevation of 210m.
\end{enumerate}

%----------------------------------------------------------------------------------------
%	SECTION 8
%----------------------------------------------------------------------------------------

\section{Conclusions / Discussions}

This report has shown how to piece together multiple images of the same object in order to map the object's 
properties. This report has also shown how to calculate the size of craters on the moon, the size of the meteorites
which caused them as well as how often a large meteorite will impact the Earth.

%----------------------------------------------------------------------------------------
%	SECTION 9
%----------------------------------------------------------------------------------------

\section{Evaluation}

I found this report to have interesting properties for my own celestial observations. I have recently started
to observe the moon with my Celestron Super C8 plus telescope and have also begun to hunt for CCD cameras
in order to start astrophotography. I now see that some of the best images are actually composites of many 
images. This has factored into me looking at the software for the CCD as much as the hardware. I would
have liked to learn more about the Apollo missions and some of the experiments they carried out on the
surface of the moon and what other knowledge has been gained from the moon rocks that they brought back.


%----------------------------------------------------------------------------------------
%	BIBLIOGRAPHY
%----------------------------------------------------------------------------------------

\begin{thebibliography}{9}

\bibitem{Kleine:2005}
 Kleine, T.; Palme, H.; Mezger, K.; Halliday, A.N. (2005). "Hf–W Chronometry of Lunar Metals and the Age and Early Differentiation of the Moon". Science 310 (5754): 1671–1674.

\bibitem{Taylor:1998}
Taylor, G. Jeffrey (31 December 1998). "Origin of the Earth and Moon". Planetary Science Research Discoveries. Retrieved 7 April 2010.

\bibitem{Margot:1999}
Margot, J. L.; Campbell, D. B.; Jurgens, R. F.; Slade, M. A. (4 June 1999). "Topography of the Lunar Poles from Radar Interferometry: A Survey of Cold Trap Locations". Science 284 (5420): 1658–1660

\bibitem{Aaboe:1999}
 Aaboe, A.; Britton, J. P.; Henderson,, J. A.; Neugebauer, Otto; Sachs, A. J. (1991). "Saros Cycle Dates and Related Babylonian Astronomical Texts". Transactions of the American Philosophical Society (American Philosophical Society) 81 (6): 1–75.

\bibitem{Lewis:1964}
 Lewis, C. S. (1964). The Discarded Image. Cambridge: Cambridge University Press. p. 108.

\bibitem{Bates:1992}
Bates, Robin (series producer), Chesmar, Terri and Baniewicz, Rich (associate producers) (1992). The Dinosaurs! Episode 4: "Death of the Dinosaur" (TV-series). PBS Video, WHYY-TV.

\end{thebibliography}

\end{document}