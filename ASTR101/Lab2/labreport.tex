%%%%%%%%%%%%%%%%%%%%%%%%%%%%%%%%%%%%%%%%%
% Laboratory Report LaTeX Template
%
% This template has been downloaded from:
% http://www.latextemplates.com
%
%%%%%%%%%%%%%%%%%%%%%%%%%%%%%%%%%%%%%%%%%

%----------------------------------------------------------------------------------------
%	DOCUMENT CONFIGURATIONS
%----------------------------------------------------------------------------------------

\documentclass{article}

\usepackage{wasysym}
\usepackage{amssymb}% http://ctan.org/pkg/amssymb
\usepackage{pifont}% http://ctan.org/pkg/pifont
\newcommand{\cmark}{\ding{51}}
\newcommand{\xmark}{\ding{55}}

\title{Apparent Positions of The Planets \\ ASTR 101} % Title

\author{Jordan \textsc{Ell} \\ jordan.ell7@gmail.com \\ V00660306} % Author name

\begin{document}

\maketitle % Insert the title, author and date

\begin{tabular}{lr}
Date Performed: 24/09/2012 \\ % Date the experiment was performed and partner's name
Supervisor: Jillian Scudder % Instructor/supervisor
\end{tabular}

\setlength\parindent{0pt} % Removes all indentation from paragraphs

\renewcommand{\labelenumi}{\alph{enumi}.} % Make numbering in the enumerate environment by letter rather than number (e.g. section 6)

%----------------------------------------------------------------------------------------
%	SECTION 1
%----------------------------------------------------------------------------------------

\section{Objective}

This report will demonstrate the proper procedure of finding the five visible planets apparent positions in the sky.
It explains how to map the planets on a polar coordinate graph and ecliptic constellation chart as well as the proper 
use and data collection from a planisphere.\\
 
%----------------------------------------------------------------------------------------
%	SECTION 2
%----------------------------------------------------------------------------------------

\section{Introduction}

Being able to locate the planets inside our solar system is a key task during astronomical observations. 

%----------------------------------------------------------------------------------------
%	SECTION 3
%----------------------------------------------------------------------------------------

\section{Equipment}

The three main pieces of equipment required for this lab are as follows. First, a polar coordinate graph which is used for
charting where planets are relative to the sun at any given point. This report will use the chart with measuring 1 AU as 
one dark black ring from the centre. Second, a constellation equatorial chart. This chart has measurements in declination, 
measured in degrees, and right ascension, measured in hours and minutes. Lastly, a planisphere is needed, which is a
quick reference to stars given a day and time of night. Other common equipment needed for this lab includes: protractor,
scissors, and coloured pencils.



%----------------------------------------------------------------------------------------
%	SECTION 4
%----------------------------------------------------------------------------------------

\section{Procedure}

Given the data inside of Table~\ref{tab:long} and Table~\ref{tab:radii}, a heliocentric sketch was made of Venus', Earth's, Mars', 
Jupiter's, and Saturn's position at the dates of 2012-Sept-22, 2012-No-21, and 2013-Mar-20. This was accomplished by first
looking up a planet in Table~\ref{tab:radii} to find its orbit radius (measured in A.U) in order to give it a distance away from the
sun. We are assuming perfect circles for orbits. Next, a lookup of the planet's heliocentric longitude was preformed on 
Table~\ref{tab:long}. With these two coordinates, a sketch of where the planet should be located was able to be made on the 
polar coordinate graph given that the sun was placed at the centre and that each dark black line represented 1 A.U. This process
was repeated for all five planets and all three dates listed in Table~\ref{tab:long}\\

Next, the visibility of each planet given the four times of day: noon, sunset, midnight, and sunrise are made. This was accomplished
using a small piece of paper as a guide for East and West on the Earth placed over top of the Earth in the sketch made above. Knowing,
the Earth rotates counter clockwise, if viewed from above, the paper guide was rotated to simulate a field of vision on the Earth at
the four given times of day. A planet's visibility can then be determined by locating it (visible) or not locating it (not visible) in the
180$^\circ$ view provided by the paper guide. These results can be found in Table~\ref{tab:tod}.\\

Now, the determination of the planet's alignment is made. By looking at sketch made above, it was determined (for the day of
2012-Sept-22) what a planet's alignment is relative to the Earth and Sun. This was done purely by observation and with 
approximating the planet's actual position to that of a specific alignment. The results from these observations can be found
in the Configuration column of Table~\ref{tab:tod}.\\

With the now sketched heliocentric view of planetary positions, geocentric equatorial positions of the planets were then
calculated for the day of 2012-Sept-22. This was accomplished by first measuring the geocentric ecliptic longitudes of the planets.
This was calculated by measuring the angle between the Earth and the planets in question, noting that the Earth is now the origin 
of measurement as opposed to the sun and that angles are measured from. These measurements can be found in the 
Ecliptic Long. column of Table~\ref{tab:pos}. Next, a plot of the ecliptic longitudes was made onto the provided SC001
constellation chart. This was accomplished by simply mapping the found ecliptic longitude above with the given degrees on
the ecliptic on the SC001. Finally, the right ascension (measured in hours and minutes) and declination (measured in degrees)
were measured off the SC001 chart for the Sun and given planets. The results of these readings can be found in Table~\ref{tab:pos}.
These results are our final geocentric equatorial positions of the planets.\\

Lastly, minor approximate measurements were made using a Planisphere. The Planisphere was used by placing a provided circular
map of the stars with labelled months and days of the year underneath an ellipse which has times of night and rotating the 
circular map of the stars so that your desired time and day match up. What is shown through the ellipse of the stars is what will 
be visible from the given latitude of the planisphere. The measurements taken can be found at the bottom of 
Section~\ref{sec:tnm}.\\


%----------------------------------------------------------------------------------------
%	SECTION 5
%----------------------------------------------------------------------------------------

\section{Observations}



%----------------------------------------------------------------------------------------
%	SECTION 6
%----------------------------------------------------------------------------------------

\section{Tables and Measurements}
\label{sec:tnm}

\begin{table}[h!]
\begin{center}
\begin{tabular}{l c c c c c }
\hline
Date & Venus & Earth & Mars & Jupiter & Saturn\\
\hline
\hline
2012-Sep-22 & 068 & 000 & 261 & 065 & 211\\
2012-Nov-21 & 166 & 060 & 297 & 070 & 213\\
2013-Mar-20 & 354 & 180 & 011 & 081 & 217\\
\hline
\end{tabular}
\end{center}
\caption{Heliocentric Longitudes from Astronomical Almanac.\label{tab:long}}
\end{table}

\begin{table}[h!]
\begin{center}
\begin{tabular}{ c c c c }
\hline
Planet & Orbit Radius (A.U) & Period (years) & Symbol\\
\hline
\hline
Sun &  &  & \astrosun \\
Venus & 0.72 & 0.62 & \venus\\
Earth & 1.00 & 1.00 & \earth\\
Mars & 1.52 & 1.88 & \mars\\
Jupiter & 5.20 & 11.86 & \jupiter\\
Saturn & 9.54 & 29.46 & \saturn\\
\hline
\end{tabular}
\end{center}
\caption{Radii and Period of Orbits.\label{tab:radii}}
\end{table}

\begin{table}[h!]
\begin{center}
\begin{tabular}{| c | c | c | c | c | c |}
\hline
Planet & Noon & Sunset & Midnight & Sunrise & Configuration\\
\hline
\hline
Venus & \cmark & & & \cmark & Greatest Western Elongation \\
Mars & \cmark & \cmark & & & Greatest Eastern Elongation\\
Jupiter & & & \cmark & \cmark & Quadrature\\
Saturn & \cmark & \cmark & & & Conjunction \\
\hline
\end{tabular}
\end{center}
\caption{Planets as seen from the Earth.\label{tab:tod}}
\end{table}

\begin{table}[h!]
\begin{center}
\begin{tabular}{| c | c | c | c | c | }
\hline
Planet & Ecliptic Long. & Constellation & Right Ascension & Declination\\
\hline
\hline
Sun & 180$^\circ$ & Virgo & 00:00 & 0$^\circ$. \\
Venus & 134$^\circ$ & Cancer & 09:02 & 16$^\circ$. \\
Mars & 230$^\circ$ & Libra & 15:10 & -17$^\circ$. \\
Jupiter & 73.5$^\circ$ & Taurus & 04:41 & 21$^\circ$. \\
Saturn & 207$^\circ$ & Virgo & 13:42 & -11$^\circ$. \\
\hline
\end{tabular}
\end{center}
\caption{Geocentric Equatorial Position of the Planets.\label{tab:pos}}
\end{table}

%----------------------------------------------------------------------------------------
%	BIBLIOGRAPHY
%----------------------------------------------------------------------------------------

\begin{thebibliography}{9}

\bibitem{Smith:2012qr}
Smith, J.~M. and Jones, A.~B. (2012).
\newblock {\em Chemistry}.
\newblock Publisher, City, 7th edition.

\end{thebibliography}

\end{document}