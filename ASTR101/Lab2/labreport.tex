%%%%%%%%%%%%%%%%%%%%%%%%%%%%%%%%%%%%%%%%%
% Laboratory Report LaTeX Template
%
% This template has been downloaded from:
% http://www.latextemplates.com
%
%%%%%%%%%%%%%%%%%%%%%%%%%%%%%%%%%%%%%%%%%

%----------------------------------------------------------------------------------------
%	DOCUMENT CONFIGURATIONS
%----------------------------------------------------------------------------------------

\documentclass{article}

\usepackage{wasysym}
\usepackage{amssymb}% http://ctan.org/pkg/amssymb
\usepackage{pifont}% http://ctan.org/pkg/pifont
\newcommand{\cmark}{\ding{51}}
\newcommand{\xmark}{\ding{55}}

\usepackage{graphicx}

\title{Apparent Positions of The Planets \\ ASTR 101} % Title

\author{Jordan \textsc{Ell} \\ jordan.ell7@gmail.com \\ V00660306} % Author name

\begin{document}

\maketitle % Insert the title, author and date

\begin{tabular}{lr}
Date Performed: 24/09/2012 \\ % Date the experiment was performed and partner's name
Supervisor: Jillian Scudder % Instructor/supervisor
\end{tabular}

\setlength\parindent{0pt} % Removes all indentation from paragraphs

\renewcommand{\labelenumi}{\alph{enumi}.} % Make numbering in the enumerate environment by letter rather than number (e.g. section 6)

%----------------------------------------------------------------------------------------
%	SECTION 1
%----------------------------------------------------------------------------------------

\section{Objective}

This report will demonstrate the proper procedure of finding planets apparent positions in the sky.
It explains how to map the planets on a polar coordinate graph and ecliptic constellation chart as well as the 
use of a planisphere.\\
 
%----------------------------------------------------------------------------------------
%	SECTION 2
%----------------------------------------------------------------------------------------

\section{Introduction}

Observation of the planets dates is an important task in the field of astronomy. The term planet comes from Ancient Greek
meaning "wandering star". Ancient tribes first spotted these stars moving against the backdrop of other stars and have been
since studied for their orbits, elemental make-up as well as other properties. While most planets in our solar system have been
discovered by simply looking up at the sky, some have also been discovered by mathematics (Uranus). As science has evolved,
our model for the planets has moved from and Earth to Sun centred, which has drastic effects on the planet's orbits. After having
read this report, our view of how to find the planets in the sky should become clear as well as having a broad understanding
of their orbits in both distance and period.

%----------------------------------------------------------------------------------------
%	SECTION 3
%----------------------------------------------------------------------------------------

\section{Equipment}

The three main pieces of equipment required for this lab are as follows. First, a polar coordinate graph which is used for
charting where planets are relative to the sun at any given point. This report will use the chart with measuring 1 AU as 
one dark black ring from the centre. Second, a constellation equatorial chart. This chart has measurements in declination, 
measured in degrees, and right ascension, measured in hours and minutes. Lastly, a planisphere is needed, which is a
quick reference to stars given a day and time of night. Other common equipment needed for this lab includes: protractor,
scissors, and coloured pencils.



%----------------------------------------------------------------------------------------
%	SECTION 4
%----------------------------------------------------------------------------------------

\section{Procedure}

Given the data inside of Table~\ref{tab:long} and Table~\ref{tab:radii}, a heliocentric sketch was made of Venus', Earth's, Mars', 
Jupiter's, and Saturn's position at the dates of 2012-Sept-22, 2012-No-21, and 2013-Mar-20. This was accomplished by first
looking up a planet in Table~\ref{tab:radii} to find its orbit radius (measured in A.U) in order to give it a distance away from the
sun. We are assuming perfect circles for orbits. Next, a lookup of the planet's heliocentric longitude was preformed on 
Table~\ref{tab:long}. With these two coordinates, a sketch of where the planet should be located was able to be made on the 
polar coordinate graph given that the sun was placed at the centre and that each dark black line represented 1 A.U. This process
was repeated for all five planets and all three dates listed in Table~\ref{tab:long}\\

Next, the visibility of each planet given the four times of day: noon, sunset, midnight, and sunrise are made. This was accomplished
using a small piece of paper as a guide for East and West on the Earth placed over top of the Earth in the sketch made above. Knowing,
the Earth rotates counter clockwise, if viewed from above, the paper guide was rotated to simulate a field of vision on the Earth at
the four given times of day. A planet's visibility can then be determined by locating it (visible) or not locating it (not visible) in the
180$^\circ$ view provided by the paper guide. These results can be found in Table~\ref{tab:tod}.\\

Now, the determination of the planet's alignment is made. By looking at sketch made above, it was determined (for the day of
2012-Sept-22) what a planet's alignment is relative to the Earth and Sun. This was done purely by observation and with 
approximating the planet's actual position to that of a specific alignment. The results from these observations can be found
in the Configuration column of Table~\ref{tab:tod}.\\

With the now sketched heliocentric view of planetary positions, geocentric equatorial positions of the planets were then
calculated for the day of 2012-Sept-22. This was accomplished by first measuring the geocentric ecliptic longitudes of the planets.
This was calculated by measuring the angle between the Earth and the planets in question, noting that the Earth is now the origin 
of measurement as opposed to the sun and that angles are measured from. These measurements can be found in the 
Ecliptic Long. column of Table~\ref{tab:pos}. Next, a plot of the ecliptic longitudes was made onto the provided SC001
constellation chart. This was accomplished by simply mapping the found ecliptic longitude above with the given degrees on
the ecliptic on the SC001. Finally, the right ascension (measured in hours and minutes) and declination (measured in degrees)
were measured off the SC001 chart for the Sun and given planets. The results of these readings can be found in Table~\ref{tab:pos}.
These results are our final geocentric equatorial positions of the planets.\\

Lastly, minor approximate measurements were made using a Planisphere. The Planisphere was used by placing a provided circular
map of the stars with labelled months and days of the year underneath an ellipse which has times of night and rotating the 
circular map of the stars so that your desired time and day match up. What is shown through the ellipse of the stars is what will 
be visible from the given latitude of the planisphere. The measurements taken can be found at the bottom of 
Section~\ref{sec:tnm}.\\


%----------------------------------------------------------------------------------------
%	SECTION 5
%----------------------------------------------------------------------------------------

\section{Observations}

All observations for this report were made on the day of 24/09/2012. As this report is based off of given numerical data as
well as paper observations, weather conditions and time of day have not been reported as they hold no effect. For all 
observations made using the polar coordinate chart, the chart and its sketches can be found as Figure 1 in the back of
this lab report. These observations include: the planet's position at the given dates in Table~\ref{tab:long}, ecliptic 
longitude of planets, and approximations of planet's alignment. All observations made using the SC001 constellation
chart can be found as Figure 2 in the back of this report. These observations include both the measurements of right
ascension and declination of the sun and planets.\\

%----------------------------------------------------------------------------------------
%	SECTION 6
%----------------------------------------------------------------------------------------

\section{Tables and Measurements}
\label{sec:tnm}

\begin{table}[h!]
\begin{center}
\begin{tabular}{l c c c c c }
\hline
Date & Venus & Earth & Mars & Jupiter & Saturn\\
\hline
\hline
2012-Sep-22 & 068 & 000 & 261 & 065 & 211\\
2012-Nov-21 & 166 & 060 & 297 & 070 & 213\\
2013-Mar-20 & 354 & 180 & 011 & 081 & 217\\
\hline
\end{tabular}
\end{center}
\caption{Heliocentric Longitudes from Astronomical Almanac.\label{tab:long}}
\end{table}

\begin{table}[h!]
\begin{center}
\begin{tabular}{ c c c c }
\hline
Planet & Orbit Radius (A.U) & Period (years) & Symbol\\
\hline
\hline
Sun &  &  & \astrosun \\
Venus & 0.72 & 0.62 & \venus\\
Earth & 1.00 & 1.00 & \earth\\
Mars & 1.52 & 1.88 & \mars\\
Jupiter & 5.20 & 11.86 & \jupiter\\
Saturn & 9.54 & 29.46 & \saturn\\
\hline
\end{tabular}
\end{center}
\caption{Radii and Period of Orbits.\label{tab:radii}}
\end{table}

\begin{table}[h!]
\begin{center}
\begin{tabular}{| c | c | c | c | c | c |}
\hline
Planet & Noon & Sunset & Midnight & Sunrise & Configuration\\
\hline
\hline
Venus & \cmark & & & \cmark & Greatest Western Elongation \\
Mars & \cmark & \cmark & & & Greatest Eastern Elongation\\
Jupiter & & & \cmark & \cmark & Quadrature\\
Saturn & \cmark & \cmark & & & Conjunction \\
\hline
\end{tabular}
\end{center}
\caption{Planets as seen from the Earth.\label{tab:tod}}
\end{table}

\begin{table}[h!]
\begin{center}
\begin{tabular}{| c | c | c | c | c | }
\hline
Planet & Ecliptic Long. & Constellation & Right Ascension & Declination\\
\hline
\hline
Sun & 180$^\circ$ & Virgo & 12:00 & 0$^\circ$. \\
Venus & 134$^\circ$ & Cancer & 09:02 & 16$^\circ$. \\
Mars & 230$^\circ$ & Libra & 15:10 & -17$^\circ$. \\
Jupiter & 73.5$^\circ$ & Taurus & 04:41 & 21$^\circ$. \\
Saturn & 207$^\circ$ & Virgo & 13:42 & -11$^\circ$. \\
\hline
\end{tabular}
\end{center}
\caption{Geocentric Equatorial Position of the Planets.\label{tab:pos}}
\end{table}

%----------------------------------------------------------------------------------------
%	SECTION 7
%----------------------------------------------------------------------------------------

\section{Questions}

The following questions and answers are asked throughout the lab manual for ASTR101 Lab 2. The
questions have been repeated for the reader.

\begin{enumerate}
\item[Q.] Why does Venus go through phases? 
\item[A.] Venus has phases similar to that of the sun because of the view the Earth has of Venus as it travels
around the sun \textit{inside} of the orbit of the Earth around the sun. The different phases are caused by the different
perspective we have on Earth of the sun shining on Venus. Figure~\ref{fig:pov} below shows how the different phases of
Venus appear from Earth.

\addtocounter{figure}{2}
\begin{figure}
\centering
\includegraphics[width=0.4\textwidth]{images/blank}
\caption{The phases of Venus.\label{fig:pov}}
\end{figure}

\item[Q.] In which zodiacal constellation was the sun located when you were born? What is your astrological sign?
Check with your partners and discuss any discrepancy.
\item[A.] As per Figure 2, the sun was located in Gemini when I was born on July 12th. This is a discrepancy with
my astrological sign of Cancer. This difference is caused by the precession of the Earth.
\end{enumerate}

The following answers are labelled to coincide with the lab manual for ASTR101 Lab 2 section
"The Use of a Planisphere". As
questions 1 and 3 are instructions rather than actual questions/measurements, they have been skipped. All answers 
are given using the provided planisphere and approximations.

\begin{enumerate}
\item[2.] While the circular dial is being rotated, the star Polaris does not seem to move. 
\item[4.] At the time of 12AM on the night of June 01, the star Vega, which is apart of the constellation Lyra, appears to 
be in the centre (Zenith) of the planisphere.
\item[5.] At the time of 12AM on the night of June 01, the star Spica, which is apart of the constellation Virgo, appears
to be on the "Western Horizon" of the planisphere.
\item[6.] At the time of 11PM on the night of June 15, the star Vega again appears at the Zenith while the star Spica
again appears at the "Western Horizon" of the planisphere.
\item[7.] At the time of 11PM on the night of August 15, the star Deneb, which is apart of the constellation Cygnus, appears
to be at the Zenith, while the star Arcturus appears to be at the "Western Horizon" of the planisphere.
\item[8.] On the night of September 22nd, the star Antares will set at approximately 9PM. The star Antares will also
set at 1AM on the night of July 30th.
\item[9.] On the night of January 1st, the star Sirius will rise at approximately 7PM and will set at approximately 5AM, 
giving a total time of 10 hours above the horizon.
\end{enumerate}

%----------------------------------------------------------------------------------------
%	SECTION 8
%----------------------------------------------------------------------------------------

\section{Conclusions/Discussions}

This report has shown how to locate a planet in the sky given initial data points for orbit radius and heliocentric 
longitudes. This report has also shown the use of a planisphere for quick star reference at a given latitude.

%----------------------------------------------------------------------------------------
%	SECTION 9
%----------------------------------------------------------------------------------------

\section{Evaluation}

I found this report very useful for my own observational interest. I recently acquired a Celestron Super C8 Plus and have
been interested in observing the planets but have not known how to properly find them in the sky. I now have the 
required skills to be able to find planets and make predictions of their whereabouts throughout the year. However, I would
have liked to have had the opportunity to actually take a measurement of the planet rather than just have the initial
data provided.

%----------------------------------------------------------------------------------------
%	BIBLIOGRAPHY
%----------------------------------------------------------------------------------------

\begin{thebibliography}{9}

\bibitem{Smith:2012qr}
Smith, J.~M. and Jones, A.~B. (2012).
\newblock {\em Chemistry}.
\newblock Publisher, City, 7th edition.

\end{thebibliography}

\end{document}