%%%%%%%%%%%%%%%%%%%%%%%%%%%%%%%%%%%%%%%%%
% Laboratory Report LaTeX Template
%
% This template has been downloaded from:
% http://www.latextemplates.com
%
%%%%%%%%%%%%%%%%%%%%%%%%%%%%%%%%%%%%%%%%%

%----------------------------------------------------------------------------------------
%	DOCUMENT CONFIGURATIONS
%----------------------------------------------------------------------------------------

\documentclass{article}

\usepackage{amssymb, amsmath}

\title{Solar Rotation \\ ASTR 101} % Title

\author{Jordan \textsc{Ell} \\ jordan.ell7@gmail.com \\ V00660306} % Author name

\begin{document}

\maketitle % Insert the title, author and date

\begin{tabular}{lr}
Date Performed: 15/10/2012\\ % Date the experiment was performed and partner's name
Instructor: Jillian Scudder % Instructor/supervisor
\end{tabular}

\setlength\parindent{0pt} % Removes all indentation from paragraphs

\renewcommand{\labelenumi}{\alph{enumi}.} % Make numbering in the enumerate environment by letter rather than number (e.g. section 6)

%----------------------------------------------------------------------------------------
%	SECTION 1
%----------------------------------------------------------------------------------------

\section{Objective}

This report will demonstrate how observations of sunspots can be used to determine the rotational period of the Sun. It will
also explain how two dimensional pictures can be used to calculate three dimensional angles of rotation across the surface 
of the Sun.\\
 
%----------------------------------------------------------------------------------------
%	SECTION 2
%----------------------------------------------------------------------------------------

\section{Introduction}

The Sun is the star located at the centre of our solar system which the Earth and all other planets orbit. The sun is about 109 time larger 
than the Earth and accounts for about 99.86\% of mass in the entire solar system~\cite{Wolf:2000}. The Sun mostly consists of hydrogen
and helium~\cite{Basu:2008}, but is more known for being the star that heats our planet and provides us with light. As was with Aristotle
and Plato, it was once believed that all heavenly bodies were perfect and without deformation. This being the case, it was impossible
for observed sunspots to actually be on the Sun, but rather lie in our line of sight directly to the Sun. This being the case, it was thought
that the sunspots did not show rotation of the Sun. However, as telescopic views of the Sun became more common with Thomas Harriot
and the more known Galileo, sunspots were observed to be on the surface of the Sun and showed various aspects such as the rotation
of the Sun and proved that the Sun was not a perfect celestial sphere as thought by Aristotle. The Sun has also been the key to unlocking
many mysteries in science such as the light spectrum~\cite{BBC:2006} and radioactive decay for energy~\cite{Darden:1998}. While these
discoveries are great contributions to the world of science, the lab will focus on attempting to duplicate the given result that the Sun
rotates once every 27 days. In doing so we make an assumption that observed sunspots do not move relative to the Sun, sunspots are
indeed located on the surface of the Sun, and that the Earth's rotation around the Sun does not effect our given data (change given
angles).\\

%----------------------------------------------------------------------------------------
%	SECTION 3
%----------------------------------------------------------------------------------------

\section{Equipment}

The equipment used for this lab is as follows: a sheet of semi-transparent paper,  a sheet of opaque paper, ruler, protractor and
 a drafting compass. Initial images of the Sun in the date range of 2012-Sept-27 to 2012-Oct-04 were provided by the instructor.\\

%----------------------------------------------------------------------------------------
%	SECTION 4
%----------------------------------------------------------------------------------------

\section{Procedure}

Given photographs of the Sun taken at the same distance (the Sun appears to be the same size in all images), a sketch of each
photograph was made on the sheet of semi-transparent paper. Each sketched was placed "on top" of the last, meaning that
the same sheet of semi-transparent paper was used for all 8 sketches and each sketch is simply done on top of the last.
Each sketch represents a date in the range of 2012-Sept-27 to 2012-Oct-04.  To
ensure that the Sun's rotational angle was consistent across all 8 sketches, markings in four corners of the semi-transparent
sheet were made to ensure proper alignment the it was moved from photograph to photograph. For each sketch, the only 
points of interest drawn were the outline of the Sun and an outline of each sunspot present in each photograph. Each sunspot
was labelled with the data of the photograph. To see the time that the photograph was taken, see Table~\ref{tab:time}.
These times will be used in future calculations.\\

Now that all major sunspots have been labelled on the semi-transparent paper, a single common sunspot was selected from
each of the 8 sketches to ensure we are making the future calculations based on a single sunspots movements and not
interchanging them. After this selection, a line is drawn from one edge of the Sun's outline to the other which best represents
this particular sunspot's movement across the surface of the Sun. At this point, any sunspots that were not directly on the
path of the newly drawn line, had perpendicular lines drawn from the newly formed line to the sunspot. This is to ease
measurements to come.\\

Next, the sheet of opaque paper was taped onto the semi-transparent paper in a fashion such that an edge of the opaque
paper follows the same path as the line drawn above. At this point, a measurement was taken of the sunspot path line to 
determine its middle point in regards to the boundaries set by the Sun's outline. Once the middle was determine and marked
with an "M", the 
drafting compass was set to this width and placed at the middle location. From here a semi-circle was traced onto the 
opaque sheet of paper that began where the outline of the sun intersected with the horizontal line drawn above and
continued around to the opposite side of the Sun's outline and intersection of the horizontal line. If this newly formed
semi-circle was held at a 90$^\circ$ angle to the semi-transparent sheet of paper, the opaque sheet would represent
the three dimensional space of the outline of the Sun.\\

With the opaque sheet of paper flat on the semi-transparent sheet, dotted lines were drawn from the 8 sunspot locations
on the horizontal line to the perimeter of the newly traced semi-circle. These new locations are representative of the 
sunspots location in three dimensions. Now a solid line is drawn from the middle point of the horizontal line, determined
earlier, to the new location of each sunspot along the perimeter of the new semi-circle. These lines represent the angle
from the centre of the Sun to the sunspot.\\

Next, the angles between the lines for each sunspot to the centre of the Sun were determined. This was done by calculating
each adjacent date's sunspot angle. This was accomplish using the protractor (i.e. The angle between 2012-Sept-27 and 
2012-Sept-28 was measured). The results for these measurements can be seen in Table~\ref{tab:main} column 
$\Delta$ $\theta$ (degrees). Next, the times between photographs were calculated from the given photographs. The date
and time information from these photographs can be found in Table~\ref{tab:time} The results
for these calculations can be found in Table~\ref{tab:main} column $\Delta$ T (days). Finally, the period change was calculated
using the previous two figured obtained using Equation~\ref{eq:per}. These calculations can be found in Table~\ref{tab:main} column 
P (days).\\

Finally, the rotational period of the Sun was calculated by using Equation~\ref{eq:avg} and uncertainty using Equation~\ref{eq:un}.
The results for these values can be found in Section~\ref{sec:calc}.


%----------------------------------------------------------------------------------------
%	SECTION 5
%----------------------------------------------------------------------------------------

\section{Observations}

All observations for this report were made on the day of 2012-Oct-15. All photographs of the Sun were made between the
days of 2012-Sept-27 to 2012-Oct-04. As all calculations and measurements made inside of this report are based off of
given data, weather conditions and time of time have not been reported as they hold no effect. For all observations and
calculations made using both the semi-transparent sheet of paper and opaque sheet of paper, the drawn diagrams
can be found as Figure 1 in the back of this lab report. These observations include sketches of all 8 Sun photographs, 
the horizontal path of sunspots, three dimensional projects of location and angle as well as the determined centre of
the horizontal path of sunspot movements. Other rough calculations and simple tables can be found at the back
of this lab report as well.

%----------------------------------------------------------------------------------------
%	SECTION 6
%----------------------------------------------------------------------------------------

\section{Tables and Measurements}
\label{sec:tnm}

\begin{table}[h]
\begin{center}
\begin{tabular}{l c c c}
\hline
Date & Time (days) \\
\hline
\hline
2012-Sept-27 & 27.11 \\
2012-Sept-28 & 28.21 \\
2012-Sept-29 & 29.21 \\
2012-Sept-30 & 30.69 \\
2012-Oct-01 & 1.10 \\
2012-Oct-02 & 2.18 \\
2012-Oct-03 & 3.74 \\
2012-Oct-04 & 4.19 \\
\hline
\end{tabular}
\end{center}
\caption{Time of pre-existing measurements taken on their given days.\label{tab:time}}
\end{table}

\begin{table}[h!]
\begin{center}
\begin{tabular}{l c c c}
\hline
Date & $\Delta$ T (days) & $\Delta$ $\theta$ (degrees) & P (days) \\
\hline
\hline
2012-Sept-27 & - & - & - \\
2012-Sept-28 & 1.10 & 18$^\circ$ & 22.0 \\
2012-Sept-29 & 1.00 & 15$^\circ$ & 24.0 \\
2012-Sept-30 & 1.48 & 20.5$^\circ$ & 26.0 \\
2012-Oct-01 & 0.41 & 7$^\circ$ & 21.1 \\
2012-Oct-02 & 1.08 & 14$^\circ$ & 27.8 \\
2012-Oct-03 & 1.56 & 18.5$^\circ$ & 30.4 \\
2012-Oct-04 & 0.45 & 10$^\circ$ & 16.2 \\
\hline
\end{tabular}
\end{center}
\caption{Measurements of sunspot movement over various time periods.\label{tab:main}}
\end{table}

%----------------------------------------------------------------------------------------
%	SECTION 7
%----------------------------------------------------------------------------------------

\section{Calculations}
\label{sec:calc}

To calculate the period of rotation, in days, for the Sun, Equation~\ref{eq:per} was used to obtain the period from
the change in time and the change in angle of the sunspot in between two adjacent photographs.

\begin{equation}
\label{eq:per}
\text{P} = \frac{\Delta\text{T} * 360^\circ}{\Delta\theta}
\end{equation}

To calculate the uncertainty (U) of the measurements, Equation~\ref{eq:un} was used by subtracting our second 
smallest period calculation from the second largest then dividing by two. The numbers used for this calculation
can be found in Table~\ref{tab:main} in column P (days). This calculation yields the result of 3.35 days.

\begin{equation}
\label{eq:un}
\text{U} = \frac{27.8 - 21.1}{2}
\end{equation}

To calculate the final result, which is the rotational period of the Sun, the previous periods calculated in 
Table~\ref{tab:main} in column P (days), and summed and divided by total calculations as seen in
Equation~\ref{eq:avg}. This creates our final result ($P_avg$) as an average of all data calculated. This
equation yields the result of 23.9 days.

\begin{equation}
\label{eq:avg}
\text{P}_{avg} = \frac{\sum\text{P}}{7}
\end{equation}

To calculate the size of a sunspot ($l_real$), Equation~\ref{eq:size} is used, where l is the measured size of
a sunspot on our sketch, D is the measured size of the Sun of the sketch and $D_real$ is the real size of
the Sun. The assumed size of the sun is given inside of the lab manual as well as the size of the Earth for comparison.
With the measured size of a sunspot being 3.5mm and the measured size of the Sun being 141mm, the 
result yields 34,565.6 km.

\begin{equation}
\label{eq:size}
\text{l}_{real} = \frac{l * \text{D}_{real}}{D}
\end{equation}


%----------------------------------------------------------------------------------------
%	SECTION 7
%----------------------------------------------------------------------------------------

\section{Questions}

The following questions and answered are asked inside of lab 11, Solar Rotation, inside of the lab manual
for ASTR101. The questions have been repeated for the reader.

\begin{enumerate}
\item[Q.] The accepted value of the observed rotational period of the sun is 27 days at the equator and 28 days
at 30 degrees latitude. Compare your determination of the period with these values, giving your best estimate
of the uncertainty in your determination.
\item[A.] As seen with Equations~\ref{eq:avg}\ref{eq:un}, the period of the Sun's rotation was found to be
23.9 days with an uncertainty of 3.35 days. Our estimate falls short of the generally accepted value of 27 days,
however, if maximum positive uncertainty is applied, a result of 27.25 days is found and can be accepted. The
large uncertainty here can be largely attributed to the inaccuracy of the hand drawn sketches.
\item[Q.] What are we assuming about sunspots, which is fundamental to determining the rotation of the Sun
by this method? How might you investigate the validity of this assumption?
\item[A.] This report makes large use of the assumption that the sunspots observed are actually located on
the surface of the Sun as opposed to being between our line of sight and the Sun's surface which was previously
though by the likes of Aristotle~\cite{Aristotle:2012}. This theory can be easily tested with the notion of parallax.
If the sunspot was not on the surface be rather in front of the Sun, parallax should be detectable when viewed from
different positions on the Earth or as the Earth orbits the Sun. Without the observation of parallax it should be
determined that the sunspots are indeed on the surface of the Sun.
\item[Q.] In a paragraph describe how the sunspots change in size, shape and number over the few days of observation.
\item[A.] An easy conclusion to draw for the number of sunspots visible is that as the Sun rotates, sunspots approach
the "horizon", edge of the Sun's outline", then disappear to the far side of the Sun. This causes the number of sunspots
to change over time. This is also the root cause of the sunspots apparent change in shape and size. As the Sun rotates,
the sunspots are viewed from different angles on Earth. Near the horizon they may appear as a thin tall line, but 
while viewed at the centre of the Sun, they may appear as a large circular blob. This explanation gives the sunspots
their apparent transitions in size and shape.
\item[Q.] Assuming the diameter of the Sun is 1,392,500 km, how big is a sunspot? Compare it to the diameter
of the Earth (12,765 km).
\item[A.] As seen with Equation~\ref{eq:size}, the size of a sunspot is 34,565.6 km. In comparison to the Earth
which is 12,765 km, the sunspot is roughly 2.7 times larger than the Earth.
\end{enumerate}

%----------------------------------------------------------------------------------------
%	SECTION 8
%----------------------------------------------------------------------------------------

\section{Conclusions / Discussions}

This reports has shown how to calculate the period of the Sun's rotation using observed sunspots and their
apparent motion over time. This report has also shown how two dimensional observations can be
transformed to take three dimensional calculations.

%----------------------------------------------------------------------------------------
%	SECTION 9
%----------------------------------------------------------------------------------------

\section{Evaluation}

I found this report to have interesting properties for my own celestial observations. I have recently started to
observe planets with my Celestron Super C8 Plus telescope and have noticed slight variations in how the planet
appears over time (especially with Mars). With the knowledge learned from this lab I will be able to carry out
similar experiments in my own time with land features of planets in order to determine their own rotational periods.
I would have liked to learn more about the internal structure of the Sun and possibly the causes of the nuclear furnace
which heats us every day, or perhaps learn more about the sunspots themselves as opposed to just viewing them
as a type of black box.


%----------------------------------------------------------------------------------------
%	BIBLIOGRAPHY
%----------------------------------------------------------------------------------------

\begin{thebibliography}{9}

\bibitem{Wolf:2000}
Woolfson, M (2000). "The origin and evolution of the solar system". Astronomy and Geophysics 41

\bibitem{Basu:2008}
Basu, S.; Antia, H. M. (2008). "Helioseismology and Solar Abundances". Physics Reports 457

\bibitem{Aristotle:2012}
Letter to the Editor: Sunspot observations by Theophrastus revisited

\bibitem{BBC:2006}
 "Sir Isaac Newton (1643–1727)". BBC. Retrieved 2006-03-22.

\bibitem{Darden:1998}
 Darden, L. (1998). "The Nature of Scientific Inquiry".

\end{thebibliography}

\end{document}