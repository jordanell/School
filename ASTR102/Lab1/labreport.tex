%%%%%%%%%%%%%%%%%%%%%%%%%%%%%%%%%%%%%%%%%
% Laboratory Report LaTeX Template
%
% This template has been downloaded from:
% http://www.latextemplates.com
%
%%%%%%%%%%%%%%%%%%%%%%%%%%%%%%%%%%%%%%%%%

%----------------------------------------------------------------------------------------
%	DOCUMENT CONFIGURATIONS
%----------------------------------------------------------------------------------------

\documentclass{article}

\usepackage{graphicx}

\title{Spectra of Gases and Solids \\ ASTR 102} % Title

\author{Jordan \textsc{Ell} \\ V00660306} % Author name

\begin{document}

\maketitle % Insert the title, author and date

\begin{tabular}{lr}
Date Performed: 1/14/2013\\ % Date the experiment was performed and partner's name
Instructor: Paolo turri % Instructor/supervisor
\end{tabular}

\setlength\parindent{0pt} % Removes all indentation from paragraphs

\renewcommand{\labelenumi}{\alph{enumi}.} % Make numbering in the enumerate environment by letter rather than number (e.g. section 6)

%----------------------------------------------------------------------------------------
%	SECTION 1
%----------------------------------------------------------------------------------------

\section{Objective}

The object of this lab and report are three fold. First is to observe the three
types of spectra as described by Kirchoff's Laws. Second is two observe emission 
spectra from a range of elements for contrast and comparison. Last is to determine
the wavelength of emission lines in hydrogen.
 
%----------------------------------------------------------------------------------------
%	SECTION 2
%----------------------------------------------------------------------------------------

\section{Introduction}

In astronomy, spectroscopy plays an important role is almost every sense of determining
what objects are made of and thus how they behave and act. Spectroscopy is the
study of the interaction between matter and radiated energy~\cite{Crouch:2007}.
Spectroscopy can be seen in use with Kirchoff's Laws often inside the field of
astronomy. Kirchoff's Law explain the different types of spectra that can emanate from
hot black bodies or types of gas. These types of spectra as: continuous, emission,
and absorption. With these 3 types of spectra, astronomers are able to determine
what black body objects are made of even at the great distances that astronomy is often
associated with.\\

This lab report will attempt to outline a very basic approach to observing effects 
of spectroscopy and Kirchoff's 3 laws. This report will outline procedures that
will enable the observation of continuous, emission, and absorption spectra. This
report will then take these procedures in order to attempt to identify (i) an unknown
element in gaseous form and (ii) the wavelengths of emission lines from hydrogen
gas.

%----------------------------------------------------------------------------------------
%	SECTION 3
%----------------------------------------------------------------------------------------

\section{Equipment}

For this report, the following list of equipment was used. A diffracting grating was used
to break light into spectrum. A light bulb with a solid tungsten filament was
used for giving off a continuous black body spectrum. A neon type light bulb was 
used to give off an emission spectrum. Multiple glass tubes filled with
various gases were used with electrical current to produce emission spectrum. 
The glass tubes are placed inside a wooden box with a thin slit to allow only
a small strip of light to emit. A web cam
was used to image emission spectrum and display it with video software on a computer.
For spectrum sketches, a single blank sheet of paper was used while for graphs
of hydrogen and helium wavelength vs. distance, a single piece of graph paper was used.
Finally, coloured pencils were used to sketch spectrum.

%----------------------------------------------------------------------------------------
%	SECTION 4
%----------------------------------------------------------------------------------------

\section{Procedure}

First, a continuous spectrum was observed and sketched. To do this, a solid tungsten 
filament is heated in a glass bulb in order to produce light. A diffracting grating was
then used to view the light. Looking through the grating produces a continuous 
spectrum of light with 1st to approximately 3rd order. The 1st order spectrum was 
sketched and can be seen in Figure 1.\\

% Continuous
\begin{figure}[h]
\centering
\includegraphics[width=0.7\textwidth]{images/BlankSpectrum}
\caption{Continuous spectrum of 1st order.\label{fig:cont}}
\end{figure}

Secondly, an emission spectrum was observed and sketched. To do this, a glass bulb 
containing a mix of elements was heated with electrical current. Once again, a 
diffracting grating was then used to view the light emitted from the bulb. Looking
through the grating produces an emission spectrum of light once again from 1st to
approximately 3rd order. The 1st order of the spectrum was then sketched and can be seen 
in Figure 2.\\

% Continuous
\begin{figure}[h]
\centering
\includegraphics[width=0.7\textwidth]{images/BlankSpectrum}
\caption{Emission spectrum of 1st order.\label{fig:cont}}
\end{figure}

Thirdly, an absorption spectrum was observed and sketched. To do this, a window was
used with blinds to only allow a small slit of light to enter the room. Once again,
a diffracting grating was then used to view the light entering the window. Looking
through the grating produces an absorption spectrum of light from 1st to approximately
3rd order. The 1st order of the spectrum was then sketched and can be seen in Figure 3.\\

% Continuous
\begin{figure}[h]
\centering
\includegraphics[width=0.7\textwidth]{images/BlankSpectrum}
\caption{Absorption spectrum of 1st order.\label{fig:cont}}
\end{figure}

The next steps of this report outline the observing and sketching of the following
element's emission spectrum: hydrogen, helium, mercury, and an unknown element. 
Using the wooden box filled with glass tubes containing various elements as outlined
in Section 3, we measure each element's spectrum and sketch them. To do this, the wooden
box is turned on which allows an electrical current to pass through a given element thus
heating it and emitting light. A diffracting grating was then used to view the light. 
Looking through the grating produces an emission spectrum for the selected 
element. The spectrum was then sketched and can be seen in Figures 1 - 7. This process 
was repeated for each of the elements listed above.\\

% Continuous
\begin{figure}[h!]
\centering
\includegraphics[width=0.7\textwidth]{images/BlankSpectrum}
\caption{Hydrogen emission spectrum.\label{fig:cont}}
\end{figure}

% Continuous
\begin{figure}[h!]
\centering
\includegraphics[width=0.7\textwidth]{images/BlankSpectrum}
\caption{Helium emission spectrum.\label{fig:cont}}
\end{figure}

% Continuous
\begin{figure}[h!]
\centering
\includegraphics[width=0.7\textwidth]{images/BlankSpectrum}
\caption{Mercury emission spectrum.\label{fig:cont}}
\end{figure}

% Continuous
\begin{figure}[h!]
\centering
\includegraphics[width=0.7\textwidth]{images/BlankSpectrum}
\caption{Unknown emission spectrum.\label{fig:cont}}
\end{figure}

Lastly, the wavelengths of emission lines from hydrogen were measured. This was 
accomplished in a two step procedure. First a base line was established using helium.
In order to get the wavelengths of helium, the following steps were taken. First, 
the aforementioned wooden box was once again used, however, this time the web cam was
used to take a picture of the emission spectrum of helium. To do this, the grating was
placed in front of the web cam before the picture was taken. Once the picture was taken,
a measurement of the 0 order emission to each of the emission lines of helium was taken.
These measurements were done in pixels and can be seen in Table 1. 
Once each spectral line was measured, a plot was made, in the back of this report as
Figure 8, to map distance of spectral line from the 0 order emission vs. the wavelengths
of emission lines in angstroms. The wavelengths of emission lines of helium were
provided in the lab by the instructor. This graph produces the baseline of the visible
part of the electromagnetic spectrum for emission lines distance and their wavelengths.
Next, the web cam, grating, and wooden box setup was once again used to obtain an
image of hydrogen emission lines. These measurements can be found in Table 2. These 
measurements were then projected onto the graph
made in the previous step in order to determine their wavelengths. This can be
seen in the back of the lab report labelled as Figure 8.\\

The determining of the unknown element in the wooden box as well as the 
reading of hydrogen 
emission line wavelengths were then carried out. These results can be seen in
Section 8.\\


%----------------------------------------------------------------------------------------
%	SECTION 5
%----------------------------------------------------------------------------------------

\section{Observations}

All observations for this report were made on the day of 2013-Jan-14. The only
calculation affected by time of day (observing of daylight absorption spectrum)
was made at approximately 16:30. The  original observation of all spectrum in this report
can be found in the back of this report. The measurements
of emission line distances of both helium and hydrogen can be found in Table 1
and Table 2 respectively. Table 2 also contains the observed wavelengths
of emission lines from hydrogen as described in Section 4. The plotting of helium
and hydrogen emission lines can be seen in the back of this lab report labelled as
Figure 8.


%----------------------------------------------------------------------------------------
%	SECTION 6
%----------------------------------------------------------------------------------------

\section{Tables and Measurements}
\label{sec:tnm}

The tables in this section correspond to measurements made on the wavelength vs. distance
graphs found in the back of this report labelled as Figure 8.

\begin{table}[h!]
\begin{center}
\begin{tabular}{l c}
\hline
Line Colour & Distance (px)\\
\hline
\hline
Pink/Purple & 223\\
Purple & 256\\
Blue & 272\\
Green & 290\\
Yellow & 338\\
\hline
\end{tabular}
\end{center}
\caption{Measurements of helium emission lines.\label{tab:helium}}
\end{table}

\begin{table}[h!]
\begin{center}
\begin{tabular}{c c c}
\hline
Wavelength (\AA) & Line Colour & Distance (px)\\
\hline
\hline
4850 & Purple & 278\\
5450 & Blue & 312\\
7075 & Red & 421\\
\hline
\end{tabular}
\end{center}
\caption{Measurements of hydrogen emission lines.\label{tab:helium}}
\end{table}

%----------------------------------------------------------------------------------------
%	SECTION 7
%----------------------------------------------------------------------------------------

\section{Graphs}

The graph of helium and hydrogen emission lines can be found at the back of this
report labelled as Figure 8.

%----------------------------------------------------------------------------------------
%	SECTION 8
%----------------------------------------------------------------------------------------

\section{Results}

This report consists of two results. The first result is that of the unknown element
inside the aforementioned wooden box setup. From the data presented, it is determined
that the unknown element is that of mercury. The unknown element's emission lines are
a match to the emission lines of mercury which can be seen in Figures 6 and 7. Objects 
which emit the same spectral lines are the same element.\\

The second result of this lab involves the determination of hydrogen emission line
wavelengths.
From previous knowledge, it is known that hydrogen emits 5 spectral lines in the visible
spectrum at 3970\AA, 4100\AA, 4340\AA, 4860\AA, and 6560\AA~\cite{Balmer:1885}. 
This report outlines the
measuring of only 3 spectral lines as seen in Table 2, which measured at wavelengths
4850\AA, 5450\AA, and 7075\AA. An obvious discrepancy is found here. \\

Problems and possible threats to validity of these results are discussed in Section
9 below.


%----------------------------------------------------------------------------------------
%	SECTION 9
%----------------------------------------------------------------------------------------

\section{Threats to Validity}

The sketching of all continuous, emission, and absorption spectrum was performed by hand.
Naturally this leads to a large amount of judgemental error in all sketches 
presented in Figure 1. Because of these free hand sketches, the matching of the unknown
element in the wooden box and mercury can not be exact. The match performed in this 
lab is an approximation made by only comparing the free hand sketches of each 
element's emission spectrum which could potentially lead to an inaccurate result
presented above.\\

The largest threat of validity to this report comes in the procedure of measuring
hydrogen's emission line wavelengths. For one comes the construction of the helium
baseline. The measuring of 0 order center to 1st order spectral lines was not exact
due to a poor quality web cam. The plot made of the helium emission line distances
was done by hand causing obvious errors. These errors are also repeated for the measuring
and plotting of hydrogen emission lines. The camera was also displaced between the two
measurements causing the largest region of error by far. The compilation of these errors
can be seen in the results presented in Section 8. Here we can see the 3 measured 
emission lines approximately match the last 3 known emission lines if 500\AA is 
subtracted from their value. This discrepancy probably comes mostly from the movement
of the web cam in between measurements.


%----------------------------------------------------------------------------------------
%	SECTION 10
%----------------------------------------------------------------------------------------

\section{Conclusions}

This report has shown a number of items. First it has shown how diffracting gratings
can be used in order to view continuous, absorption, and emission spectra. This
report has shown how previously known spectral lines can be used to identify what
elements are present in an unknown gas. This report has shown how to measure
the wavelength of an element against a baseline of a previously known element's
wavelengths. Finally, an attempt was made as measuring the wavelength of hydrogen
emission lines against a baseline of helium emission lines, however, this measurement
suffered from physical experimental procedures.\\

%----------------------------------------------------------------------------------------
%	SECTION 11
%----------------------------------------------------------------------------------------

\section{Evaluation}

Overall I enjoyed this lab, although at times it was quite slow. The ability to find
hydrogen's emission line wavelengths was very neat to see in process as well as
getting hands on experience with a spectroscopy setup was great. In the future it
would be great to be able to use the 38 inch telescope to perhaps perform some real
spectroscopy on the Sun or stars at night.\\

%----------------------------------------------------------------------------------------
%	BIBLIOGRAPHY
%----------------------------------------------------------------------------------------

\begin{thebibliography}{9}

\bibitem{Balmer:1885}
Balmer, J. J. (1885), "Notiz uber die Spectrallinien des Wasserstoffs", Annalen der 
Physik 261 (5): 80–87

\bibitem{Crouch:2007}
Crouch, Stanley; Skoog, Douglas A. (2007). Principles of instrumental analysis. Australia: 
Thomson Brooks/Cole.

\end{thebibliography}

\end{document}