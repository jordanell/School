%%%%%%%%%%%%%%%%%%%%%%%%%%%%%%%%%%%%%%%%%
% Laboratory Report LaTeX Template
%
% This template has been downloaded from:
% http://www.latextemplates.com
%
%%%%%%%%%%%%%%%%%%%%%%%%%%%%%%%%%%%%%%%%%

%----------------------------------------------------------------------------------------
%	DOCUMENT CONFIGURATIONS
%----------------------------------------------------------------------------------------

\documentclass{article}

\usepackage{amssymb, amsmath}
\usepackage{float}

\title{How Big is Our Galaxy \\ ASTR 102} % Title

\author{Jordan \textsc{Ell} \\ V00660306} % Author name

\begin{document}

\maketitle % Insert the title, author and date

\begin{tabular}{lr}
Date Performed: 1/1/2012\\ % Date the experiment was performed and partner's name
Instructor: Paolo Turri % Instructor/supervisor
\end{tabular}

\setlength\parindent{0pt} % Removes all indentation from paragraphs

\renewcommand{\labelenumi}{\alph{enumi}.} % Make numbering in the enumerate environment by letter rather than number (e.g. section 6)

%----------------------------------------------------------------------------------------
%	SECTION 1
%----------------------------------------------------------------------------------------

\section{Objective}

The objective of this lab report, is to determine the size of the Milky Way Galaxy which
we reside in. This object will be achieved through two mean, firs by determining the 
distance to the M15 globular cluster, and secondly by using a method similar to Shapley's
determining of the distance to the center of a globular cluster system.
 
%----------------------------------------------------------------------------------------
%	SECTION 2
%----------------------------------------------------------------------------------------

\section{Introduction}

We all live inside the Milky Way, the galaxy that contains our solar system. The name
Milky Way is derived from it's appearance as a milky glowing band across the night
sky on a clear and dark night~\cite{encyc:2012}. Not even 100 years ago, astronomers
and common folk believed that the Milky Way was the universe. Nothing was believed
to have existed outside of the Milky Way as these types of sizes were deemed 
impossibly big. Then along came a man named Harlow Shapley and his famous 1918
study of globular clusters, the galaxy, and our Sun's position inside of it
~\cite{Bart:1972}. \\

Using Cepheid variable stars~\cite{Udalski:1999}, Shapley was able to correctly
estimate the size of the Milky Way Galaxy and the Sun's position within it. From
these calculations are came the idea of deep space objects. Objects that lay
outside of the Milky Way. These objects eventually became to be seen as
other galaxies, black holes, and deep space supernovae~\cite{Fred:40}. \\

Although Shapley created ground breaking procedures and calculations, this
report will focus on merely replicating (with slight variations) some of
his measurements and calculations.

%----------------------------------------------------------------------------------------
%	SECTION 3
%----------------------------------------------------------------------------------------

\section{Equipment}

For this report, the following list of equipment was used. A set of 8 images of the globular
cluster M15 all taken on a single night. A single sheet of transparent paper with graph lines
pre-fabricated onto it. A single felt pen used for writing on the transparent paper.
A single sheet of opaque graph paper. A regular pencil for drawing on the opaque 
graph paper. A SHARP-EL510R calculator. The Linux computer program
known as skycat which was used for viewing images of globular clusters and measuring their
sizes.

%----------------------------------------------------------------------------------------
%	SECTION 4
%----------------------------------------------------------------------------------------

\section{Procedure}

\subsection{Determine the distance to M15}

First, a graph was draw to represent a variable star's magnitude in the globular cluster
known as M15. To do this, the 8 images of the cluster previously mentioned were used to
measure the magnitude of the star at each of the frames. This was done by comparing the
star's magnitude from a first frame to a known magnitude of a secondary star in the proceeding
frame. Some of the frames however did not have known magnitude stars. In these cases, either
the variable star was compared to known magnitudes in it's own frame or the next frame in
the series. Once the variable star's magnitude was measured for each frame, the graph
was ready to be constructed. The results of these measurements can be found in
Table~\ref{tab:m15}. The graph was constructed first on the transparent sheet of paper with
the felt pen. The graph is drawn as apparent magnitude vs. time in fractions of a day.
These fractions of a day correspond to the given image frames. Once the point were plotted,
an error bar of magnitude 0.5 was added to the graph to correct for possible measurement
errors. \\

Next the mean of the single variable star measured was calculated. This calculation
can be found in Equation~\ref{calc:m}. As other participants were taking part in the
experiment, the same procedure was conducted with other teams and other variable stars
in the globular cluster. These means can be found in Table~\ref{tab:mean}. Next, 
the mean of the variable stars mean magnitude was taken, as can be seen in 
Equation~\ref{calc:mm}. This true mean of the variable stars measure was then used
for a final calculation to determine the globular cluster's distance away from Earth
as seen in Equation~\ref{calc:d1}. A final step was now taken to transfer the 
graph previously drawn on the transparent paper to the opaque graph paper. This final
graph can be found in the back of this report labelled as Figure 1.

\subsection{Determine the distance to the galactic center}

To determine the distance to the center of the Milky Way Galaxy from Earth, the Linux
program skycat was first used. Skycat came pre loaded with images of various globular
clusters that are located in the field of view from Earth pointed towards the 
galactic center. Using skycat, each image was loaded to the screen. Then, each globular
cluster was measure in diameter using arc minutes. These measurements were done by 
measuring from one side of the "solid light" in the center of the cluster to the other
horizontally as the image dictates. The "solid light" for this report is defined as
where a single star can no longer be distinguished from the rest of its neighbouring 
stars. The results of the measurements can be found in Table~\ref{tab:size}. \\

Once the cluster were measure, they were inserted in to a web based application supplied
by the lab instructor.\footnote{orca.phys.uvic.ca/OLDWEBPAGES/a120/centgal/index.html}
This program preformed a type of weighting procedure based on the measurement of M15
and all other globular clusters measured. the result of this program's calculation
can be found in Equation~\ref{calc:d2}.

\subsection{Miscellaneous}

Lastly, questions proposed by the lab manual for Astronomy 102 at the University of
Victoria in the lab "How big is out galaxy" were answered using various calculation.
The questions and answers can be found in Section~\ref{sec:qna} while the calculations
themselves can be found in Section~\ref{sec:calc}.

%----------------------------------------------------------------------------------------
%	SECTION 5
%----------------------------------------------------------------------------------------

\section{Observations}

All observations for this report were made on the day of 2013-Mar-04. Since the result
of any given observation did not depend on time of day or weather, they have been omitted
from this report. The measurements of the M15 globular cluster and the brightness of its
variable stars can be found in Table~\ref{tab:m15}. The measurements of all variable stars
mean magnitude preformed by all groups inside the lab at the time of observation can 
be found in Table~\ref{tab:mean}. The measurements for the diameter of given globular
clusters preformed with the skycat program can be found in Table~\ref{tab:size} given
in arc minutes. Finally, the plotting of the variable star magnitude in the M15 globular
cluster can be found at the back of this lab report labelled as Figure 1.

%----------------------------------------------------------------------------------------
%	SECTION 6
%----------------------------------------------------------------------------------------

\section{Tables and Measurements}

The following two table correspond to lab supplied images and measurements taken
of that image of M15 the globular cluster. These images consisted of 8 frames taken
at variable times apart.

\begin{table}[H]
\begin{center}
\begin{tabular}{l c}
\hline
Star & Magnitude\\
\hline
\hline
B & 16.0\\
B & 15.8\\
B & 15.75\\
B & 16.4\\
B & 15.5\\
B & 15.3\\
B & 15.35\\
B & 15.6\\
\hline
\end{tabular}
\end{center}
\caption{Measurements B star magnitude in M15.\label{tab:m15}}
\end{table}

\begin{table}[H]
\begin{center}
\begin{tabular}{l c}
\hline
Star & Mean Magnitude\\
\hline
\hline
A & 16.2\\
B & 15.7\\
C & 15.7\\
D & 15.8\\
E & 15.8\\
F & 15.6\\
\hline
\end{tabular}
\end{center}
\caption{Measurements variable star means in M15.\label{tab:mean}}
\end{table}

\begin{table}[H]
\begin{center}
\begin{tabular}{l c}
\hline
Cluster & Diameter (arc minutes)\\
\hline
\hline
M15 & 3.20\\
NGC6624 & 0.30\\
NGC6626 & 0.82\\
NGC6637 & 0.57\\
NGC6638 & 0.33\\
NGC6652 & 0.25\\
NGC6656 & 1.13\\
NGC6681 & 0.37\\
NGC6715 & 0.58\\
NGC6723 & 0.47\\
NGC6809 & 2.40\\
\hline
\end{tabular}
\end{center}
\caption{Globular cluster diameters in skycat.\label{tab:size}}
\end{table}

%----------------------------------------------------------------------------------------
%	SECTION 7
%----------------------------------------------------------------------------------------

\section{Graphs}

The graphs of a given variable star's magnitude variance in globular cluster M15
can be found at the back of this lab report labelled as Figure 1.

%----------------------------------------------------------------------------------------
%	SECTION 7
%----------------------------------------------------------------------------------------

\section{Calculations}
\label{sec:calc}

The following calculation creates the mean magnitude value for the measured variable star in
the globular cluster M15 where i represents a row in Table~\ref{tab:m15}.

\begin{equation}
\label{calc:m}
mean = \frac{\Sigma m_{i}}{8} = 15.7125
\end{equation}

The following calculation creates the mean magnitude value for all measured variable star
means as calculated by other participants in the lab where i represent a row in
Table~\ref{tab:mean}

\begin{equation}
\label{calc:mm}
mean = \frac{\Sigma m_{i}}{6} = 15.8
\end{equation}

The following calculation gives the distance in parsec from Earth to the globular cluster
M15.

\begin{equation}
\label{calc:d1}
D = 10^{\frac{15.8 - 0.8 + 5}{5}} = 10,000 pc
\end{equation}

The following calculation gives the results of the orca web application used to determine
the distance from Earth to the galactic center

\begin{equation}
\label{calc:d2}
Orca_{d} = 25,700.24 pc
\end{equation}

The following calculations refer to the miscellaneous questions asked at the end
of this lab manual lab.

\begin{equation}
\label{calc:m1}
Light Time = 8,000 pc * 3.26 lry/pc
\end{equation}

\begin{equation}
\label{calc:m2}
Distance = 8,000pc * 2\pi
\end{equation}

\begin{equation}
\label{calc:m3}
Time = \frac{50,000pc}{.00021pc/yr}
\end{equation}

\begin{equation}
\label{calc:m4}
Mass = 8.8X10^{15}*\frac{A^3}{P^2}
\end{equation}

\begin{equation}
\label{calc:m5}
Count = \frac{78,000,000,000}{(3*10^7)}
\end{equation}

%----------------------------------------------------------------------------------------
%	SECTION 8
%----------------------------------------------------------------------------------------

\section{Results}

The results of this lab can be found by examining the calculations preformed in 
Section~\ref{sec:calc}. The results for the question of "how far away is the globular
cluster M15" are found through Equation~\ref{calc:d1} and show the result as being
10,000 parsecs.\\

The results for the question of "how far away is the galactic center from Earth"
are found in the results of Equation~\ref{calc:d2} and show the result as being
25,700.24 parsecs.

The results for any question that the lab manual (as previously mentioned) poses
can be found in Section~\ref{sec:qna} while the supporting calculations can be 
found in Section~\ref{sec:calc}.

%----------------------------------------------------------------------------------------
%	SECTION 8
%----------------------------------------------------------------------------------------

\section{Questions and Answers}
\label{sec:qna}

The following questions and answers are asked inside of lab 6, How Big Is Our Galaxy,
inside of the lab manual for ASTR102. The questions have been repeated for the reader.

\begin{enumerate}

\item[Q.] Calculate how long it takes for light to get to the sun from the center of
the galaxy. There are 3.26 light years in one parsec.
\item[A.] As per Equation~\ref{calc:m1} the answer is 26,000lyr. This corresponds to
the accepted answer of 27,000lyr +- 1,000lyr~\cite{Reid:1993}.
\item[Q.] Calculate the distance the Sun travel around the center of the galaxy.
\item[A.] As per Equation~\ref{calc:m2} the answer is 50,000pc.
\item[Q.] How long does it take the Sun to orbit the center of the galaxy?
\item[A.] As per Equation~\ref{calc:m3} the answer is 240,000,000yr. This corresponds to the
accepted answer of 240,000,000 years~\cite{Hess:2002}.
\item[Q.] Find the mass of the galaxy.
\item[A.] As per Equation~\ref{calc:m4} the answer is ~78,000,000,000. This corresponds
to the accepted answer of $1.0-1.5*10^{12}$~\cite{McMillan:2011}.
\item[Q.] If you are looking for life in our galaxy, and you spend 1 second looking
at each star, how many years would it take to check out our galaxy?
\item[A.] As per Equation~\ref{calc:m5} the answer is 2600yr.

\end{enumerate}


%----------------------------------------------------------------------------------------
%	SECTION 9
%----------------------------------------------------------------------------------------

\section{Threats to Validity}

There are three main threats to validity for this lab report. Two come at the hands
of self measurement, while the third comes at the hands of reliance of external
teams for measurement and calculation. For once, the measurements of the variable
star magnitude measured in M15 is a threat to validity. These measurements were preformed
purely with the naked eye and can be subject to much debate as the magnitudes of
stars were simply compared visually. Errors in this measurement can lead to errors
in the drawing of Figure 1 and the further calculations preformed on both means and
on the distance from Earth to M15. \\

The second self measurement threat to validity is the measuring of globular cluster
sizes on the program skycat. The definition of what constitutes the boundaries of a
globular cluster causes the basis of this error aside from the actual imperfection
of the measurement itself. The boundary used as described early was the "solid light"
boundary. However, some globular clusters measure lacked a definite "solid light"
boundary. This lead to potential measuring errors from one globular cluster to the
next. These measurements effect the error in the calculation of the distance to
the galactic center from Earth. This error leads to the rather large final calculation
error for this question. \\

The final threat to validity is the reliance of external teams for their own measurements
of variable stars in M15 and following calculations. The numbers passed to myself
were not validated by any other external party for validity. One external team
admitted to large measurement errors in a variable star magnitude. These types of errors
and possible error effect the calculation done in Equation~\ref{calc:d1}.



%----------------------------------------------------------------------------------------
%	SECTION 10
%----------------------------------------------------------------------------------------

\section{Conclusions}

This report has shown how to calculate two items. First, by measuring the magnitude of
variable stars, one can determine the distance of a globular cluster from Earth,
in this case M15. Second, this report has shown how diameter measurements of globular
clusters in a globular cluster system can be used to determine the distance from
Earth to the galactic center.

%----------------------------------------------------------------------------------------
%	SECTION 11
%----------------------------------------------------------------------------------------

\section{Evaluation}

I found this lab interesting purely for the demonstration and experiment involving
the magnitudes of variable stars inside a globular cluster to determine its distance.
I had heard before that variable stars can be used to determine the distances to
astronomical objects but had never known the procedure. It would have been interesting
if a larger explanation was given as to how this procedure works or how the formula
used in Equation~\ref{calc:d1} was derived.


%----------------------------------------------------------------------------------------
%	BIBLIOGRAPHY
%----------------------------------------------------------------------------------------

\begin{thebibliography}{9}

\bibitem{encyc:2012}
"Milky Way Galaxy". Encyclopædia Britannica, Inc.. Retrieved 2012-10-31.

\bibitem{Bart:1972}
Bart J. Bok. Harlow Shapely 1885-1972 A Biographical Memoir. National Academy of Sciences

\bibitem{Udalski:1999}
Udalski, A.; Soszynski, I.; Szymanski, M.; Kubiak, M.; Pietrzynski, G.; Wozniak, P.; Zebrun, K. (1999). "The Optical Gravitational Lensing Experiment. Cepheids in the Magellanic Clouds. IV. Catalog of Cepheids from the Large Magellanic Cloud". Acta Astronomica 49: 223.

\bibitem{Fred:40}
Fred Schaaf, 40 nights to knowing the sky: a night-by-night skywatching primer, page 113

\bibitem{Reid:1993}
Reid, Mark J. (1993). "The distance to the center of the Galaxy"

\bibitem{Hess:2002}
Hess, Frances. Earth Science. New York: Glencoe Mc Graw-Hill, 2002: 348.

\bibitem{McMillan:2011}
McMillan, P. J. (July 2011). "Mass models of the Milky Way". Monthly Notices of the Royal Astronomical Society 414 (3): 2446–2457

\end{thebibliography}

\end{document}