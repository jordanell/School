%%%%%%%%%%%%%%%%%%%%%%%%%%%%%%%%%%%%%%%%%
% Laboratory Report LaTeX Template
%
% This template has been downloaded from:
% http://www.latextemplates.com
%
%%%%%%%%%%%%%%%%%%%%%%%%%%%%%%%%%%%%%%%%%

%----------------------------------------------------------------------------------------
%	DOCUMENT CONFIGURATIONS
%----------------------------------------------------------------------------------------

\documentclass{article}

\usepackage{amssymb}% http://ctan.org/pkg/amssymb
\usepackage{pifont}% http://ctan.org/pkg/pifont
\newcommand{\cmark}{\ding{51}}%
\newcommand{\xmark}{\ding{55}}%

\title{Colour-Magnitude Diagrams \\ ASTR 102} % Title

\author{Jordan \textsc{Ell} \\ V00660306} % Author name

\begin{document}

\maketitle % Insert the title, author and date

\begin{tabular}{lr}
Date Performed: 25/03/2013\\ % Date the experiment was performed and partner's name
Instructor: Paolo Turri % Instructor/supervisor
\end{tabular}

\setlength\parindent{0pt} % Removes all indentation from paragraphs

\renewcommand{\labelenumi}{\alph{enumi}.} % Make numbering in the enumerate environment by letter rather than number (e.g. section 6)

%----------------------------------------------------------------------------------------
%	SECTION 1
%----------------------------------------------------------------------------------------

\section{Objective}

The objective of this lab report is to find the age and distance of both open
and globular clusters of stars within our Milky Way galaxy using colour-magnitude
diagrams inside the "isochrone" Linux program.
 
%----------------------------------------------------------------------------------------
%	SECTION 2
%----------------------------------------------------------------------------------------

\section{Introduction}

The Pillars of Creation are some of the most awe inspiring objects an observer can
gaze upon when they look up at the night sky. Here we see stars being formed, solar systems
flaring into existance, and pure beauty. However, what the untrained observer may not
notice is that stars can be seen here, and all over the universe inside of galaxies,
forming in clusters. There are two common types of clusters known as open and globular
cluster. Open clusters have groups of stars forming at one time, but the stars are not
gravitationally bound and can eventually leave the cluster, hence the term open. Globular
clusters on the other hand have so much mass in them that the stars become gravitationally
bound and make for very dense groupings of stars in the night sky.\\

Stars form, they live, and they eventually die. Their life sequence if oten plotted on the
Hertzsprung-Russell Diagram. This diagram has many variations such as the colour-magnitude
diagram which can be used to measure distances and other properties of large groups of stars.
This report will focus on exactly that task. This report will outline a procedure and
conclusions of how to determine the distance to both open and globular clusters using the
isochrone Linux program.\

%----------------------------------------------------------------------------------------
%	SECTION 3
%----------------------------------------------------------------------------------------

\section{Equipment}

For this report, the following list of equipment was used. The "isochrone" Linux program
which is a piece of software developed by James Clem. A SHARP-EL510R calculator.
A single sheet of opaque paper used to record a table of measurements which latter
became Table~\ref{tab:main}. Finally, a single pencil used to record measurements.

%----------------------------------------------------------------------------------------
%	SECTION 4
%----------------------------------------------------------------------------------------

\section{Procedure}

The procedure preformed in the report is borken down into the following two sections:
Measurements, and Calculations.

\subsection{Measurements}

For the following procedure, the Linux program "isochrone" was used. First from
isochrone, a single open or globular cluster was selected from a list of clusters
presented to the user. From here the user was presented with a colour-magnitude
diagram of the cluster's stars. Now, the user had esentially three options. The
first option was to make the cluster appear further or closer to the observer. The
second was to add or subtract dust between the observer and the cluster. Finally,
the user had the option to add or subtract age from the cluster. With these three 
options, a line of best fit appeared over top the the colour-magnitude diagram. 
This line of best fit was attempted to be centered appropriately on the main 
sequence, red giant branch, and horizontal branch where appropriate. Each of the
three options moved the line of best fit in a distinct manor and the user
move the line until it matched closely with the diagram of stars.\\

Once the line of best fit was appropriately places on the diagram, several 
measurements were taken. All of these measurements can be found in Table~\ref{tab:main}.
The measurements involving numerical values were all provided by the isochrone
program while the boolean mearuements were made using the best judgment of
the observer in terms of the presense of certain diagram features.\\

The steps just outline were preformed for each of the 11 clusters availible
to measure in the isochrone program.

\subsection{Calculations}

After all measurements had been made for each of the clusters in the isochrone program,
a simple calculation was then made to determine the distance to the cluster from
Earth. Equation~\ref{eq:dist} was used on each of the clusters and the results of
these calcuations can be found in Table~\ref{tab:main}.

%----------------------------------------------------------------------------------------
%	SECTION 5
%----------------------------------------------------------------------------------------

\section{Observations}

All observation for this report were made on the day of 2013-Mar-25. Since the result
of any given observation did not depende on time of day or weather, they have been omitted
from this report. All open and globular cluster measurements made in this report were
made using the "isochrone" Linux computer program and can be found in Table~\ref{tab:main}
These are the only observations made for this report.

%----------------------------------------------------------------------------------------
%	SECTION 6
%----------------------------------------------------------------------------------------

\section{Tables and Measurements}

The following table corresponds to the measurements taken using the Linux program
isochrone which is used to measure colour-magnitude characteristics of open and
globular clusters.

\begin{table}[h!]
\begin{center}
\tabcolsep=0.09cm
\begin{tabular}{l c c c c c c c c c c c}
\hline
Cluster & MS & RGB & HB & BS & Field Stars & $m_{sun}$ & m-M & D(pc) & E(B-V) & Age(Gyr) & [Fe/H]\\
\hline
\hline
Hyades & \checkmark & \xmark & \xmark & \xmark & \xmark & NA & 4.00 & 63.10 & 0.19 & 0.11 & 0.00 \\
M15 & \checkmark & \checkmark & \checkmark & \xmark & \xmark & 21.095 & 15.25 & 11220.18 & 0.09 & 16.00 & -2.14\\
NGC104 & \checkmark & \checkmark & \checkmark & \checkmark & \checkmark & 18.304 & 13.15 & 4265.80 & 0.02 & 14.00 & -0.71\\
NGC6791 & \checkmark & \checkmark & \xmark & \checkmark & \checkmark & 18.278 & 13.15 & 265.80 & 0.10 & 11.22 & 0.20\\
h+x Persei & \checkmark & \xmark & \xmark & \xmark & \checkmark & 16.043 & 11.35 & 1862.09 & 0.00 & ZAMS & 0.00\\
NGC6612 & \checkmark & \xmark & \xmark & \xmark & \checkmark & NA & 10.15 & 1071.52 & 0.00 & ZAMS & 0.00\\
M67 & \checkmark & \checkmark & \xmark & \checkmark & \xmark & 14.146 & 9.40 & 758.58 & 0.02 & 4.47 & -0.04\\
NGC188 & \checkmark & \checkmark & \xmark & \xmark & \checkmark & 15.331 & 11.40 & 1905.45 & 0.11 & 6.31 & 0.04\\
Pleiades & \checkmark & \xmark & \xmark & \xmark & \xmark & 10.259 & 5.90 & 151.36 & 0.09 & ZAMS & 0.00\\
Praesepe & \checkmark & \xmark & \xmark & \xmark & \xmark & 10.634 & 7.90 & 380.19 & 0.36 & 0.06 & 0.00\\
Hipparcos & \checkmark & \checkmark & \xmark & \checkmark & \checkmark & 4.614 & -0.15 & 9.33 & 0.00 & 2.24 & 0.00\\
\hline
\end{tabular}
\end{center}
\caption{Open and globular cluster measurements in isochrone.\label{tab:main}}
\end{table}


%----------------------------------------------------------------------------------------
%	SECTION 7
%----------------------------------------------------------------------------------------

\section{Calculations}

The following equation was used to calculate all values in Table~\ref{tab:main} under
column D(pc). This equation gives the distance in parsecs of how far away an open or
globular cluster is from Earth. Here the m-M comes also from Table~\ref{tab:main}.

\begin{equation}
\label{eq:dist}
D = 10^{\frac{m-M+5}{5}}
\end{equation}



%----------------------------------------------------------------------------------------
%	SECTION 8
%----------------------------------------------------------------------------------------

\section{Results}

The results of this lab report come in the form of the distances obtained by using
Equation~\ref{eq:dist}. These results for the distance to each globular and open cluster
can be found in Table~\ref{tab:main} under the D(pc) column. The results obtained were
largely within an acceptable error rate, with the two minor exceptions of
M15 and NGC104 as they were determined to be older than the universe. The second set
of results are those discussed in Section~\ref{sec:qna} which are answers to various
question found in the back of the Colour-Magnitude Diagrams lab in the ASTR102 lab
manual.



%----------------------------------------------------------------------------------------
%	SECTION 9
%----------------------------------------------------------------------------------------

\section{Conclusions}

The only conclusions to draw from this lab report are that of a successful determination
of the distance to various open and globular clusters found in Table~\ref{tab:main}. The
pseudo conclusions to questions found in the ASTR102 lab manual can be found in
Section~\ref{sec:qna}.

%----------------------------------------------------------------------------------------
%	SECTION 9
%----------------------------------------------------------------------------------------

\section{Threats to Validity}

The main threat to validity during this lab process was the measuring of the colour-
magnitude diagrams found in the Linux program isochrone. Here, tools from the program
were used such as add more dust between the observer and the cluster, or make the cluster
appear futher away to get a line of best fit with the main sequence. This line of best 
fit is the best approximation made by human eyes and was not a rigorous algorithm to
find the perfect line of best fit.\\

This threat can cause the disturbance of the following measurements: $m_{sun}$, m-M,
E(B-V), Age(Gyr), and [FE/H] all found in Table~\ref{tab:main}. Because these measurements
may have been skewed by the experimenters version of best fit, the calculation used
to calculate the distance (the overall objective of this lab) may have also been skewed.
This threat to validity would explain the two occurances in the results where the cluster's
age appeared to be older than the universe.\\

A secondary, best less extreme threat is that of the isochrone program. The program
gave various measurements in Table~\ref{tab:main} which were not tested by the observer.
If these numbers are inaccurate they could lead to serious issues in future calculations
and conclusions.

%----------------------------------------------------------------------------------------
%	SECTION 10
%----------------------------------------------------------------------------------------

\section{Question and Answers}
\label{sec:qna}

The following question and answers are asked inside lab 5, Colour-Magnitude Diagrams
inside of the lab manual for ASTR102. The question have been repeated for the reader.

\begin{enumerate}

\item[Q.] How does the isochrone change when you increase the distance? Why?
\item[A.] The isocrone moves down and slightly to the left the further the cluster is
and opposite when the cluster is nearer. The further and object is away from the observer
the fainter it will appear. This is seen by the isochrone shifting towards the more faint
end of the graph.
\item[Q.] How does the isochrone change when you increase the dust? Why?
\item[A.] The isochrone move right and slightly up when more dust is added and the 
opposite when dust is removed. This occurs because the more dust that is added the
the redder the object becomes and thus its colour gets shifted towards the red.
\item[Q.] How does the isochrone change when you increase the age? Why?
\item[A.] The older the cluster becomes the lower the turnoff point begins to come
down the main sequence and the RGB begins to form with the HB. This can be explained
easily as the older starts become in a cluster, the more red giants you will over.
If your cluster has more red giants, its life cycle must be further along because
of how long it takes to form red giants.
\item[Q.] Do the cluster with age=0 years have a read giant branch? Why or why not?
\item[A.] The answer to the previous question gives us some insight into this question.
Clusters with an age of 0 will not have a RGB because it takes stars billions of
years to turn of the main sequence and rise up into the RGB and form red giants.
Again, having a RGB is a sign of the cluster's age.
\item[Q.] Are any of the clusters older than the Earth, or the universe? Comment.
\item[A.] There are many clusters older than the Earth found in Table~\ref{tab:main}.
These clusters do not popse a problem as stars formed before and after our own 
star was formed. There are two clusters which appear to be older than the universe
(M15 and NGC104). The causing of this was touched on in the threats to validity
but to reiterate, these issues are causes by errors of best fitting the isochrone
line to the main sequence of the cluster in the isochrone Linux program.
\item[Q.] Compare M15 and NGC104 to the rest of the clusters in terms of the [Fe/H] levels. 
\item[A.] These two clusters appear to have extremely less iron in their stars than
the other clusters studied. This also coincides with their age as they are much older.
While stars were forming in the early universe, the amount of metal in the ISM
was very low due to the fact that supernovae and other metal generating processes
had not occured yet. This is why very old stars have little metal. Comapred to newer
stars which have more iron present because they formed after a time where metals
were already present in the universe.

\end{enumerate}



%----------------------------------------------------------------------------------------
%	SECTION 11
%----------------------------------------------------------------------------------------

\section{Evalutation}

As the last lab for ASTR102, I enjoyed this quick lab as a type of recap to some
of the earlier ideas expressed in the astronomy course. It was interesting to learn
that the age of a cluster can be found by studying its isochrone and with facts such
as the existance of a red giant branch. It was also extremely interesting to learn
how some clusters in our galaxy, formed near the beginning of the universe. It would
have been nice to know when our Milky Way formed and how the older clusters studied
in this lab played a role in either forming or combining with the Milky Way at some
point in its life.

\end{document}