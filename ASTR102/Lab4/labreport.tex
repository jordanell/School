%%%%%%%%%%%%%%%%%%%%%%%%%%%%%%%%%%%%%%%%%
% Laboratory Report LaTeX Template
%
% This template has been downloaded from:
% http://www.latextemplates.com
%
%%%%%%%%%%%%%%%%%%%%%%%%%%%%%%%%%%%%%%%%%

%----------------------------------------------------------------------------------------
%	DOCUMENT CONFIGURATIONS
%----------------------------------------------------------------------------------------

\documentclass{article}

\usepackage{amssymb, amsmath}
\usepackage{float}

\title{What is the Age and Size of the Universe \\ ASTR 102} % Title

\author{Jordan \textsc{Ell} \\ V00660306} % Author name

\begin{document}

\maketitle % Insert the title, author and date

\begin{tabular}{lr}
Date Performed: 18/03/2013\\ % Date the experiment was performed and partner's name
Instructor: Paolo Turri % Instructor/supervisor
\end{tabular}

\setlength\parindent{0pt} % Removes all indentation from paragraphs

\renewcommand{\labelenumi}{\alph{enumi}.} % Make numbering in the enumerate environment by letter rather than number (e.g. section 6)

%----------------------------------------------------------------------------------------
%	SECTION 1
%----------------------------------------------------------------------------------------

\section{Objective}

The objective of this lab report is two answer two questions: What is the age of the 
universe, and what is the size of the universe. Through these questions, I will also
demostrate how the Hubble Law can be calculated by observing distance galaxies and
how it can be used to calculate the size and age of the universe.
 
%----------------------------------------------------------------------------------------
%	SECTION 2
%----------------------------------------------------------------------------------------

\section{Introduction}

We all live in a house. That house is in a city, in a province, in a country, in a 
continent, on Earth. Earth is in a solar system, in a galaxy, in a cluster of galaxies,
in the Universe. Just for the scope of things. But how big and how old is that scope?
Back in 1905, Einstein's special relativity~\cite{alb:1905} hit the scene showing that
Light is the fastest moving matter and will always move at the same speed. This insight
will eventually lead us down the path to how big the universe is. Einstein also
came up with his infamous cosmological constant~\cite{urry:2008}
which allowed the universe to stay
the same size. He would later state this was the biggest mistake of his career.\\

Later, Edwin Hubble came along and discovered the Hubble Constant~\cite{lem:1927}
, a remnant of
Einstein's cosmological constant (he was sort of right), a figure that shows
the relationship to how distant two obejcts are and their velocities. Through this
constant, and Einstein's special relativity,
Hubble was able to make general predictions of how big the observable
universe was and how old it was leading up to the big bang.\\

While we generally take for granted the results of these experiements and calculations
of universe size and age, very few people know how it is calculated. This report will
demostrate how using spectroscopy on distant galaxies, we can find the hubble constant
and thus determine the size of the universe and its age.

%----------------------------------------------------------------------------------------
%	SECTION 3
%----------------------------------------------------------------------------------------

\section{Equipment}

For this report, the following list of equipment was used. The "Contemporary Laboratory
Experience in Astronomy" which is a Windows program developed by Larry Marshall's group
at the Department of Physics at Gettysburg College. A SHARP-EL510R calculator. A single
sheet of opaque graph paper. A blue pen for drawing on the graph paper. Finally, Google
Calculator~\footnote{http://www.google.ca/help/features.html} was also used.

%----------------------------------------------------------------------------------------
%	SECTION 4
%----------------------------------------------------------------------------------------

\section{Procedure}

The procedure preformed in this report is broken down into the following two sections:
Measruements, and Calculations

\subsection{Measruements}

For the following procedure, the Windows program "Contemporary Laboratory Experience in
Astronomy" (CLEiA) was used. First, CLEiA was run from a Windows personal computer.
From here, a number of steps in the CLEiA simulation had to be preformed for the use
inside this lab. First the virtual dome was opened by clicking the Dome button. Next, 
star tracking was turned on by clicking the Tracking button. Finally, the slew rate
was set to 16 to allow faster movements of the telescope by clicking the Slew Rate
button twice. The next procedures were preformed for 3 galaxies inside each of the
given fields of the CLEiA program.\\

The chosen galaxy was centered in the telescope's field of view. Next, the Change View
button was clicked to get a magnified view of what the telescope's field of view was 
pointed at. From here, the galaxy was once again centered on the cross hair. This time
however, the cross hair were two parallel lines. These lines were placed over the
middle of the galaxy where the best measruement could be taken. Next, the Take Reading
button was placed which opened up a new window that had a spectroscopy graph on it.
From here, the Start Count button was pushed. This button started the spectroscopy
readings of the galaxy and their K and H calcium absorbption lines. For most galaxies,
the readings continued until the Signal/Noise was at least 20.0 (some were stopped at
10.0 for time). After the Signal/Noise level was appropriate, the Stop Count button was
pushed to stop the reading. Now, the object's name, the apparent magnitude and K
calcium absorption lines were read and placed in Table~\ref{tab:main}. To measure the
K calcium line, the cursor was used to click on the lowest point of the line and
then measure the wavelength that was given. These steps were repeated for 3 galaxies
in each of the 6 fields given by the CLEiA program.\\

\subsection{Calculations}

Once all measurements had been taken, a series of calculation was preformed. For
each row inside of Table~\ref{tab:main}, the following calculation were preformed.\\

First, the change in K calcium wavelength was computed using Equation~\ref{eq:delta}.
Nest, the velocity of the object was calculated using Equation~\ref{eq:vel}. And finally,
the distance of the object from Earth was calculated using Equation~\ref{eq:dist}.\\

After each of the latter equations were used on each row in Table~\ref{tab:main},
Three final calculations were used to calculate the Hubble Constant
(Equation~\ref{eq:slope}, the slope of Figure 1), 
the size of the observable universe (Equation~\ref{eq:size}),
and the age of the universe (Equation~\ref{eq:age}).

%----------------------------------------------------------------------------------------
%	SECTION 5
%----------------------------------------------------------------------------------------

\section{Observations}

All observations for thie report were made on the day of 2013-Mar-18. Since the result
of any given observation did not depend on time of day or weather, they have been
omitted from this report. All spectroscopy measurements made in this report were
made using the "Contemporary Laboratory Experience in Astronomy" (CLEiA) and can be
found in Table~\ref{tab:main}. These are the only observations made for this report.

%----------------------------------------------------------------------------------------
%	SECTION 6
%----------------------------------------------------------------------------------------

\section{Tables and Measurements}

The following table corrsponds to the spectroscopy measurements and various other
details of distance galaxies using CLEiA. The later columns correspond to
calculations made using Equation~\ref{eq:delta}, Equation~\ref{eq:vel}, and
Equation~\ref{eq:dist}.

\begin{table}[H]
\begin{center}
\begin{tabular}{l c c c c c}
\hline
Galaxy & magnitude & lambda & delta & v(km/s) & D(Mpc)\\
\hline
\hline
Coma1 & 12.30 & 4012.0 & 78.0 & 5944 & 72.4\\
EDW & 13.45 & 4050.0 & 116.0 & 8839 & 123.0\\
LAM & 11.15 & 3973.0 & 39.0 & 2972 & 42.7\\
PRC & 13.36 & 4051.0 & 117.0 & 8916 & 118.0\\
CrBor2 & 15.43 & 4209.0 & 275.0 & 20956 & 306.2\\
CrBor1 & 15.08 & 4208.0 & 274.0 & 20880 & 260.6\\
CrBor3 & 15.35 & 4208.0 & 274.0 & 20880 & 295.1\\
Boot2 & 16.76 & 4446.0 & 512.0 & 39017 & 564.9\\
Boot3 & 16.72 & 4445.0 & 511.0 & 38941 & 554.6\\
Boot1 & 16.52 & 4444.0 & 510.0 & 38864 & 505.8\\
Coma3 & 12.45 & 4012.0 & 78.0 & 5944 & 77.6\\
Coma2 & 12.55 & 4012.0 & 78.0 & 5944 & 81.3\\
uma1-3 & 14.49 & 4130.0 & 196.0 & 14936 & 198.6\\
uma1-1 & 14.62 & 4130.0 & 196.0 & 14936 & 210.9\\
uma1-2 & 14.52 & 4130.0 & 196.0 & 14936 & 201.4\\
uma2-1 & 16.87 & 4484.0 & 550.0 & 41913 & 594.3\\
uma2-3 & 16.89 & 4484.0 & 550.0 & 41913 & 599.8\\
uma2-2 & 16.67 & 4484.0 & 550.0 & 41913 & 542.0\\
\hline
\end{tabular}
\end{center}
\caption{Spectroscopy measurements of galaxies in CLEiA.\label{tab:main}}
\end{table}

%----------------------------------------------------------------------------------------
%	SECTION 7
%----------------------------------------------------------------------------------------

\section{Graphs}

The graph of meassured galaxy Speed(km/s) vs. Distance(Mpc) can be found at the back
of this lab report labelled as Figure 1.

%----------------------------------------------------------------------------------------
%	SECTION 7
%----------------------------------------------------------------------------------------

\section{Calculations}
\label{sec:calc}

The following equation was used to calculate all values in Table~\ref{tab:main} under
column delta. This equation gives the difference between an at rest body's K calcium
absorbtion line wavelength and a moving body's K calcium absorbtion line wavelength.

\begin{equation}
\label{eq:delta}
\Delta \lambda_{K} = \lambda' - \lambda
\end{equation}

The following equation was used to calculate all values in Table~\ref{tab:main} under
column velocity. This equation gives the speed at which a galaxy is moving away
from the Earth at.

\begin{equation}
\label{eq:vel}
v = c * \frac{\Delta \lambda}{\lambda}
\end{equation}

The following equation was used to calculate all values in Table~\ref{tab:main} under
column distance(Mpc). This equation gives the distance in megaparsecs of how far away
a galaxy is from Earth. Here we assume the absolute magnitude, M = -22.0.

\begin{equation}
\label{eq:dist}
D = 10^{\frac{m-M-25}{5}}
\end{equation}

The following equation was used to calculate the slope of the graph in Figure 1.

\begin{equation}
\label{eq:slope}
H = \frac{(18000 - 6000)}{(240 - 80)} = 75km/s
\end{equation}

The following equation was used to calculate the size of the observable universe.

\begin{equation}
\label{eq:size}
size = \frac{c}{H} = 3997Mpc = 13,000,000,000 lyr.
\end{equation}

The following equation was used to calculate the age of the universe.

\begin{equation}
\label{eq:age}
age = \frac{1000}{H} = 13,300,000,000 years
\end{equation}

%----------------------------------------------------------------------------------------
%	SECTION 8
%----------------------------------------------------------------------------------------

\section{Results}

The results of this report can be found by examining the results of 
Equation~\ref{eq:size} and Equation~\ref{eq:age}. Here we find that the size
of the observable universe was found to be 13 Billion ligh years in radius
from the Earth. The age of the universe was found to be 13.3 billion years old.\\

These results were found by using the intermediate result of the found Hubble 
Constant which was found to be 75km/s/Mpc. The accepted results for these two
primary questions and one intermediate step are as follows. The accepted size of
the observable universe 14.3 billion parsecs ~\cite{gott:2005}. 
The accepted age of the universe 13.77s billion years old~\cite{lars:2012}. 
The accepted value
of the Hubble constant is 69.32km/s/Mpc~\cite{WMAP:2012}.

%----------------------------------------------------------------------------------------
%	SECTION 9
%----------------------------------------------------------------------------------------

\section{Threats to Validity}

The main threat to validity in this lab report is the reliance on the CLEiA Windows
program. Here all values into the program have been somewhat precomputed and are therefore
subject to any errors in the initial collection of data by the creators of CLEiA and the
astronomers who supplied the data. The other issue with CLEiA that causes a threat to
validity is the measurements preformed on the spectroscopy printout of each galaxy
as seen in Table~\ref{tab:main}. Each of the K calcium absorption lines were done
by attempting to read the perfect center location of the line. These measurements
are subject to error as the exact center of the line may not have been selected.\\

A secondary thrate to validity is the calculation of the Speed vs. Distance graph
found in the back of this lab report labelled as Figure 1. Here the line used to create
the graph was done by hand and was a best fit line. The slope therefore does not fully
represent the true slope of the points plotted on the graph. This could have been 
preformed better with graphing software.



%----------------------------------------------------------------------------------------
%	SECTION 10
%----------------------------------------------------------------------------------------

\section{Conclusions}

This report has shown how to calculate three items using the CLEiA Windows program.
First, the Hubble Constant was calculated by determining the speed and distance of
galaxies. Next, the hubble Constant was used in order to determine both the size of
the observable universe as well as the age of the universe. In this report, reasonable
answers were found that closely related to the true accepted values for the main
questions posed in this report.

%----------------------------------------------------------------------------------------
%	SECTION 11
%----------------------------------------------------------------------------------------

\section{Evaluation}

I found this lab extermely interesting for the demostration of the universe size
and age. I have always wondered how these ages were precisely calculated and how
the Hubble Constant came about. The Hubble Constant seems interesting to me as
it looks like it closely relates to both inflation and Einstein's infamous 
cosmological constant. I had lots of questions after this lab was over however 
but would take too long to ask. If space can expand faster than light can move
then how does this not violate special relativity~\cite{alb:1905}. 
Among others. I would have also
liked to see an explination of the difference between the observable universe
and the entire universe, or some explination of the possible shape of the 
entire universe.


%----------------------------------------------------------------------------------------
%	BIBLIOGRAPHY
%----------------------------------------------------------------------------------------

\begin{thebibliography}{9}

\bibitem{alb:1905}
Albert Einstein (1905) "Zur Elektrodynamik bewegter Körper", Annalen der Physik 17: 891

\bibitem{gott:2005}
Gott III, J. Richard; Mario Jurić, David Schlegel, Fiona Hoyle, Michael Vogeley, Max Tegmark, Neta Bahcall, Jon Brinkmann (2005). "A Map of the Universe"

\bibitem{lars:2012}
Bennett, C.L.; Larson, L.; Weiland, J.L.; Jarosk, N.; Hinshaw, N.; Odegard, N.; Smith, K.M.; Hill, R.S. et al. (December 20, 2012). Nine-Year Wilkinson Microwave Anisotropy Probe (WMAP) Observations: Final Maps and Results

\bibitem{WMAP:2012}
http://map.gsfc.nasa.gov/ (WMAP website)

\bibitem{urry:2008}
Urry, Meg (2008). The Mysteries of Dark Energy. Yale Science. Yale University.

\bibitem{lem:1927}
Lemaître, Georges (1927). "Un univers homogène de masse constante et de rayon croissant rendant compte de la vitesse radiale des nébuleuses extra-galactiques". Annales de la Société Scientifique de Bruxelles A47: 49–56

\end{thebibliography}

\end{document}