%%%%%%%%%%%%%%%%%%%%%%%%%%%%%%%%%%%%%%%%%
% Laboratory Report LaTeX Template
%
% This template has been downloaded from:
% http://www.latextemplates.com
%
%%%%%%%%%%%%%%%%%%%%%%%%%%%%%%%%%%%%%%%%%

%----------------------------------------------------------------------------------------
%	DOCUMENT CONFIGURATIONS
%----------------------------------------------------------------------------------------

\documentclass{article}

\title{Stars, Clusters Nebulae and Galaxies \\ ASTR 102} % Title

\author{Jordan \textsc{Ell}} % Author name

\begin{document}

\maketitle % Insert the title, author and date

\begin{tabular}{lr}
Date Performed: 28/01/2013\\ % Date the experiment was performed and partner's name
Instructor: Paolo Turri % Instructor/supervisor
\end{tabular}

\setlength\parindent{0pt} % Removes all indentation from paragraphs

\renewcommand{\labelenumi}{\alph{enumi}.} % Make numbering in the enumerate environment by letter rather than number (e.g. section 6)

%----------------------------------------------------------------------------------------
%	SECTION 1
%----------------------------------------------------------------------------------------

\section{Objective}
The objective of this lab report is not to outline a scientific process, but rather to inform
of the information learned during this lab section using the tool Google Earth. I will describe
different types of objects which were viewed and what was learned about them.

 
%----------------------------------------------------------------------------------------
%	SECTION 2
%----------------------------------------------------------------------------------------

\section{Astronomical Objects}
All astronomical objects in this report were viewed using Google Earth~\cite{Google:2013}. 
Google Earth has an
astronomy feature that allows you user to view the sky and images taken in a number of
astronomical surveys (including one done by the famous Edwin Hubble). Here, I will group 
objects by there type (even though specific objects were looked at) and talk about what was
learned in general about the types as well as any specific instances which have unique 
properties.

\subsection{Stars}
Stars, when viewed from the Earth's surface or from Google Earth, appear to have some size
dimensions associated with them. However, this is untrue and is actually caused by air turbulence
in the Earth's atmosphere. This condition is known as the "seeing", and since Google Earth's
images were taken from Earth's surface, the condition is present here as well. Stars also appear
to be different colours in the sky. This is due to their temperature, or wavelength of peak emission
emitted from the star~\cite{Walker:2008}. 
When looking at Cygni 30 and 31, 30 appears blue which is associated with
a shorter wavelength meaning higher energy. Higher energy in a star means higher temperature.
These stars also have the property of appearing close together in the sky. This is however not
true as the stars are actually many many light years apart and only appear close together due to
the angle of view. When viewing stars in Google Earth, there can also be slight artefacts
caused by the camera which are known as ghosts. This was seen viewing Vega.\\

Google Earth was also used to find a lower bound and upper bound on the number of stars in
our Milky Way Galaxy. If you take the least dense region of stars, count the stars, and multiply it by the 
relative size of the galaxy you find the lower bound for the number of stars in the Milky Way.
Similarly, if you take the densest region, the centre, and count stars and multiply, you
get an upper bound. It is thought that there are between 100 and 400 billion stars in the Milky
Way.

\subsection{Clusters}
Two types of clusters were examined in this lab: open, and globular. Open clusters are large
groups of stars in the sky, however there is usually space between the stars and not a 
nuclei of the cluster. Globular clusters on the other hand tend to have a nuclei which 
the concentration of stars is at a peak. Globular clusters appear to be more concentrated
and have less visible space between the stars. You can determine the general age of a cluster
based on the average colour of stars inside it. If the stars are blue then the cluster is
younger while if the stars are red then the cluster is older. There are about 150 known
globular clusters in the Milky Way~\cite{Harris:2003}

\subsection{Nebulae}
Three types of nebulae were examined: reflection, emission, and dark. These three types of
nebulae are all present in the North American nebula which has a shape similar to that of
North America. Reflection nebulas occur due to a scattering of light particles from the
emitting stars which are present. The other wavelengths of light are less prone to
scattering so they pass through the gas. This is the same reason why our sky appears
blue during the day (the scattering of the Sun's light). Reflection nebulas are mainly
composed of HI (temperature not hot enough to ionize the hydrogen). 
The emission nebula has a red colour to it and is composed of HII (ionized hydrogen).
The dark nebula is composed of $H_{2}$ and is the most dense of the nebulae. Dark nebulas
are where stars are formed. Here the condensed gas begins to compress in on itself and
form the starting of a brown dwarf. The starting the evolutionary process of a star is
know as the nebular theory~\cite{Abruzzo:2009}. 
The dark nebulas are so dense that they absorb any star
light that is behind them, making them appear as black holes in the sky.

\subsection{Galaxies}
Four types of galaxies were examined using Google Earth. These type of galaxies are:
elliptical, spiral, barred, and peculiar galaxies. It is worth noting that these
galaxies are so far away that no particular star can be seen inside of them. Any star
seen in images or on Google Earth which appears to be inside the galaxy is actually in
front of the galaxy in space. Elliptical galaxies when viewed in images have a more
round and elliptical shape while appearing almost all red and being similar to that
of a globular cluster. These galaxies are the oldest of the known galaxies. M67 was
examined and can be seen to be ejecting some blue material (in the visible) from 
its centre. This turns out to be a black hole ejecting gas primarily in the X-ray
part of the spectrum. I will come back to black holes and quasars later on in this report.
Spiral galaxies are those like our own Milky Way. They tend to be blue in colour and thus
younger with spiral arms coming from their centre. These are are filled with nebulas,
especially dark nebulas which are used in the forming of stars. These galaxies
also tend to have a slight bulge in their centre. Barred galaxies are spiral galaxies
with an apparent "bar" running through their centre and extending out on either side of
the galaxy. Galaxies such as the "Mice" galaxies which do not fit into the three types
of galaxies already listed are labelled as peculiar galaxies. Here, I looked at the "Mice"
galaxy which is actually two galaxies colliding and ejecting stars from one another. 
An interesting note about galaxy collisions is that stars are so relatively small compared
to the galaxies that start on star collisions are estimated to only occur once between
the billions of stars in each galaxy in a collision. This is good new considering Andromeda
is on a collision course with the Milky Way. Although we will probably be ejected out into 
space and never to return. Luckily this will be in about 4.5 billion years~\cite{Col:2007}.\\

The final item to note in this report about galaxies was the viewing of the Hubble Ultra-
Deep Field image. Here, the Hubble Space Telescope was pointed at an area of sky which
looked as if it had little activity and took a many month exposure image. The resulting
image returned a view littered with galaxies. From a view of 20 arc seconds, I counted 66
galaxies. which means the image contains 66 billion galaxies! These galaxies were formed
around 13 billion years ago which is right near the formation of the universe which
is 14 billion years old. These galaxies would have formed after the big bang, inflation,
the dark ages, and then stars forming. It is interesting to note that when we see these galaxies,
we are actually looking back in time 13 billion years as it took the light 13 billion
years to reach our cameras.

\subsection{Miscellaneous}
Here I will discuss some of the miscellaneous objects viewed using Google Earth as they
do not fit in with any previous category.

\subsubsection{Planetary Nebula}
At the end of a star's life, a couple things can happen which depend on its mass. One of those
is to become a black hole (will come back to), but more importantly one is cause a planetary
nebula. A planetary nebula is caused by a star ejecting its atmosphere at the end of its life.
This causes a large ring to form around the star. The ring is coloured from inside to out 
of short wavelengths to long because the star is still radiating in its new state of a white
dwarf. Planetary nebula's were given their name because early astronomers believed that what
they were seeing was similar to the look of planets~\cite{Plan:1997}. 
It is now know however that planets are not involved.

\subsubsection{Asteroid Track}
Asteroid are commonly discovered, and here in Google Earth, we can see how imaging can be
used to find asteroids by accident. If an image is taken of the stars, with proper star
tracking equipment, no streaks will be visible as the stars move due to the rotation of
the Earth. However, asteroid move independently of the stars. Therefore, when a large streak 
is seen in an image, it usually mean that it is an asteroid in the field of view of the
camera. Google Earth shows a red, green, and blue streak all corresponding to the same
asteroid. Asteroids are commonly found accidentally using this technique.

\subsubsection{Supernova Remnant}
Here, I looked at what is know as the Crab nebula. This nebula (also know as M1) was formed
by an exploding star in its last stages of life. Charles Messier discovered this nebula which
he recorded in his astronomy catalogue while looking for comets. (The objects he found however
were not moving and thus could not be comets.) At the centre of this nebula is what is known
as a pulsar or a neutron star. This star (composed of neutrons) spins at a very rapid and
constant rate which produces pulses of radiation from gamma to radio waves. Some of the first
SETI astronomers mistook pulsars for alien life as the radio waves seemed to be emitted with
intelligent thought. The neutron star is so dense that it is believed that a single tablespoon
of neutrons has the same mass as Mt. Everest. A supernova can expel as much radiation as the
Sun does its entire life~\cite{Sun:2007}.

\subsubsection{Quasar}
Some galaxies, although recent research suggests all, have a supermassive black hole at their 
centre. This black hole is surround by gas and dust which get thrown about and heated. This
heated gas is similar in appearance to a star and is actually called a quasar. These quasars
are know to eject gas high into the space of the galaxy and can be seen often as a blue jet
of gas coming from the centre.

\subsubsection{Black Hole}
Although we can not directly view black holes, we can see their effects. As previously stated,
black holes are often found at the centre of galaxies~\cite{Wik:2013} with super heated gas around them often
forming quasars. The Milky Way's black hole can be seen indirectly from released surveys. In
these surveys, objects are seen travelling very quickly around a central point in the galaxy and
being quickly ejected once they come to perigee. The point at which they orbit is thought
to be the black hole.

%----------------------------------------------------------------------------------------
%	BIBLIOGRAPHY
%----------------------------------------------------------------------------------------

\begin{thebibliography}{9}

\bibitem{Google:2013}
http://www.google.com/earth/index.html

\bibitem{Walker:2008}
Walker, J. Fundamentals of Physics, 8th ed., John Wiley and Sons, 2008

\bibitem{Harris:2003}
Harris, William E. (February 2003). "CATALOG OF PARAMETERS FOR MILKY WAY GLOBULAR CLUSTERS: THE DATABASE". Retrieved 2009-12-23.

\bibitem{Abruzzo:2009}
Abruzzo, Anthony J. The Origins of the Nebular Hypothesis – Or, the Genesis of a Theoretical Cul-de-sac The General Science Journal June 15;2009

\bibitem{Col:2007}
"The Grand Collision". The Sky At Night. November 5, 2007.

\bibitem{Plan:1997}
Hubblesite.org 1997

\bibitem{Sun:2007}
Giacobbe, F. W. (2005). "How a Type II Supernova Explodes". Electronic Journal of Theoretical Physics 2

\bibitem{Wik:2013}
http://en.wikipedia.org/wiki/Blackhole

\end{thebibliography}

\end{document}