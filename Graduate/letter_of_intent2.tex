%%%%%%%%%%%%%%%%%%%%%%%%%%%%%%%%%%%%%%%%%
% Plain Cover Letter
% LaTeX Template
%
% This template has been downloaded from:
% http://www.latextemplates.com
%
% Original author:
% Rensselaer Polytechnic Institute (http://www.rpi.edu/dept/arc/training/latex/resumes/)
%
%%%%%%%%%%%%%%%%%%%%%%%%%%%%%%%%%%%%%%%%%

%----------------------------------------------------------------------------------------
%	PACKAGES AND OTHER DOCUMENT CONFIGURATIONS
%----------------------------------------------------------------------------------------

\documentclass[11pt]{letter} % Default font size of the document, change to 10pt to fit more text

\usepackage{newcent} % Default font is the New Century Schoolbook PostScript font 
%\usepackage{helvet} % Uncomment this (while commenting the above line) to use the Helvetica font

% Margins
\topmargin=-1in % Moves the top of the document 1 inch above the default
\textheight=9in % Total height of the text on the page before text goes on to the next page, this can be increased in a longer letter
\oddsidemargin=-10pt % Position of the left margin, can be negative or positive if you want more or less room
\textwidth=6.5in % Total width of the text, increase this if the left margin was decreased and vice-versa

\let\raggedleft\raggedright % Pushes the date (at the top) to the left, comment this line to have the date on the right

\begin{document}

%----------------------------------------------------------------------------------------
%	ADDRESSEE SECTION
%----------------------------------------------------------------------------------------

\begin{letter}{} 

%----------------------------------------------------------------------------------------
%	YOUR NAME & ADDRESS SECTION
%----------------------------------------------------------------------------------------

\begin{center}
\large\bf Jordan Ell \\ % Your name
%\vspace{20pt} \hrule height 1pt % If you would like a horizontal line separating the name from the address, uncomment the line to the left of this text
2701 Gosworth Rd. Apt-303 \\ Victoria, British Columbia\\ (250) 415-9542 % Your address and phone number
\end{center} 
\vfill

\signature{Jordan Ell} % Your name for the signature at the bottom

%----------------------------------------------------------------------------------------
%	LETTER CONTENT SECTION
%----------------------------------------------------------------------------------------

\opening{To whom it concerns,} 

This letter will state my intent for graduate studies in Computer Science at the University of Victoria. My field of interest concerns the trade offs between a preventative approach to software engineering versus a curative approach and which of these methodologies, or combinations there of, produces the highest quality of software in stakeholder satisfaction.

The preventative methodologies of software engineering state that precautionary steps should be taken when possible to prevent potential issues or problems from arising later in a project. The curative methodologies of software engineering state that stakeholders should wait for problems to occur, then spend their energy in applying fixes to the existing real problems. Each of these methodologies has it positives and negatives. For instance, in preventative measures, you may stop many issues from occurring at all, but will you waste time creating mitigation strategies for problems which do no even occur. In curative measures, you may spend more time creating an end product faster, but you may run into a severely large blocking issue which could have been prevented if detected earlier. Finding the correct balance between preventative and curative approaches represents a crucial understanding of how software development should unfold in order to produce the highest quality results with the least effort spent.

I plan on investigating the relationships of preventative and curative approaches as well as how these measures are currently being performed in practice and what can be improved upon. Through this research, I hope to be able to provide a framework across the software development life cycle as to where effort should be spent by different stakeholders, in the context of prevention versus cure, in order to maximize quality output as viewed by stakeholder satisfaction.

My interest in this topic has come about through my previous work, as a masters student, with Dr. Daniela Damian and through the University of Victoria. I have worked on several projects with Dr. Damian before which encompass certain aspects of prevention versus cure and wish to continue my work as well as start new challenges with her at the University of Victoria.

\closing{Sincerely yours,}


%----------------------------------------------------------------------------------------

\end{letter}

\end{document}