%%%%%%%%%%%%%%%%%%%%%%%%%%%%%%%%%%%%%%%%%
% Plain Cover Letter
% LaTeX Template
%
% This template has been downloaded from:
% http://www.latextemplates.com
%
% Original author:
% Rensselaer Polytechnic Institute (http://www.rpi.edu/dept/arc/training/latex/resumes/)
%
%%%%%%%%%%%%%%%%%%%%%%%%%%%%%%%%%%%%%%%%%

%----------------------------------------------------------------------------------------
%	PACKAGES AND OTHER DOCUMENT CONFIGURATIONS
%----------------------------------------------------------------------------------------

\documentclass[11pt]{letter} % Default font size of the document, change to 10pt to fit more text

\usepackage{newcent} % Default font is the New Century Schoolbook PostScript font
%\usepackage{helvet} % Uncomment this (while commenting the above line) to use the Helvetica font

% Margins
\topmargin=-1in % Moves the top of the document 1 inch above the default
\textheight=9in % Total height of the text on the page before text goes on to the next page, this can be increased in a longer letter
\oddsidemargin=-10pt % Position of the left margin, can be negative or positive if you want more or less room
\textwidth=6.5in % Total width of the text, increase this if the left margin was decreased and vice-versa

\let\raggedleft\raggedright % Pushes the date (at the top) to the left, comment this line to have the date on the right

\begin{document}

%----------------------------------------------------------------------------------------
%	ADDRESSEE SECTION
%----------------------------------------------------------------------------------------

\begin{letter}{}

%----------------------------------------------------------------------------------------
%	YOUR NAME & ADDRESS SECTION
%----------------------------------------------------------------------------------------

\begin{center}
\large\bf Jordan Ell \\ % Your name
%\vspace{20pt} \hrule height 1pt % If you would like a horizontal line separating the name from the address, uncomment the line to the left of this text
2701 Gosworth Rd. Apt-303 \\ Victoria, British Columbia\\ (250) 415-9542 % Your address and phone number
\end{center}
\vfill

\signature{Jordan Ell} % Your name for the signature at the bottom

%----------------------------------------------------------------------------------------
%	LETTER CONTENT SECTION
%----------------------------------------------------------------------------------------

\opening{To whom it concerns,}

This letter will state my intent for graduate studies in Computer Science at the University of Victoria. My field of interest concerns the trade offs between a preventative approach to software engineering versus a curative approach and which of these methodologies, or combinations there of, produces the highest quality of software in stakeholder satisfaction.

I have begun this research topic in my masters studies at the University of Victoria with 2 submissions to the International Conference of Software Engineers in the conferences of Computer and Human Aspects of Software Engineering and the ACM Student Research Competition in 2013. At the Student Research Competition, I presented my work to a large audience and panel of judges in both formal presentations as well as poster board sessions, coming in 2nd place in the competition. I followed up these two submissions with a third extensive field study which will be submitted to the Foundations of Software Engineering conference in 2014. As a result of this third extensive study, I have identified this research topic as an extremely complex and difficult area of study, however, I feel very enthusiastic about this problem and am willing to take on the challenge. I feel that major contributions can be made to the field of software engineering by studying this problem and identifying solutions to the debate of prevention versus cure.

Previous work with Dr. Daniela Damian and the Software Engineering Global Interactivity Lab has allowed me to identify and study this problem through the industrial and academic networking they have been able to obtain. Further study in this lab and with Dr. Damian will be ideal because of the links to IBM which would allow me to further study this problem in industry with a reputable software vendor (IBM has also expressed interest in this research through my previous work).

Through this research, I plan to be able to provide a framework across the software development life cycle as to where effort should be spent by different stakeholders, in the context of prevention versus cure, in order to maximize quality output as viewed by stakeholder satisfaction. I have worked on several projects with Dr. Damian before which encompass many aspects of prevention versus cure, and wish to continue my work as well as start new challenges with her at the University of Victoria.

\closing{Sincerely yours,}


%----------------------------------------------------------------------------------------

\end{letter}

\end{document}