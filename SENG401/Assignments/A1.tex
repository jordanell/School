%%%%%%%%%%%%%%%%%%%%%%%%%%%%%%%%%%%%%%%%%
% Journal Article
% LaTeX Template
% Version 1.1 (25/11/12)
%
% This template has been downloaded from:
% http://www.LaTeXTemplates.com
%
% Original author:
% Frits Wenneker (http://www.howtotex.com)
%
% License:
% CC BY-NC-SA 3.0 (http://creativecommons.org/licenses/by-nc-sa/3.0/)
%
%%%%%%%%%%%%%%%%%%%%%%%%%%%%%%%%%%%%%%%%%

%----------------------------------------------------------------------------------------
%	PACKAGES AND OTHER DOCUMENT CONFIGURATIONS
%----------------------------------------------------------------------------------------

\documentclass[twoside]{article}

\usepackage{lipsum} % Package to generate dummy text throughout this template

\usepackage[sc]{mathpazo} % Use the Palatino font
\usepackage[T1]{fontenc} % Use 8-bit encoding that has 256 glyphs
\linespread{1.05} % Line spacing - Palatino needs more space between lines
\usepackage{microtype} % Slightly tweak font spacing for aesthetics

\usepackage[hmarginratio=1:1,top=32mm,columnsep=20pt]{geometry} % Document margins
\usepackage{multicol} % Used for the two-column layout of the document
\usepackage{hyperref} % For hyperlinks in the PDF

\usepackage[hang, small,labelfont=bf,up,textfont=it,up]{caption} % Custom captions under/above floats in tables or figures
\usepackage{booktabs} % Horizontal rules in tables
\usepackage{float} % Required for tables and figures in the multi-column environment - they need to be placed in specific locations with the [H] (e.g. \begin{table}[H])

\usepackage{lettrine} % The lettrine is the first enlarged letter at the beginning of the text
\usepackage{paralist} % Used for the compactitem environment which makes bullet points with less space between them

\usepackage{abstract} % Allows abstract customization
\renewcommand{\abstractnamefont}{\normalfont\bfseries} % Set the "Abstract" text to bold
\renewcommand{\abstracttextfont}{\normalfont\small\itshape} % Set the abstract itself to small italic text

\usepackage{titlesec} % Allows customization of titles
\renewcommand\thesection{\Roman{section}}
\titleformat{\section}[block]{\large\scshape\centering}{\thesection.}{1em}{} % Change the look of the section titles

\usepackage{fancyhdr} % Headers and footers
\pagestyle{fancy} % All pages have headers and footers
\fancyhead{} % Blank out the default header
\fancyfoot{} % Blank out the default footer
\fancyhead[C]{January 20, 2013} % Custom header text
\fancyfoot[RO,LE]{\thepage} % Custom footer text

%----------------------------------------------------------------------------------------
%	TITLE SECTION
%----------------------------------------------------------------------------------------

\title{\vspace{-15mm}\fontsize{24pt}{10pt}\selectfont\textbf{SENG401 Assignment 1}} % Article title

\author{
\large
\textsc{Jordan Ell}\\[2mm] % Your name
\normalsize University of Victoria \\ V00660306 % Your institution
\vspace{-5mm}
}
\date{}

%----------------------------------------------------------------------------------------

\begin{document}

\maketitle % Insert title

\thispagestyle{fancy} % All pages have headers and footers

%----------------------------------------------------------------------------------------
%	ARTICLE CONTENTS
%----------------------------------------------------------------------------------------

\begin{multicols}{2} % Two-column layout throughout the main article text

\section{Introduction}

\lettrine[nindent=0em,lines=3]{F} or this assignment I will be discussing the latest news
in regards to Aaron Swartz's conviction, trial, and ultimate suicide. Aaron Swartz was a computer program and political advocate for internet neutrality as well as the freedom of information. On January 6th, 2011, Swartz was arrested being accused of downloading academic journal articles which were normally protected by a pay wall and distributing them for free. Swartz was ultimately accomplishing this to defeat the publishers benefit over the pay wall in that they were reaping the rewards of the bought articles instead of the authors of said articles. On January 11, 2013, Swartz was found dead in his New York apartment. 

I will look at two sides of the coin for this assignment. First I will examine Swartz's ethical decisions throughout his computer programming career and secondly, I will evaluate the procedures used by Swartz's prosecutors during his arrest and trial. Swartz was a primary advocate for the open flow and exchange of information on the internet. This can be seen through his many acts early in his life. Swartz was involved in projects such as W3C, Reddit, Wikipedia, and Open Library sites. We can see early on in his life that perhaps above the notion of free information exchange, is the idea of \textit{free} information. Swartz believed that the internet can be used as a non greedy tool to promote thought across the globe and to keep ideas moving. This ultimately lead Swartz to be heavily involved in the campaign to prevent the passing of the Stop Online Privacy Act (SOPA) that involved internet monitoring and a generally more closed internet for society. Had SOPA passed, the information we take for granted on the internet may be harder to acquire. Again, Swartz is demonstrating his position of the freedom of information to be readily available and in cases free on the internet. 

Swartz's passion for availible and free (or rightly paid) information ultimately led him to download 4 million academic journal articles from JSTOR's online collections which are normally guarded by a pay wall. This pay wall was circumnavigated by connecting to JSTOR from MIT which has an agreement to acquire articles for free from the website. Swartz's claimed his motives were due to the fact that the pay wall did not pay the authors of the articles but rather the publishers which does not conform to his notion of free or rightly paid information. From his point of view, the money, if any, should have been paid largely to the authors of the articles to promote the ideas and to incentivize more research and articles in the future. 

Ultimately I agree with Swartz's notion of free or properly paid information on the internet. The reason the internet works is because of the freedom of information on it. Having the information be free or paid to support the ideas is also ultimately the driving force of society. Why would people be willing to put time and effort into ideas and research if they cannot promote change with their ideas by having them widely available for the public or to have incentives to continue their research (pay the authors).

While governments try to crack down on illegal activities on the internet through means such as SOPA, the punishments for illegal internet activities also becomes higher. This can be directly seen with the Aaron Swartz's conviction and prosecution. At the time of arrest, Swartz was charged with four-count indictment. This could be understandable what Swartz had done was strictly illegal, but it should be noted that Swartz merely used an agreement with MIT and JSTOR to download articles for free. The governments thoughts towards cyber crime can already be seen in this prosecution of a grey area. Next, the prosecution changed its charges to a 13-count indictment that carried a 35 year sentence all while prosecuting the same actions. The governments actions here are clear cut in with sending a message to potential future cyber crime hackers. 

The governments prosecution leads a clear cut trail in their moral and ethical though process when it comes to internet security as well as crime. Even for this grey area "crime" where a programmer used a known agreement to gain access to free information, they have zero tolerance. It should be noted that Swartz never distributed the articles that he collected, although the prosecution states that he intended to distribute them freely on the internet. 

Here, I cannot agree with Swartz's prosecutions ethical actions at all. First, the prosecution with heavy charges for such a minor crime, if it even was a crime at all. Next, the escalation of the charges under the same actions taken by the defendant. And finally, and what is the most disturbing, threatening an open information supported who likely accomplished more than anyone for free internet information with a 35 year jail sentence which could have ultimately led to his suicide. In my opinion, the U.S governments strive to "set an example" of internet crime punishment is what ultimately led them to such drastic measures against this "criminal" and his suicide. I believe it is never ethical to put one person's life in jeopardy (jail or death) in order to create a tighter grip or control structure on the society being governed. 

To recap, in my opinion, Aaron Swartz acted more ethically than his prosecution which accused him of being a criminal. Perhaps this is a downfall of the judicial system today especially when it comes to technical crimes. Judges and jurymen do not have the knowledge to either accuse or acquit their fellow man because they lack the understanding of the underlying technical structures for which these "crimes" are supposedly being committed with or on. Perhaps, these prosecutions happening against individuals such as Aaron Swartz are more about societal control rather than attempting to create better technological presence in society which can further mans drive towards understanding.

%----------------------------------------------------------------------------------------
%	REFERENCE LIST
%----------------------------------------------------------------------------------------

\begin{thebibliography}{99} % Bibliography - this is intentionally simple in this template

\bibitem{}
http://www.salon.com/2013/01/16/aaron\_swartz\_
reveals\_the\_hypocrisy\_of\_our\_justice\_department/ 

\bibitem{}
Infogami, archived from the original on 2007-12-24

\bibitem{}
Swartz, Aaron. "Sociology or Anthropology". Raw Thought. Retrieved 2013-01-16.

\bibitem{}
"Feds: Harvard fellow hacked millions of papers". Associated Press. Yahoo! News. Retrieved 2013-01-15.
 
\end{thebibliography}

%----------------------------------------------------------------------------------------

\end{multicols}

\end{document}
