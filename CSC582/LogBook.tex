% This will be the main document for the Technical Networks paper to
% be written by the Eggnet team of Jordan Ell, Triet Huynh and Braden
% Simpson in association with Adrian Schroeter and Daniela Damian.

\documentclass[conference]{IEEEtran}

% Use of outside images
\usepackage{graphicx} 
% Use text inside euqations
\usepackage{amsmath}

\usepackage{balance}
\usepackage{float}
\floatstyle{plaintop}
\restylefloat{table}

% Correct bad hyphenation here
\hyphenation{op-tical net-works semi-conduc-tor}

% Begin the paper here
\begin{document}


% Paper title
% Can use linebreaks \\ within to get better formatting as desired
\title{Theoretical Models and Applications: Log Book}

% Authors names
\author{\IEEEauthorblockN{Jordan Ell V00660306}
\IEEEauthorblockA{University of Victoria,
Victoria, British Columbia, Canada \\ jell@uvic.ca}
}

% Make the title area
\maketitle


\begin{abstract}
Software systems have not only become larger over time, but the amount of
technical contributors and dependencies have also increased. With these expansions also comes
the increasing risk of introducing a software failure into a pre-existing system.
Software failures are a multi-billion dollar problem in the industry today and while integration and
other forms of testing are helping to ensure a minimal number of failures, research to understand
full impacts of code changes and their social implications is still a major concern. This paper describes
how analysis of code changes and the technical relationships they infer can be used to detect pairs
of developers whose technical dependencies may induce software failures. These developer pairs may
also be used to predict future software failures as well as provide recommendations to contributors
to solve these failures caused by source code changes.
\end{abstract}


\section{May 2nd 2013}

The main topic of this class was to discuss how to identify high impact papers. Some of
the ideas that members of the class had to identify high impact papers are as follows.
Looking at the number of citations a paper has is a good indication of how well the paper
is known in its own field or how important it might be. Another is to look for prize winning
papers. This can be large prizes such as the Turing award or even the best paper of a conference.
This is a good way to identify a paper that has high impact potential as it is important at
the time of the award. Another good way to find a high impact paper is to look at industry impact.
Here if a paper involves the creation of a real world product which generates revenue it
may be a high impact paper or if it can change existing products in the real world.

I had an idea for identifying central papers but did not mention it in class. My idea is a 
similar idea to that of industry impact and it is media impact. If a simple Google search
can turn up a lot of results that all point to a single paper that has achieved main media 
success, it may be a high impact paper. This ties in to industry impact as the media 
may find out about it through industry experience. I plan on trying this method in the 
next assignment.

Another main area that was described this class was the discussion of what can you identify
as a theory paper or a paper that has theory at its roots. It seems to me that theory is a
pretty general term and as long as there is some underlying theoretical model to it. This
can be some type of NP-Complete issue all the way to graph models.

The assignment for next class is to find 1 or 2 high impact papers with an underlying theory
background. This is due May 6th.


\section{May 4th 2013}
Today I looked up my two high impact papers for class on May 6th. I took the approach
mentioned earlier in my log book. That is to say that I went for media and industry
impact rather than academic impact of the paper. As my interest for this course was
to do with networks, I started my search on Google looking for network impacting papers.

To do this, I used Google to search for `Coded TCP'. I knew in advanced that encoding
protocol packets can be used to speed up internet connections both wired and wireless,
and I knew that TCP is the highest used protocol for consumer use. This search landed me
on many blog and news pages that involved a group of researchers at MIT. The group found
a way of using linear equations to solve for packets that are lost in transmission instead
of requesting that the packets get resent. This immediately grabbed my attention as it
contains a theory background and has obviously been large enough to attract media.

I found that this new method has been implemented at MIT and has increased their internal
networks speeds 16x. The paper that found this method is relatively new (2011), so it has
little citations (44). However, I believe this paper can play a major role in the industry
as the result of network speed increase requires no physical upgrade to existing networks.
I have identified this paper, and its continuation paper, as high impact.


\section{May 6th 2013}

Recap from last class: Look for highly cited paper, award winning papers, ask your supervisor, 
and so forth when looking for high impact theory papers.

Theory papers: A paper to find when bugs get introduced into a project.
A paper that provides a new algorithm that determines how well a drug molecule will work. The
first two papers have very high number of citations. The chemistry papers are foundational 
papers in their field and found through Google Scholar. My paper on network packet loss using
linear equations. Another paper on projected plains was found using survey papers which is 
a potential new way of finding high impact papers. High citations seems to be the most used
criteria for high impact papers so far. A paper on approximations to NP-Complete problems and
how they can be used. It is a Godell prize winning paper. Dijkstra prize winning paper was 
also said. A paper on performance analysis of networks. This paper was found by looking at
industry impact. A paper on music retrieval says that the award winners in this field were not
high impact even though they won. This paper has a high number of authors and also cites several
other papers. A paper involving k-nearest neighbors for image searching. A paper that is
central to social interactions in software engineering is now used in almost all fields of
software engineering that involve communication. Found from experience, but also has 500
citations. Another paper on indexing file structure to create clusters which is the theory 
component. This paper again had lots of citations. A paper on dynamic programming from 1956 with
a citation count of more than 11,000! Very old papers with high citation count are extremely
foundational in their field.

It seems like most people today went for high citations over anything for identifying
high impact papers. I personally thing for a paper to be high impact it has to have
two components. First is has to lay foundations for future research in a broad sense.
For example, the first paper to explain the grand unified theory in physics allowed for
a whole field of cosmology in the field. Yes it does have high citations but it
also allows for a large amount of future work while also answering current questions.
Secondly, I think high impact papers should have real world implications. This could
be a paper on vacuum tubes which led to the first vacuum tube computer. This is a real impact
that can help humanity.

We broke into groups and started to discuss our papers. My team has 3 students in it 
but only had 2 papers for today. The first paper was about dynamic programming and speech
recognition software. This paper has over 11,000 citations! This paper lays foundational
work that is being used in industry today with Apple and Google through their speech
recognition software. The second paper was mine and it was about packet loss in TCP 
connections. This papers solves the problem of packet loss by using linear equations
to solve for missing packets. We are choosing to use the dynamic programming paper
as it has been around longer and has a larger impact (as of now) on the software
industry than the packet loss paper which is newer and still has smaller impacts
on industry.

\section{May 6th 2013}

\bibliographystyle{IEEEtran}
\balance
\bibliography{paper}


% End of the paper
\end{document}
