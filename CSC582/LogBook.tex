%%%%%%%%%%%%%%%%%%%%%%%%%%%%%%%%%%%%%%%%%
% Journal Article
% LaTeX Template
% Version 1.1 (25/11/12)
%
% This template has been downloaded from:
% http://www.LaTeXTemplates.com
%
% Original author:
% Frits Wenneker (http://www.howtotex.com)
%
% License:
% CC BY-NC-SA 3.0 (http://creativecommons.org/licenses/by-nc-sa/3.0/)
%
%%%%%%%%%%%%%%%%%%%%%%%%%%%%%%%%%%%%%%%%%

%----------------------------------------------------------------------------------------
%	PACKAGES AND OTHER DOCUMENT CONFIGURATIONS
%----------------------------------------------------------------------------------------

\documentclass[twoside]{article}

\usepackage{lipsum} % Package to generate dummy text throughout this template

\usepackage[sc]{mathpazo} % Use the Palatino font
\usepackage[T1]{fontenc} % Use 8-bit encoding that has 256 glyphs
\linespread{1.05} % Line spacing - Palatino needs more space between lines
\usepackage{microtype} % Slightly tweak font spacing for aesthetics

\usepackage[hmarginratio=1:1,top=32mm,columnsep=20pt]{geometry} % Document margins
\usepackage{multicol} % Used for the two-column layout of the document
\usepackage{hyperref} % For hyperlinks in the PDF

\usepackage[hang, small,labelfont=bf,up,textfont=it,up]{caption} % Custom captions under/above floats in tables or figures
\usepackage{booktabs} % Horizontal rules in tables
\usepackage{float} % Required for tables and figures in the multi-column environment - they need to be placed in specific locations with the [H] (e.g. \begin{table}[H])

\usepackage{lettrine} % The lettrine is the first enlarged letter at the beginning of the text
\usepackage{paralist} % Used for the compactitem environment which makes bullet points with less space between them

\usepackage{abstract} % Allows abstract customization
\renewcommand{\abstractnamefont}{\normalfont\bfseries} % Set the "Abstract" text to bold
\renewcommand{\abstracttextfont}{\normalfont\small\itshape} % Set the abstract itself to small italic text

\usepackage{titlesec} % Allows customization of titles
\renewcommand\thesection{\Roman{section}}
\titleformat{\section}[block]{\large\scshape\centering}{\thesection.}{1em}{} % Change the look of the section titles

\usepackage{fancyhdr} % Headers and footers
\pagestyle{fancy} % All pages have headers and footers
\fancyhead{} % Blank out the default header
\fancyfoot{} % Blank out the default footer
\fancyhead[C]{Log Book $\bullet$ May - August 2013} % Custom header text
\fancyfoot[RO,LE]{\thepage} % Custom footer text

%----------------------------------------------------------------------------------------
%	TITLE SECTION
%----------------------------------------------------------------------------------------

\title{\vspace{-15mm}\fontsize{24pt}{10pt}\selectfont\textbf{Theoretical Models and Applications: Log Book}} % Article title

\author{
\large
\textsc{Jordan Ell}\\[2mm] % Your name
\normalsize University of Victoria \\ % Your institution
\normalsize \href{mailto:jell@uvic.ca}{jell@uvic.ca} % Your email address
\vspace{-5mm}
}
\date{}

%----------------------------------------------------------------------------------------

\begin{document}

\maketitle % Insert title

\thispagestyle{fancy} % All pages have headers and footers

%----------------------------------------------------------------------------------------
%	ABSTRACT
%----------------------------------------------------------------------------------------

\begin{abstract}

\noindent \lipsum[1] % Dummy abstract text

\end{abstract}

%----------------------------------------------------------------------------------------
%	ARTICLE CONTENTS
%----------------------------------------------------------------------------------------

\begin{multicols}{2} % Two-column layout throughout the main article text

\section{May 2nd 2013}

The main topic of this class was to discuss how to identify high impact papers. Some of
the ideas that members of the class had to identify high impact papers are as follows.
Looking at the number of citations a paper has is a good indication of how well the paper
is known in its own field or how important it might be. Another is to look for prize winning
papers. This can be large prizes such as the Turing award or even the best paper of a conference.
This is a good way to identify a paper that has high impact potential as it is important at
the time of the award. Another good way to find a high impact paper is to look at industry impact.
Here if a paper involves the creation of a real world product which generates revenue it
may be a high impact paper or if it can change existing products in the real world.\\

I had an idea for identifying central papers but did not mention it in class. My idea is a 
similar idea to that of industry impact and it is media impact. If a simple Google search
can turn up a lot of results that all point to a single paper that has achieved main media 
success, it may be a high impact paper. This ties in to industry impact as the media 
may find out about it through industry experience. I plan on trying this method in the 
next assignment.\\

Another main area that was described this class was the discussion of what can you identify
as a theory paper or a paper that has theory at its roots. It seems to me that theory is a
pretty general term and as long as there is some underlying theoretical model to it. This
can be some type of NP-Complete issue all the way to graph models.\\

The assignment for next class is to find 1 or 2 high impact papers with an underlying theory
background. This is due May 6th.


\section{May 4th 2013}


\section{May 6th 2013}

Recap from last class: Look for highly cited paper, award winning papers, ask your supervisor, 
and so forth when looking for high impact theory papers.\\

Theory papers: A paper to find when bugs get introduced into a project.
A paper that provides a new algorithm that determines how well a drug molecule will work. The
first two papers have very high number of citations. The chemistry papers are foundational 
papers in their field and found through Google Scholar. My paper on network packet loss using
linear equations. Another paper on projected plains was found using survey papers which is 
a potential new way of finding high impact papers. High citations seems to be the most used
criteria for high impact papers so far. A paper on approximations to NP-Complete problems and
how they can be used. It is a Godell prize winning paper. Dijkstra prize winning paper was 
also said. A paper on performance analysis of networks. This paper was found by looking at
industry impact. A paper on music retrieval says that the award winners in this field were not
high impact even though they won. This paper has a high number of authors and also cites several
other papers. A paper involving k-nearest neighbors for image searching. A paper that is
central to social interactions in software engineering is now used in almost all fields of
software engineering that involve communication. Found from experience, but also has 500
citations. Another paper on indexing file structure to create clusters which is the theory 
component. This paper again had lots of citations. A paper on dynamic programming from 1956 with
a citation count of more than 11,000! Very old papers with high citation count are extremely
foundational in their field.\\

It seems like most people today went for high citations over anything for identifying
high impact papers. I personally thing for a paper to be high impact it has to have
two components. First is has to lay foundations for future research in a broad sense.
For example, the first paper to explain the grand unified theory in physics allowed for
a whole field of cosmology in the field. Yes it does have high citations but it
also allows for a large amount of future work while also answering current questions.
Secondly, I think high impact papers should have real world implications. This could
be a paper on vacuum tubes which led to the first vacuum tube computer. This is a real impact
that can help humanity.\\

%----------------------------------------------------

%----------------------------------------------------------------------------------------
%	REFERENCE LIST
%----------------------------------------------------------------------------------------

\begin{thebibliography}{99} % Bibliography - this is intentionally simple in this template

\bibitem[Figueredo and Wolf, 2009]{Figueredo:2009dg}
Figueredo, A.~J. and Wolf, P. S.~A. (2009).
\newblock Assortative pairing and life history strategy - a cross-cultural
  study.
\newblock {\em Human Nature}, 20:317--330.
 
\end{thebibliography}

%----------------------------------------------------------------------------------------

\end{multicols}

\end{document}
