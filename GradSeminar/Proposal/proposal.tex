% This will be the main document for the Technical Networks paper to
% be written by the Eggnet team of Jordan Ell, Triet Huynh and Braden
% Simpson in association with Adrian Schroeter and Daniela Damian.

\documentclass[conference]{IEEEtran}

% Use of outside images
\usepackage{graphicx} 
% Use text inside euqations
\usepackage{amsmath}

\usepackage{float}
\floatstyle{plaintop}
\restylefloat{table}

% Correct bad hyphenation here
\hyphenation{op-tical net-works semi-conduc-tor}

% Begin the paper here
\begin{document}


% Paper title
% Can use linebreaks \\ within to get better formatting as desired
\title{Indirect Conflicts: A Research Proposal}

% Authors names
\author{\IEEEauthorblockN{Jordan Ell}
\IEEEauthorblockA{University of Victoria,
Victoria, British Columbia, Canada \\ jell@uvic.ca}
}

% Make the title area
\maketitle


\section{Proposal}
%Introduction
As Software Configuration Management (SCM) systems have grown over the years, their maturity has allowed parallel development
practices to become the norm in software development. With this parallel development comes the need for larger awareness among developers
to have ``an understanding of the activities of other which provides a context for one's own activities''~\cite{Dourish:1992:ACS}. This
added awareness mitigates some negative aspects of parallel development which include the cost of conflict prevention and resolution,
however, in practice we see these mitigated losses continue to exist and occur frequently. Indirect conflicts are one of the parallel 
development downsides in that as once developer changes code inside a project, it may affect another developer's code at a different
location within the same code base or to an external code base. Many researchers, myself included, have taken on the task to study and
create awareness mechanisms for software developers in order to help them either prevent, catch, or resolve indirect conflicts over the
lifetime of a project. However, the resulting tools of academic pursuit have attracted little interest from developers and have had
many common issues such as information overload, high false 
positive rates~\cite{Sarma:2007:TSA}~\cite{Biehl:2007:FVD}~\cite{Sarma:2009:TIV}~\cite{Servant:2010:CPI}.

% Methodology
This research aims at creating an understanding as to why our mitigation attempts in indirect conflict prevention and resolution have
not had as large a success as was initially thought to occur by understanding the nature of indirect conflicts, studying how real world
developers currently handle the prevention or resolution of indirect conflicts, and by looking at current tools used. These new understandings
will help us fill in the gaps presented by previous research in this area.
In my research methodology, I will use expert software developer experience
and opinion to analyze what has become the downfall of indirect conflicts mitigation tools as well as the state of the research surrounding
indirect conflicts. While many researchers have conducted small sample case studies of their indirect conflict research, few have
produced follow up academic papers as to why repetitive problems continue to occur in the body of work. By conducting semi structured interviews,
I hope to fill this gap with a large body of developer insight and opinion. In order to understand how indirect conflict issues occur across
the whole spectrum of software development, I have chosen to conduct interviews from a mix of software companies which differ from waterfall
development, agile development, API creation, software as a service, open source, closed source, databases management, and more. This wide
spectrum allows this research to become highly generalizable and applicable not only to academics, but to real world industry professionals.

%Results
In the early stages of my research (Summer 2013) regarding indirect conflicts, I have conducted and analyzed interviews with 19 industry
professionals from a large collection of software companies which includes IBM, Microsoft, and Amazon. Findings have begun to indicate a 
disjoint between the academic study of indirect conflicts and the real world application of techniques and practices to prevent and
resolve such conflicts. These finding have primarily indicated that developers approach indirect conflicts with a curative thought process,
being that of only dealing with issues after they arise, oppose to a preventative measure which is more sought after in academia, being that
of preventing indirect conflicts before they occur at the risk of preventing conflicts which may never arise. In the continuing steps in my
research, I will conduct two larger studies. The first will be to continue the exploration of the ideas of prevention versus cure methodologies
in regards to indirect conflicts. By conducting large controlled experiments supplemented by qualitative developer interviews, I hope to
discover how and when prevention techniques or curative techniques may be better than the other. Second, I will create an awareness mechanism
which takes the form of a methodology, practice, or development tool which will be best suited towards developer's needs and the results 
found in the previous research. I will validate this mechanism through industry officials using and examining the final product for its
merits in the software development industry. 

%Research contribution
This proposed research has significant impact, not only on academia, but on industry as well. Through the exploration of indirect conflicts,
we will provide researchers with a clear outline for what has been done wrong in the past, but also as to how we should, as an academic
community, focus our efforts moving forward to better serve those which will benefit from indirect conflict research (industry). These
efforts, as anticipated, will be the study of prevention versus cure techniques for indirect conflict management. Industry
will benefit from the resulting processes or tools to come out of this research, in that they will know better how to manage their time effectively
by preventing or curing indirect conflicts. Industry should see better production from software developers, as well as an overall improvement
in software quality. 

\bibliographystyle{IEEEtran}
\bibliography{proposal}


% End of the paper
\end{document}